\documentclass[fleqn,12pt]{article}

\usepackage[margin=15mm]{geometry}
\usepackage[utf8]{inputenc}
\usepackage[bulgarian]{babel}
\usepackage[unicode]{hyperref}
\usepackage{amsfonts}
\usepackage{amssymb}
\usepackage{indentfirst}
\usepackage{enumitem, hyperref}
\usepackage{blindtext}
\usepackage{multicol}

\usepackage{amsmath}
\DeclareMathOperator{\cotg}{cotg}
\DeclareMathOperator{\LCS}{LCS}
\DeclareMathOperator{\longer}{longer}

\title{Бази от данни. Релационен модел на данните}
\author{v1.0}
\date{14 юни 2021}

\begin{document}

\maketitle

\tableofcontents

\section{Релационен модел}
    Релационния модел е модел за съхраняване и обработка на данни. При тези модели данните се представят под формата на кортежи и групирани спрямо взаимотношенията помежду им. Това е графичен модел използващ кутийки и стрелки, за да опише основните елементи и техните връзки. Релационните модели съдържат три основни типа елементи: обекти, атрибути и релации. 

    Релационите модели може да се изгради и чрез обектно ориенирани дизайн чрез UML. UML се използват в комплексни проблеми, където базата от данни не е в центъра.


\subsection{Обекти/Entities}
    Обектите описват елемент от реалния свят и могат да бъдат уникално идентифицирани. Те са вид абстракция, а колекция от подобни обекти образуват entity sets. Entity set наподобява идеята за обектно оринетирани класове. Обектие в релационите модели са статични, така че не се очаква да има методи свързани с тях, заразлика от класовете.

    Обектите притежават атрибути, които ги различават и взаимотношения с други обекти в модела.


\subsubsection{Subclasses}
    Може да имаме обекти, които притежават специални атрибути, които не првят смисъл за всички обекти от entity set-a. В такива случаи може да се използва subclasses, всеки със собствените си специални атрибути и връзка с entity set-a, чрез isa релация. isa връзките са специални връзки и всяка връзка е от тип едно към едно. Като цяло isa връзките наподобяват наследяване в обектно ориентирани модели.


\subsection{Атрибути}
    Атрибутите описват характеристиките на обектите. Те служат за различаване на различни обекти в entity set.

    Атрибутите могат да бъдат атомарни типове (числа, стрингове), структури с фиксиран брой атомарни променливи или множество от стойности от един тип - атомарен или структура (масив)


\subsection{Релация}
    Релациите наричаме взаимотношенията между entity sets. Te свръзват два или повече entity sets. Бинарните връзки са най-често срещаните, но релациония модел позволява връзки между повече от два сета.

    Релациите представляват множество от кортежи (редове от таблицата на база от данни), които трябва да имат еднакви атрибути. Обикновенни се бележи като $R(A_1, A_2, .... A_n)$, където А е името на релацията, а $A_1, A_2, ... A_n$ са атрибутите на релацията

    Даден обект може да участва в релация колкото пъти е необходимо. Обекта участва под различни роли в една и съща връзка.

    Понякоа е удобно релациите да притежават свои собствени атрибути. Това са атрибути, които не принадлежат на никое от участващите обекти, а на самата връзка помежду им. 


\subsubsection{Бинарна релация}


\begin{itemize}
	\item много към един (едни към много е същата връзка, просто стрелката е в различна посока)
	\item един към един 
	\item много към много
\end{itemize}

    Трябва да се отбележи, че връазки от тип много към едно е специфичен случай на много към едно, а едно към едно е специфичен случай съответно на много към едно. Това означава, че всичко, което може да се използва за връзка много към много може да се използва за връзки много към един, макар и обратното да не е възможно.


\subsubsection{Многопосочна релация}
    Тройни или релации с повече участници са редки, но са необходими, за да може даден релационен модел да отговаря по-точно на реалността. В схемата на релация диаманта на релационата връзка е свързан към всики участващи обекта 

    Някой езици, като ODL (Object Definition Language), не позволява изобразяването на многопосочни реации, затова такива релации, могат да бъдат преобразувани до колекция от бинарни релации от тип много към едно. Това може да направим представяйки нов entity set, който представлява кортежи от връзките от многопосочната релация. Този сет се нарича свързващ сет и към него се свързват останалите обекти участващо в релацията.


\subsection{Ограничения}
    Ограничения върху обектите се използват за да изразят аспекти, които не могат да се опишат просто чрез обекти, атрибути и релациите между тях. Ограниченията се разделят в следните категории:

\begin{itemize}
    \item ключове - група атрибути в обект уникални стойности и служат за идентификация на обектите в даден сет. 
    \item ограничения за уникалност на стойносттта - ключовете са голяма част от тези ограничения, но има и други аспекти, при които може да искаме даден атрибут на сет от обекти да е уникален за всеки обект.
    \item ограничения за реферирани обекти - това е ограничения, че един рефериран обект съществува
    \item ограничения зависими от домейна - ограничения за даден атрибут, чиято стойност трябва да е в някакъва числова област или да е част от сецифично определтн сет
    \item общи ограничения
\end{itemize}


\subsection{Слаби сетове от обекти}
В случаи, когато ключа на един сет се състои от няколко атрибути, един или всичките, от които принадлежат на друг сет от обекти, то този сет наричаме слаб. За да може сет Е да бъде определен като слаб трябва:

\begin{itemize}
	\item Ключа на сета се състои от нула или няколко от собствените му полета
	\item Ключа се допълва с атрибути от други сетове, които се достигат чрез определени много към едно релации от Е. Тези релации се наричат поддържащи релации. Ако релация свързва Е с поддържащия му сет F, то атрибутите, които F предоставя към ключа на Е, то те трябва да са ключове и за F. Ако F e слабо се добавят атрибутите от друг сет G, към който е свързан F.
\end{itemize}


\subsection{Кортеж}
Кортеж е наредена n-торка. В релационния модел релациите се изразяват с такива наредени n-торки следователно кортежите трябва да са уникални. 


\subsection{Схема на релация}
Схема на реалция се определя от името на релацията и нейните атрибути. 


\subsection{Схема на релационна база от данни}
База от данни представлява съвкупност от множество релации. Схема на релационна база от данни е съвкупността от всички релационни схеми на всички релации

Схемата на релационна база от данни често се изразява чрез диаграма, за по лесно визоализиране и разбиране. Диаграмата на релационна база данни представлява граф. Върховете на този граф са представени от entity sets, атрибути и връзките. Всеки един от тези елементи представяме с различна форма:
\begin{itemize}
	\item обектите - правоъгълници
	\item атрибути - овали
	\item връзки - диамант (this shape <>)
	\item isa релация - триъгълник
	\item слаби сетове - правоъгълник с двойна граница, съотвените им поддържащи релации са диаманти с двойна граница. Ако сет предоставя някакви атрибути към ключа, то те трябва да са подчертани.
\end{itemize}

Ребрата свързват разлличните обекти.


Екземпляр на релационна база данни наричаме набор от данни, описани от дадена диаграма(схема) на релационна база от данни. (С две думи това е snapshot на базата в определен момент)


\subsection{Реализация на релационна база от данни}
Реализацията на дадена релационна база се състои от няколко основни принципа:

\begin{itemize}
	\item Вярност - На първо място дизайна трябва да е верен спрамо спецификациите на приложението. Колекциите от обекти и техните атрибути трябва да отговарят на истинския свят.
	\item Избягване на повторения - Добре прокетирана база не трябва да има повторения на информацията. Наличието на повторения означава, че дадена информация ще трябва да се пази на повече от едно място (заема повече памет). В допълнение при ъпдейтване на данните на базата, тя може да бъде оставена в невалидно състояние (повторените данни да не са ъпдейтнати всичките)
	\item Простота - Схемата трябва да е възможно най-проста. Не бива да се представят повече елементи към дизайна, отколкото са необходими
	\item Подбиране на правилните връзки - Въпреки, че обектите могат да са свързани по различни начини, трябва да се подберат само необходимите връзки. Създаването на всички възможни връзки между сетовете може да доведе повторения и създадената база ще има нужда от повече памет.
	\item Подбиране на правилните елементи - Опциите за това как да представим елементи, представящи реалния свят, най-често са между използването на атрибути или комбинация на колекции от обекти и връзките помежду им. Като цяло моделирането на данни като атрибути е по-просто, но превръщането на всичко в атрибути също създава множество проблеми.
	
    Ако имаме колекция Е условията, при които е по-подходящо да се използва атрибут вместо тази колекция са:
	\begin{itemize}
        \item Всички връзки, в които участва Е, трябва от тип много към едно, където Е е едното. 
        \item Атрибутите на Е трябва всичките да са част от ключа, всичките да са зависими помежду си.
        \item Няма връзки, в които Е участва повече от веднъж.
    \end{itemize}

    Ако Е отговаея на тези условия, тази колекция може да бъде премахната по следния начин:
    \begin{itemize}
        \item Ако съществува релация много към едно R от F към Е, релацията може да бъде премахната, а атрибутите на Е се превръщат в атрибути на F
        \item Ако има многопосочна релация R, в която участва Е, атрибутите на Е трябва да се превърнат в атрибути на релацията R.
    \end{itemize}
\end{itemize}


\subsection{Видове операции върху релационната база данни. Заявки към релационната база данни}
    \textbf{Data Definition Language(DDL)} е програмен език за създаване и модифициране на обектите на бази от данни или индекси. Команди, които участват в DDL езици са CREATE, ALTER и DROP.

    \textbf{Data Manipulation Languafe(DML)} e програмен език за извличане, добавяне, изтриване и модифициране на таблици на базите от данни. Операции, които са част от DML са SELECT, INSERT, UPDATE и DELETE. 


\section{Релационна алгебра}

\subsection{Основни операции}

\begin{itemize}
    \item Обединенние - бинарна операция, комутативна, асоциативна операция
    \item Сечение - бинарна, некомутативна операция
    \item Рзлика - бинарна, комутативна, асоциативна операция
	\item Проекция унарна операция, която на дадена релация съпоставя друга релация, но без определени атрибути (като SELECT в SQL). Пример: $\pi_{[attribute list]}(R)$
    \item Селекция - филтрира релация, така че да останат само кортежи, отговарящи на дадени условия (като WHERE в SQL). Пример: $\sigma_{[predicate]}(R)$
	\item Преименуване - сменя имената на атрибутите на кортежа в дадена релация. Не сменя типа на атрибутите, само имената им (като AS в SQL). Пример: $\rho_{[new names]}(R)$
	\item Декартово произведения - всеки кортеж от едната релация се закеоя за всеки кортеж на друга релация. Пример: $R_1 \times R_2$
\end{itemize}


\subsection{Допълнителни операции}

\begin{itemize}
	\item Theta join - декартово произведение със селекция; след изпълнението на декартовото произведение остават само кортежите, които отговарят на условието
	\item Equijoin - чатен случай на theta join, когато условието на съединението включва само съвпадение по атрибути
    \item Natural join - разширение на Equijoin, където съединението се случва по всички атрибути с еднакви имена. $R_1 \bowtie  R_2$. 
    \item Групиране (GROUP BY) $\gamma_{[attribute to group][aggregate function list]}(R)$
    \item Сортиране (ORDER BY) $\tau_{[atribute list]}(R)$
    \item Запазване само на уникални релации (DISTINCT) $\delta(R)$
\end{itemize}


\end{document}