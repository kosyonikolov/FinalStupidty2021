\documentclass[fleqn,12pt]{article}

\usepackage[margin=20mm]{geometry}
\usepackage[T2A]{fontenc}
\usepackage[utf8]{inputenc}
\usepackage[english,bulgarian]{babel}
\usepackage[document]{ragged2e}
\usepackage{amsthm}
\usepackage{amssymb}
\usepackage{mathtools}
\usepackage{hyperref}

\hypersetup{
    colorlinks,
    linktoc=all,
    citecolor=black,
    filecolor=black,
    linkcolor=black,
    urlcolor=black
}

\newtheorem*{Th}{Теорема}
\newtheorem*{Lem}{Лема}
\newtheorem*{Def}{Дефиниция}
\newtheorem*{Claim}{Твърдение}

\renewcommand\qedsymbol{$\blacksquare$}

\title{finals-theme-25-algebra}
\author{ivan ch}
\date{April 7 2021}

\begin{document}

\maketitle

\tableofcontents

\begin{justify}

\section{Предварителни дефиниции}
\begin{Def}
    Нека $a_1, a_2, \dots, a_n$ са вектори от линейно пространство V над полето F. Казваме, че тези
    вектори са ЛНЗ(линейно независими), ако единствената линейна комбинация, задаваща нулевия вектор
    $0$ е нулевата, т.е. за $\lambda_1, \dots, \lambda_n \in V : \lambda_1 a_1 + \lambda_2 a_2 + \dots +
    \lambda_n a_n = 0 \Rightarrow \lambda_i = 0, i = 1,\dots,n$.\\
    Една система с безбройно много вектори наричаме ЛНЗ, ако всяка нейна крайна подсистемат вектори е ЛНЗ система от
    вектори
\end{Def}
\begin{Def}
    Нека $a_1,a_2,\dots,a_n$ са вектори от линейно пространство V над полето F. Казваме, че тези
    вектори са ЛЗ(линейно зависими), ако $\exists(\lambda_1,\lambda_2,\dots,\lambda_n), \lambda_i \in
    F$, такива че има поне едно $\lambda_i \neq 0$, но въпреки това \\
    $\lambda_1 a_1 + \lambda_2 a_2 + \dots + \lambda_n a_n = 0$.
\end{Def}
\begin{Def}
    Нека V е линейно пространство над числово поле F и $A \subset V$. Множеството $l(A)$, състоящо
    се от всички линейни комбинации на елементи от А с коефициентите от F ще наричаме линейна
    обвивка на множеството A.
\end{Def}

\section{Определения за базис, размерност и коодинати}
Нека V $ \neq \{\emptyset\}$ за целия въпрос.
\subsection{Базис}
\begin{Lem}
    Нека $\{a_1, a_2, \dots, a_n\}$ са ЛНЗ система от вектори от V и $a\notin l(a_1, \dots, a_n)$.\\
    Тогава $\{a_1,a_2, \dots, a_n,a\}$ е отново ЛНЗ система.
\end{Lem}
\begin{proof}
    Допускаме, че системата $\{a_1, a_2, \dots, a_n, a\}$ е ЛЗ, т.е\\
    $\lambda_1a_1 + \lambda_2a_2 + \dots + \lambda_n a_n + \lambda a = 0$ и поне един от коефициентите
    е различен от 0.\\
    1 случай) $\lambda=0$\\
    $\Rightarrow \lambda_1a_1 + \lambda_2a_2 + \dots + \lambda_n a_n = 0$ и поне един от коефициентите
    $\lambda_i, i = 1, \dots n$ е различен от 0\\
    $\Rightarrow \{a_i\}_1^n$ са ЛЗ, противоречие $\Rightarrow \{a_1,\dots,a_n\}$ са ЛНЗ\\
    2 случай) $\lambda \neq 0$\\
    $\Rightarrow a = \frac{-1}{\lambda}(\lambda_1 a_1 + \dots + \lambda_n a_n) \\
    \Rightarrow a \in l(a_1,\dots,a_n)$ противоречие $\Rightarrow \{a_1,\dots,a_n\}$ са ЛНЗ
\end{proof}
\begin{Def}
    Нека V е линейно пространство над числово поле F. Казваме, че една система вектори B от V e
    \underline{базис} на V, ако са изпъллени следните условия:
    \begin{enumerate}
        \item  B е ЛНЗ система от вектори
        \item $l(B) = V$ или еквивалентно \\
        2') Всеки вектор $v \in V$ е линейна комбинация на векторите от
        системата B
    \end{enumerate}
    Пример: векторите $\{e_i\}_{i=1}^n$ (вектори с n координати, като на позиция i имаме 1-ца, а на
    всички останали 0-ли) са базис на $V = F^n$\\
    $\{e_i\}_{i=1}^n$ са ЛНЗ система от вектори, т.к. $\sum_{i=1}^{n} e_i = 0$ само при
    $(\lambda_1,\dots,\lambda_n) = (0,\dots,0)$ и всеки вектор $\forall a \in V, a=(a_1,\dots,a_n)=a_1 e_1
    + a_n e_n$
\end{Def}
\begin{Def}
    Казваме, че едно линейно пространство V е крайномерно, ако в него може да бъде избран базис,
    състоящ се от краен брой вектори. В противен случай, казваме че V е безкрайномерно пространство.
\end{Def}
\begin{Claim}
    Нека ненулевото $V = l(a_1,\dots,a_n)$. Тогава може да се избере подсистема на векторите
    $a_1,\dots,a_n$, т.че подсистемата да е базис на V.
\end{Claim}
\begin{proof}
    От $V \neq \{\vec{0}\}$ следва, че поне един от векторите $a_i$ e ненулев. Ако $a_1 \neq \vec{0}
    \Rightarrow l(a_1)=V$\\
    $\Rightarrow$ базис на V е $a_1, a_1 \neq \vec{0}$ ЛНЗ\\
    Ако $l(a_1) \subseteq V, \exists a_2 \notin l(a_1) \Rightarrow a_1,a_2$ са ЛНЗ(от Лема).\\
    $\Rightarrow l(a_1,a_2)=V \Rightarrow a_1,a_2$ - базис на V.\\
    В противен случай $\exists a_3 \neq \vec{0}$, \dots, $\Rightarrow a_1,a_2,a_3$ са ЛНЗ.\\
    след краен брой стъпки ще изберем $(a_1,\dots,a_k), k \leqslant n$ са ЛНЗ и $l(a_1,\dots,a_k)=V
    \Rightarrow \{a_i\}_1^k$ са базис на V.
\end{proof}

\subsection{Всеки два базиса на ненулево крайномерно пространство V над F притежават равен брой вектори.}
\begin{Th}
    Нека V е крайномерно л.пр-во. Нека $\{a_i\}_{i=1}^n, \{b_j\}_{j=1}^k$ са базиси на V. Тогава
    $n=k$.
\end{Th}
\begin{proof}
    Щом двете системи от вектори са базиси, то всяка за всяка от тях важи, че са ЛНЗ и линейна ѝ
    обвивка дава V.\\
    Да допуснем, че $k>n$. Тогава $\Rightarrow\{b_j\}$ са ЛЗ(от ОЛЛА). Това е противоречие.
    $\Rightarrow k \leq n$.\\
    Аналогично при $n \leq k$. Ако допуснем строго неравенство, то от ОЛЛА $\{a_i\}_1^n$ са ЛЗ
    вектори, което е противоречие.\\
    $\Rightarrow n=k$ 
\end{proof}

\subsection{Размерност}
\begin{Def}
    Броят на векторите в кой да е базис на крайномерно ненулево пространство нариме размерност на
    пространство. Означаваме с $dim_F V = n = dimV$.\\
    Ако $V=\{\vec{0}\}$, то $dimV=0$; ако V e безкрайномерно, то $dimV=\infty$.
\end{Def}

\begin{Claim}
    Нека V е $dimV = n < \infty$(крайномерно с размерност n) лин. пр-во и $W \leqslant
    V$(подпространство). Тогава $dimW \leqslant dimV$ и, ако $dimW=dimV \iff W=V$.
\end{Claim}

\begin{Claim}
    Нека V е $dimV = n < \infty$ и $W \leqslant V$. Ако $\{a_i\}_{i=1}^k$ е са базис на W, то можем
    да допълним тази система до  с-ма $\{a_1,\dots,a_k,a_{k+1},\dots,a_n\}$, която е базис на V, т.е.
    $dimV=n$.
\end{Claim}

\subsection{V е n-мерно линейно пространство над F, тогава и само тогава когато във V съществуват n на
брой линейно независими вектора и всеки n+1 на брой вектора са линейно зависими.}
\begin{Th}
    Нека V e л.пр-во на числово поле F. Тогава\\
    а) V е n-мерно $\iff \exists(a_1,\dots,a_n)$ ЛНЗ вектора и всеки n+1 на брой вектора
    $b_1,\dots,b_n,b_{n+1}$ са ЛЗ. В частност, всяка съвкупност от n-на брой ЛНЗ вектора е базис на
    V.\\
    б) V e безкрайномерно $\iff \forall n \in \mathbb{N} \: \exists n$ на брой ЛНЗ вектора.
\end{Th}
\begin{proof}
    а) случай $\Leftarrow$) Имаме V е крайномерно и $dimV=n$. Тогава V притежава базис $a_1,..,a_n$
    и $l(a_1,\dots,a_n)=V$.\\
    Нека $b_1,\dots,b_{n+1}$ е произволна система от вектори от V.\\
    $\Rightarrow \: \{b_j\}_1^{n+1}$ се изразяват линейно чрез базисните вектори $\{a_i\}_1^n$.\\
    $\Rightarrow \{b_j\}_1^{n+1}$ са ЛЗ вектори.(от ОЛЛА)\\
    случай $\Rightarrow$) Имаме $\{a_i\}_1^n$ са ЛНЗ в-ри от всеки n+1 на брой в-ра от V са ЛЗ. Нека
    $v \in V$ е произволен вектор.\\
    1 случай) $v \notin l(a_1,\dots,a_n)$ Тогава според лемата в-рите ${a_1,\dots,a_n,v}$ са ЛНЗ.
    противоречие с допускането.\\
    2 случай) $v \in l(a_1,\dots,a_n)$ Тогава имаме, че в-рите $a_1,\dots,a_n$ са ЛНЗ и произволен
    вектор от V е тяхна линейна комбинация.\\
    $\xRightarrow{def} \: \{a_i\}_1^n$ е базис на V.\\
    Накрая нека $dimV=n \:, \{b_j\}_1^n$ са произволни ЛНЗ вектора.\\
    Тогава ако съществува вектор извън $l(\{b_j\}_1^n)$, то от Лема следва, че имаме n+1 ЛНЗ
    вектора, което е противоречие с размерността.\\
    $\Rightarrow V=l(\{b_j\}_1^n)$ и тези вектори са базис на V.\\
    б) случай $\Rightarrow$) Имаме V e безкрайномерно. Нека n е произволно естествено число.\\
    Да допуснем, че във V няма n на брой ЛНЗ вектора, т.е. всеки n вектора са ЛЗ. Тогава от а)
    следва, че $dimV < n$, което е противоречие.\\
    случай $\Leftarrow$) Имаме, че за всяко естествено число n има n-на брой ЛНЗ вектори. От а)
    следва, че V не може да е крайномерно $\Rightarrow dimV=\infty$ 
\end{proof}
    
\subsection{Всяка линейно независима система вектори в крайномерно пространство може да се допълни до
базис}
\begin{Claim}
    Всяка ЛНЗ система в-ри в крайномерно пространство V може да се допълни до базис на V.
\end{Claim}
\begin{proof}
    Нека $\{b_i\}_1^k$ са ЛНЗ вектори.\\
    Ако $V \in l(b_1,\dots,b_k)$, то тези вектори са базис на V.\\
    В противен случай съществува в-р $b_{k+1} \in V$ и $b_{k+1} \notin l(b_1,\dots,b_k)$. Според Лема,
    системата в-ри $\{b_1,\dots,b_k,b_{k+1}\}$ са ЛНЗ и ако $V=l(b_1,\dots,b_k,b_{k+1})$, то тези
    вектори са базис на V.\\
    В противен случай съществува в-р $b_{k+2} \in V$ и $b_{k+2} \notin l(b_1,\dots,b_k,b_{k+1})$\dots\\
    Продължавайки по този начин(процесът е краен, защото $dimV < \infty$) стигаме до система
    $b_1,\dots,b_k,b_{k+1},\dots,b_n$, от ЛНЗ вектори и $V=l(b_1,\dots,b_n)$. Следователно тези вектори са
    базис на V.
\end{proof}

\subsection{Координати}
\begin{Def}
    Нека V е $dimV=n < \infty$ л-пр-во над полето F и $B=\{b_1,\dots,b_n\}$ е фиксиран базис на V. Нека 
    $v\in V$ и $v=\lambda_1 b_1 + \dots + \lambda_n b_n, \lambda_i \in F, i=1,\dots,n$. Еднозначно
    определените (от базиса B) числа $\lambda_i$ наричаме координати на v в базиса B.
\end{Def}

\section{Системи линейни уравнения}
Да разгледаме системата от линейни уравнения (1)
\begin{math}
    \left\{
    \begin{aligned}
        a_{11} x_1 &+ \dots &+ a_{1n} x_n &= b_1 \\
        \vdots  \\
        a_{m1} x_1 &+ \dots &+ a_{mn} x_n &= b_m
    \end{aligned}
    \right.
\end{math}

Да означим нейната матрица с А =
\begin{math}
    \begin{bmatrix}
        a_{11} & \dots & a_{1n} \\
        \vdots & \ddots & \dots \\
        a_{m1} & \dots & a_{mn} \\
    \end{bmatrix}
\end{math}

и нейната разширена с $\overline{A}$ = 
\begin{math}
\left[
\begin{array}{@{}ccc|c@{}}
    a_{11} & \dots & a_{1n} & b_1 \\
    \vdots & \ddots & \vdots & \vdots \\
    a_{m1} & \dots & a_{mn} & b_m \\
\end{array}
\right]
\end{math}

\section{Теорема на Руше}
\begin{Th}
    Системата (1) е съвместима $\iff$ $r(A) = r(\overline{A})$, т.е. рангът на матрицата на системата е равен на ранга 
    на разширената матрицата.
\end{Th}
\section{Хомогенни системи}
\section{Връзка между решенията на хомогенна и нехомогенна система линейни уравнения}


\end{justify}
\end{document}