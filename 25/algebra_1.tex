\documentclass[fleqn,12pt]{article}

\usepackage[margin=15mm]{geometry}
\usepackage[utf8]{inputenc}
\usepackage[bulgarian]{babel}
\usepackage[unicode]{hyperref}
\usepackage{amsfonts}
\usepackage{amssymb}
\usepackage{indentfirst}
\usepackage{enumitem, hyperref}
\usepackage{blindtext}
\usepackage{multicol}

\usepackage{systeme}
\usepackage{amsmath}
\DeclareMathOperator{\cotg}{cotg}
\DeclareMathOperator{\LCS}{LCS}
\DeclareMathOperator{\longer}{longer}

\title{Базис, размерност, координати.  Системи линейни уравнения.Теорема на Руше. Връзка между решенията на хомогенна и нехомогенна система линейни уравнения.}
\author{v1.0}
\date{28 юни 2021}

\begin{document}

\maketitle

\tableofcontents
\pagebreak

\begin{flushleft}

\section{Предварителни дефиниции}
\subsection{Линейно пространство}
Засега няма да казваме какво е (дългичко), но важното е че съдържа вектори. Обикновено се означава с $V$.

\subsection{Нулев вектор}
Нулевият вектор ще означаваме с $\theta$.

\subsection{Линейна (не)зависимост}
Нека $A = \{a_1, a_2, \dots, a_n\}$ е система вектори от $V$ над поле $F$. Казваме, че $A$ е
\textbf{линейно зависима} система вектори, ако съществуват $\lambda_1, \lambda_2, \dots, \lambda_n \in F$, такива че 
поне едно $\lambda_i \neq 0$ и
\[ \lambda_1 a_1 + \lambda_2 a_2 + \dots + \lambda_n a_n = \theta \]

Ако единствената комбинация на $\lambda_1, \lambda_2, \dots, \lambda_n \in F$, която задава нулевият вектор, е нулевата, 
тогава системата $A$ наричаме \textbf{линейно независима}.

\section{Основна лема на линейната алгебра}
Нека $A = \{ a_i \}_1^n $ и $B = \{ b_j \}_1^k $ и нека векторите от $B$ се изразяват като линейни комбинации на векторите на $A$, тоест $B \subset l(A)$.
Ако $ k > n $, то системата $B$ е линейно зависима система от вектори, тоест ако повече на брой вектори се изразяват като линeйни комбинации на по-малко на брой вектори (Ако са в линейна обвивка на по-малко на брой вектори), то повечето на брой вектори са линейно зависими.

\vspace{5mm}

\textbf{Доказателство:}

Чрез индукция по $n$:

\vspace{5mm}

\textbf{\textit{База}}: Нека $ n = 1 \Rightarrow A = \{a_1\} $ и $ B = \{b_1, b_2, \dots, b_k \} = \{\lambda_1 a_1, \lambda_2 a_1, \dots, \lambda_k a_1 \} $

В частност може $ a_1 = \theta$ и $\lambda_1 \neq 0$ $ \Rightarrow \lambda_10 + \lambda_20 + \dots \lambda_k0 = 0$ $\Rightarrow$ $B$ е ЛЗ система вектори.

\vspace{5mm}

\textbf{\textit{ИП}}:  Ако $A$ се състои от по-малко от $n$ вектора и имаме условието на Лемата $ \Rightarrow B $ е линейно зависима система.

\textbf{\textit{Стъпка}}: Имаме, че $ A = \{a_1, a_2, \dots, a_n\}, B = \{b_1, b_2, \dots, b_k\}, k > n $ и $ b_j \in l(A) \Rightarrow \exists \lambda_{i_j} \in F:$
\begin{equation}
    \begin{cases}
        b_1 = \lambda_{1_1} a_1 + \lambda_{1_2} a_2 + \dots + \lambda_{1_n} a_n\\
        b_2 = \lambda_{2_1} a_1 + \lambda_{2_2} a_2 + \dots + \lambda_{2_n} a_n\\
        \dots\\
        b_k = \lambda_{k_1} a_1 + \lambda_{k_2} a_2 + \dots + \lambda_{k_n} a_n
    \end{cases}
\end{equation}

Ако $ \theta \in B \Rightarrow B $ е линейно зависима

Ако $ \theta \notin B $, тоест нека б.о.о. $ b_k \neq \theta \Rightarrow $ нека б.о.о. $\lambda_{k_n} \neq 0 $

\begin{equation}
    \begin{cases}
        b_1 = \lambda_{1_1} a_1 + \lambda_{1_2} a_2 + \dots + \lambda_{1_n} a_n\\
        b_2 = \lambda_{2_1} a_1 + \lambda_{2_2} a_2 + \dots + \lambda_{2_n} a_n\\
        \dots\\
        \frac{b_k}{\lambda_{k_n}} = \frac{\lambda_{k_1}}{\lambda_{k_n}} a_1 + \frac{\lambda_{k_2}}{\lambda_{k_n}} a_2 + \dots + a_n
    \end{cases}
    \Rightarrow
\end{equation}

\begin{equation}
    \begin{cases}
        b_1 - \frac{\lambda_{1_n}}{\lambda_{k_n}} b_k = (\lambda_{1_1} - \frac{\lambda_{1_n}}{\lambda_{k_n}} \lambda_{k_1}) a_1 + \dots + (\lambda_{1_{n-1}} - \frac{\lambda_{1_n}}{\lambda_{k_n}} \lambda_{k_{n-1}}) a_{n-1}\\
        \dots\\
        b_{k-1} - \frac{\lambda_{{(k-1)}_n}}{\lambda_{k_n}} b_k = (\lambda_{{(k-1)}_1} - \frac{\lambda_{{(k-1)}_n}}{\lambda_{k_n}} \lambda_{k_1}) a_1 + \dots + (\lambda_{{(k-1)}_{n-1}} - \frac{\lambda_{{(k-1)}_n}}{\lambda_{k_n}} \lambda_{k_{n-1}}) a_{n-1}
    \end{cases}
\end{equation}

$ \{b_1 - \frac{\lambda_{1_n}}{\lambda_{k_n}} b_k, \dots, b_{k-1} - \frac{\lambda_{{(k-1)}_n}}{\lambda_{k_n}} b_k\} $ са линейна комбинация на $ \{a_1, \dots, a_{n-1} \} $ и $ k > n \Rightarrow k-1 > n - 1$

\vspace{5mm}

От ИП $ \Rightarrow \{ b_1 - \frac{\lambda_{1_n}}{\lambda_{k_n}} b_k, \dots, b_{k-1} - \frac{\lambda_{{(k-1)}_n}}{\lambda_{k_n}} b_k\} $ са линейно зависими, тоест $ \exists \mu_1, \dots, \mu_{k-1} \in F$, б.о.о. $\mu_1 \neq 0 $:

\vspace{5mm}

$ \mu_1 (b_1 - \frac{\lambda_{1_n}}{\lambda_{k_n}} b_k) + \mu_2 (b_2 - \frac{\lambda_{2_n}}{\lambda_{k_n}} b_k) + \dots + \mu_{k-1} (b_{k-1} - \frac{\lambda_{{k-1}_n}}{\lambda_{k_n}} b_k) = \theta$

$ \mu_1 b_1 + \mu_2 b_2 + \dots + \mu_{k-1} b_{k-1} + (\dots) b_k = \theta $

$ \Rightarrow \{ b_1, b_2, \dots, b_k \} $ са ЛЗ система. $\square$


\section{Базис, размерност, координати}
Нека $ V \neq \{ \theta \} $. (за цялата секция)

\subsection{Допълване на ЛНЗ с вектор извън линейната обвивка}
\textbf{Лема:} Нека $\{a_1, a_2, \dots, a_k\}$ са линейно независима система от V и нека $ a \notin l(\{ a_1, a_2, \dots, a_k\}) $. Тогава $ \{a_1, a_2, \dots a_k, a \} $ е отново линейно независима система.

\vspace{5mm}

\textbf{Доказателство:} Нека $\{a_1, a_2, \dots, a_k\} $ са линейно независими и допуснем, че системата $ \{a_1, a_2, \dots a_k, a \} $ е линейно зависимa

\vspace{5mm}

$ \lambda_1 a_1 + \lambda_2 a_2 + \dots + \lambda_k a_k + \lambda a = \theta $ и поне един от коефицентите е различен на 0

\vspace{5mm}

Първи случай: $\lambda = 0 $

$ \lambda_1 a_1 + \lambda_2 a_2 + \dots + \lambda_k a_k = \theta $ и поне един от $\lambda_j \neq 0 $

$ \Rightarrow \{a_1, a_2, \dots, a_k\} $ са линейно зависими, което е противоречие

$ \Rightarrow \{a_1, a_2, \dots, a_k, а\} $ са линейно независими

\vspace{5mm}

Втори случай: $\lambda \neq 0 $

$ - \frac{1}{\lambda} ( \lambda_1 a_1 + \lambda_2 a_2 + \dots + \lambda_k a_k ) = a $ 

$ \Rightarrow a \in l(\{a_1, a_2, \dots, a_k\}) $, което противоречи на условието

$ \Rightarrow \{a_1, a_2, \dots, a_k, а\} $ са линейно независими. $\square$


\subsection{Базис}
\textbf{Дефиниция:} Нека V е линейно пространство над F. Казваме, че една система вектори B от V e базис на V, ако са изпълнени следните две условия:

\begin{enumerate}
    \item B е линейно независима система вектори
    \item $l(B) = V \Leftrightarrow $ Всеки вектор $ v \in V $ е линейна комбинация на векторите от системата В.
\end{enumerate}

\textbf{Пример:} $ F^n = V $

$\{ e_i\}_1^n $ са базис на $ F^n = V $, където $ e_1 = (1, 0, \dots, 0), e_2 = (0, 1, \dots, 0) \dots e_n = (0, 0, \dots, 1)$.

$ \{ e_i\}_1^n $ са ЛНЗ система вектори $\iff$ $ \lambda_1 e_1 + \lambda_2 e_2 + \dots + \lambda_n e_n = \theta $ $\iff$

$ (\lambda_1, \lambda_2, \dots, \lambda_n) = \theta \Leftrightarrow \lambda_i = 0, i = \overline{1,n}$.

\subsection{Алгоритъм за намиране на базис}
\textbf{Твърдение:} Нека $ V = l(a_1, a_2, \dots, a_n) $. Тогава може да се избере подсистема на векторите $ a_1, a_2, \dots, a_n $, такива че подсистемата да е базис на V.

\vspace{5mm}

\textbf{Доказателство:}
Нека $ V = l(a_1, a_2, \dots, a_n) $.

Нека $ a_1 \neq \theta \Rightarrow \{a_1\}$ е ЛНЗ система вектори.
Ако $ l(a_1) = V$, то $\{a_1\}$ е базис на $V$.

\vspace{4mm}

Ако $l(a_1) \subset V$, то $\exists a_2 \notin l(a_1) \Rightarrow \{a_1, a_2\}$ е ЛНЗ система вектори.
Ако $ l(a_1, a_2) = V$, то $\{a_1, a_2\}$ е базис на $V$.

\vspace{4mm}

В противен случай $\exists a_3 \dots$

След краен брой стъпки сме избрали $a_1, a_2, \dots a_k (k \leq n) $, които са линейно независими и $l(a_1, a_2, \dots a_k) = V \Rightarrow \{a_1, a_2, \dots, a_k\}$ са базис на $V$. $\square$

\subsection{Размерност}
\textbf{Дефиниция:} Казваме, че едно линейно пространство V е крайномерно, ако в него може да бъде избран базис B състоящ се от краен брой вектори. В противен случай, казваме, че V e безкрайно

\subsection{Всеки два базиса на ненулево крайномерно пространство V над F притежават равен брой вектори}
\textbf{Твърдение:} Нека V е крайномерно линейно пространство и нека $\{ a_i \}_1^n $ и $\{ b_j \}_1^k $ са базиси на V. Тогава $ k = n $.

\vspace{5mm}

\textbf{Доказателство:}

$ \{a_i\}_1^n $ е базис
\begin{enumerate}
    \item $\{a_i\}_1^n $ са линейно независими
    \item $ l(\{a_i\}_1^n) = V $
\end{enumerate}

и

$ \{b_j\}_1^k $ е базис
\begin{enumerate}
    \item $\{b_j\}_1^k $ са линейно независими
    \item $ l(\{b_j\}_1^k) = V $
\end{enumerate}

Ако предположим, че $k > n $ от Основната лема на Линейната Алгебра следва, че $ \{b_j\}_1^k $ са линейно зависими, значи не са базис $\Rightarrow k \leq n $.

Аналогично, ако $n > k $ от Основната лема на Линейната Алгебра следва, че $ \{a_i\}_1^n $ са линейно зависими, значи не са базис $\Rightarrow n \leq k $.

$\Rightarrow n = k$. $\square$

\subsection{Дефиниция за n-мерно пространство}
\textbf{Дефиниция:} Броят на векторите в кой да е базис на крайномерно $V ( \neq \{\theta\}) $ наричаме размерност на пространството и означаваме като: $dim_F V = dimV$.

Ако $V = \{\theta\} $, то по дефиниция $ dimV = 0$.

Ако $V$ e безкрайномерно, то по дефиниция $ dimV = \infty$.

\subsection{V е n-мерно
линейно пространство над F, тогава и само тогава когато във V съществуват n на
брой линейно независими вектора и всеки n+1 на брой вектора са линейно
зависими.}
\textbf{Теорема:} Нека $V$ е линейно пространство над $F$. Тогава:
\begin{enumerate}
\item $V$ e $n$-мерно (крайномерно с $dimV = n$ ) $\iff$$ \exists a_1, a_2, \dots, a_n $ ЛНЗ вектори и всяка система от $(n+1)$ на брой вектори $\{b_1, b_2, \dots, b_n, b_{n+1}\}$ е ЛЗ. В частност всяка съвкупност от $n$ на брой линейно независими вектора е базис на $V$.
\item $V$ е безкрайно $ \iff  \forall n \in \mathbb{N}, \exists n $ на брой линейно независими вектора.
\end{enumerate}

\vspace{5mm}

\textbf{Доказателство:}

1. Ще докажем по отделно в двете посоки:
    
\vspace{5mm}

$(\Rightarrow) $Ако $dimV = n \Rightarrow \exists a_1, a_2, \dots, a_n $ базис на V и $l(a_1, a_2, \dots, a_n) = V$ $\Rightarrow \forall v \in V v \in l(a_1, a_2, \dots, a_n) \Rightarrow $ от ОЛЛА $ a_1, a_2, \dots, a_n, v $ са линейно зависими 

\vspace{5mm}

$(\Leftarrow) $ Ако $ \exists a_1, a_2, \dots, a_n $ линейно независими вектори и $ \forall (n + 1) $ на брой вектора са линейно зависими $\Rightarrow \forall v \in V $ е изпълнено, че:
\begin{itemize}
    \item $a_1, a_2, \dots, a_n $ са ЛНЗ
    \item $a_1, a_2, \dots, a_n, v $ са  ЛЗ
\end{itemize}
    
$\Rightarrow v \in l(a_1, a_2, \dots, a_n), \forall v \in V \Rightarrow l(a_1, a_2, \dots, a_n) = V \Rightarrow $ по дефиниция $ \{a_i\}_1^n $ е базис на $V \Rightarrow dimV = n$. 

\vspace{5mm}

2. Нека предположим противното. От ОЛЛА следва, че сме в ситуацията от точка 1. Следователно имаме противоречие. $\square$

\subsection{Координати}
Нека $\{e_1, e_2, \dots, e_n\}$ са базис на $V$ и $v \in V$. Координати на $v$ наричаме вектора $(\lambda_1, \lambda_2, \dots, \lambda_n)$, такъв че:
\[ v = \lambda_1 e_1 + \lambda_2 e_2 + \dots + \lambda_n e_n \]

\subsubsection{Единственост}
\textbf{Твърдение: } За всеки $v \in V$ съществуват единствени координати.

\vspace{5mm}
\textbf{Доказателство: } Да допуснем, че съществуват поне две двойки координати за $v$ - $(\lambda_1, \lambda_2, \dots, \lambda_n)$ и 
$(\lambda_1', \lambda_2', \dots, \lambda_n')$, които се различават на поне една позиция. Нека б.о.о. се различават на първа позиция. 
Можем да извадим двете представяние на $v$ и получаваме:
\[ \theta = v - v = (\lambda_1 - \lambda_1') e_1 + (\lambda_2 - \lambda_2') e_2 + \dots + (\lambda_n - \lambda_n') e_1\]

Но $\lambda_1 \neq \lambda_1' \Rightarrow \lambda_1 - \lambda_1' \neq 0 \Rightarrow$ получихме ненулева комбинация, с която се получава 
нулевият вектор. Това е в противоречие с условието, че $\{e_1, e_2, \dots, e_n\}$ е базис. Следователно допускането, че координатите не са уникални, е грешно.

\section{Ранг на матрица}
Нека $A$ е $m \times n$ матрица с коефициенти от $F$:
\[ A = \begin{pmatrix}
    a_{11} & a_{12} & \dots & a_{1n} \\
    a_{21} & a_{22} & \dots & a_{2n} \\
    \vdots & \vdots & \ddots & \vdots \\
    a_{m1} & a_{m2} & \dots & a_{mn} \\
\end{pmatrix}\]

Разглеждаме системите вектори, образувани от редовете и колоните на $A$:
\begin{itemize}
    \item По редове - $\{ (a_{11}, a_{12}, \dots, a_{1n}), (a_{21}, a_{22}, \dots, a_{2n}), \dots, (a_{m1}, a_{m2}, \dots, a_{mn}) \}$
    \item По колони - $\{ (a_{11}, a_{21}, \dots, a_{m1}), (a_{12}, a_{22}, \dots, a_{m2}), \dots, (a_{1n}, a_{2n}, \dots, a_{mn}) \}$
\end{itemize}

Всяка система съдържа базис, т.е. максимална ЛНЗ подсистема. Размерът на тези базиси е еднакъв за системите по редове и по колони (но засега няма да го доказваме)
и го наричаме \textbf{ранг} на $A$ и обозначаваме $r(A)$. Очевидно $r(A) \leq \min(m,n)$.
    
\section{Системи линейни уравнения}
Нека е дадена система линейни уравнения
\begin{equation*}
    (1)\begin{cases}
        a_{1_1}x_1+\dots+a_{1_n}x_n = c_1\\
        a_{2_1}x_1+\dots+a_{2_n}x_n = c_2\\
        \dots\\
        a_{m_1}x_1+\dots+a_{m_n}x_n = c_m
    \end{cases}
\end{equation*}

$(x_1^o, x_2^o, \dots, x_n^o)$ е решение на (1), когато $a_{i_1}x_1^o+\dots+a_{i_n}x_n^o = c_i, i = \overline{1,m}$.

$ A = (a_{ij})_{m \times n} $ е матрица на (1).

$ \overline{A} = (a_{ij} | C) $ е разширена матрица на (1), където $C = (c_1, c_2, \dots, c_m)$.

Нека $ r = r(A) < \min(m, n) $. \\

\vspace{5mm}

Една матрица наричаме \textbf{несъвместима}, ако тя не притежава решение, и \textbf{съвместима}, когато има поне едно решение.

\vspace{5mm}

Една съвместима матрица наричаме \textbf{неопределена}, ако тя притежава повече от едно решение, и \textbf{определена}, когато тя притежава точно едно решение.


\subsection{Теорема на Руше}
Една система линейни уравнения е съвместима $ \Leftrightarrow r(A) = r(\overline{A}) $

\vspace{5mm}
    
    \textbf{Доказателство:}

    Ще използваме същите означения като от дефиницията на система линейни уравнения.
    Нека $b_1 = (a_{1_1}, a_{2_1}, \dots, a_{m_1}), \dots, b_n = (a_{1_n}, a_{2_n}, \dots, a_{m_n})$.

    \vspace{5mm}

    $ r(A) = r(\overline{A})$ \\
    $\Leftrightarrow C \in l(b_1, b_2, \dots, b_n)$, т.е. $\exists \lambda_1, \lambda_2, \dots, \lambda_n \in F: \lambda_1 b_1 + \lambda_2 b_2 + \dots + \lambda_n b_n = C$ \\
    $\Leftrightarrow \lambda_1, \lambda_2, \dots, \lambda_n$ е решение на (1) \\
    $\Leftrightarrow$ (1) е съвместима. $\square$

    \vspace{5mm}
    \textbf{Следствия при $r(A) = r(\overline{A}) = r$:}


    Първи случай: 

    $ r = n \Leftrightarrow detA \neq 0 \Rightarrow $ определена система (съвместима и определена). Съществува единствено решение на (1). По формули на Крамер:

    $ x_i = \frac{\Delta i}{\Delta} $

    \vspace{5mm}

    Втори случай:

    $ r < n \Leftrightarrow detA = 0 $ без ограничение на общността нека имаме (2):

    \begin{equation*}
        (2)\begin{cases}
            a_{1_1}x_1+\dots+a_{1_r}x_r = c_1 - a_{1_{r+1}}x_{r+1} - \dots - a_{1_n}x_n\\
            a_{2_1}x_1+\dots+a_{2_r}x_r = c_2 - a_{2_{r+1}}x_{r+1} - \dots - a_{2_n}x_n\\
            \dots\\
            a_{r_1}x_1+\dots+a_{r_r}x_r = c_r - a_{r_{r+1}}x_{r+1} - \dots - a_{r_n}x_n
        \end{cases}
    \end{equation*}

    $ \Delta = | a_{ij} |_{r \times r} \neq 0 $

    Всяко решение на (1) е $ ( x_1^o, x_2^o, \dots, x_r^o, x_{r+1}^o, x_{r+2}^o, \dots, x_n^o ) $

    Свободните неизвестни са $ x_1^o, x_2^o, \dots, x_r^o $ и се намират с формули на Крамер.

    $x_{r+1}^o, x_{r+2}^o, \dots, x_n^o $ са зависими неизвестни, които се определят спрямо другите.


\subsection{Връзка между решенията на хомогенна и нехомогенна система линейни уравнения}
Хомогенна система линейни уравнения наричаме
\begin{equation*}
    (1')\begin{cases}
        a_{1_1}x_1+\dots+a_{1_n}x_n = 0\\
        a_{2_1}x_1+\dots+a_{2_n}x_n = 0\\
        \dots\\
        a_{m_1}x_1+\dots+a_{m_n}x_n = 0
    \end{cases}
\end{equation*}

\begin{equation*}
    \Rightarrow(2)\begin{cases}
        a_{1_1}x_1+\dots+a_{1_r}x_r = - a_{1_{r+1}}x_{r+1} - \dots - a_{1_n}x_n\\
        a_{2_1}x_1+\dots+a_{2_r}x_r = - a_{2_{r+1}}x_{r+1} - \dots - a_{2_n}x_n\\
        \dots\\
        a_{r_1}x_1+\dots+a_{r_r}x_r = - a_{r_{r+1}}x_{r+1} - \dots - a_{r_n}x_n
    \end{cases}
\end{equation*}

$ F^n = V $ над F

$ ( x_1^o, x_2^o, \dots, x_n^o ) \in V $ е решение на хомогенната система линейни уравнения

$ \theta $ - винаги е решение на (1')

\vspace{5mm}

\textbf{Твърдение: } Нека (3) е система линейни уравнения, а (4) е нейната хомогенна система линейни уравнения.


\begin{equation*}
    (3)\begin{cases}
        a_{1_1}x_1+\dots+a_{1_n}x_n = b_1\\
        a_{2_1}x_1+\dots+a_{2_n}x_n = b_2\\
        \dots\\
        a_{m_1}x_1+\dots+a_{m_n}x_n = b_m
    \end{cases}
\end{equation*}

\begin{equation*}
    (4)\begin{cases}
        a_{1_1}x_1+\dots+a_{1_n}x_n = 0\\
        a_{2_1}x_1+\dots+a_{2_n}x_n = 0\\
        \dots\\
        a_{m_1}x_1+\dots+a_{m_n}x_n = 0
    \end{cases}
\end{equation*}

Тогава са изпълнени следните твърдения:

\begin{enumerate}
    \item Ако $\alpha = (\alpha_1, \dots, \alpha_n)$ и $\beta = (\beta_1, \dots, \beta_n)$ са решения на (3), то $\alpha - \beta$ е решение на (4).
    \item Ако $\alpha = (\alpha_1, \dots, \alpha_n)$ е решение на (3) и $\beta = (\beta_1, \dots, \beta_n)$ е решение на (4), то $\alpha + \beta$ е решение на (3).
    \item Ако системата (3) е съвместима и $\alpha = (\alpha_1, \dots, \alpha_n)$ е нейно решение, то всяко нейно решение е от вида $\alpha + \gamma$, където $\gamma$ е произволно решение на хомогенната система (4).
\end{enumerate}

\textbf{Доказателство:}

Разглеждаме последователно:

\begin{enumerate}
    \item Щом $\alpha = (\alpha_1, \dots, \alpha_n)$ и $\beta = (\beta_1, \dots, \beta_n)$ са решения на (3), то:\\
    $a_{i_1} \alpha_1 + \dots + a_{i_n} \alpha_n = b_i$ \\
    $a_{i_1} \beta_1 + \dots + a_{i_n} \alpha_n = b_i$ \\
    $\Rightarrow a_{i_1} (\alpha_1 - \beta_1) + \dots + a_{i_n} (\alpha_n - \beta_n) = b_i - b_i = 0$.
    \item Щом $\alpha = (\alpha_1, \dots, \alpha_n)$ е решение на (3) и $\beta = (\beta_1, \dots, \beta_n)$ е решение на (4), то: \\
    $a_{i_1} \alpha_1 + \dots + a_{i_n} \alpha_n = b_i$ \\
    $a_{i_1} \beta_1 + \dots + a_{i_n} \beta_n = 0$ \\
    $\Rightarrow a_{i_1} (\alpha_1 + \beta_1) + \dots + a_{i_n} (\alpha_n + \beta_n) = b_i + 0 = b_i $
    \item Щом $\alpha = (\alpha_1, \dots, \alpha_n)$ и $\beta = (\beta_1, \dots, \beta_n)$ са решения на (3), то: \\
    $\Rightarrow$ щом доказахме, че $\gamma$ е решение на (4), където $\gamma = \beta - \alpha$, то $\beta = \alpha + \gamma$. $\square$
\end{enumerate}

\end{flushleft}
\end{document}