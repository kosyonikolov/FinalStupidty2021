
\documentclass[fleqn,12pt]{article}

\usepackage[margin=15mm]{geometry}
\usepackage[utf8]{inputenc}
\usepackage[bulgarian]{babel}
\usepackage[unicode]{hyperref}
\usepackage{amsfonts}
\usepackage{amssymb}
\usepackage{enumitem, hyperref}
\usepackage{upgreek}
\usepackage{indentfirst}
\usepackage{graphicx}

\usepackage{amsmath}

\graphicspath{ {./img/} }

\title{Тема 16\\ Модели на разпределени софтуерни архитектури. Среди и протоколи за разпределени приложения.}

\author{v0.1}
\date{25 юни 2021}

\begin{document}

\maketitle
\tableofcontents
\pagebreak

\section{Основни  модели  и  методи  при  създаване  на  потребителски  интерфейс}

\subsection{Подходи}
\subsection{Oсновни процеси}
\subsection{Анализ на задачите}
\subsection{Специфициране на взаимодействията}
\subsection{Основни техники и инструментални средства}

\section{Проектиране на графичен интерфейс}

\subsection{Интерактивни стилове и техники}
\subsection{Отчитане на психологичните особености на потребителите}
\subsection{Концептуални модели и метафори}
\subsection{Методи и средства за реализация}

\section{Разработка на използваем графичен интерфейс}

\subsection{Техники базирани на експерименти}

\section{Разработка на мултимедиен графичен интерфейс}

\subsection{Проектиране на цветове}
\subsection{Проектиране на звуци}
\subsection{Проектиране на текст}
\subsection{Проектиране на графика}
\subsection{Проектиране на анимация}

\section{Особености при създаване на интегриран интерфейс}

\subsection{Методи за моделиране насочени  към  крайния  потребител}
\subsection{Екранен  дизайн}
\subsection{Oбработка  на взаимодействията}
\subsection{Интерактивнни методи за проектиране}
\subsection{Прототипиране}

\end{document}
