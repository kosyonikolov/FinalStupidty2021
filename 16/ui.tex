
\documentclass[fleqn,12pt]{article}

\usepackage[margin=15mm]{geometry}
\usepackage[utf8]{inputenc}
\usepackage[bulgarian]{babel}
\usepackage[unicode]{hyperref}
\usepackage{amsfonts}
\usepackage{amssymb}
\usepackage{enumitem, hyperref}
\usepackage{upgreek}
\usepackage{indentfirst}
\usepackage{graphicx}

\usepackage{amsmath}

\graphicspath{ {./img/} }

\title{Тема 16\\ Модели на разпределени софтуерни архитектури. Среди и протоколи за разпределени приложения.}

\author{v0.1}
\date{26 юни 2021}

\begin{document}

\maketitle
\tableofcontents
\pagebreak

\section{Основни  модели  и  методи  при  създаване  на  потребителски  интерфейс}

\textbf{\textit{Дейността по проектиране на човеко-машинен интерфейс}} включва \textbf{проектиране, създаване и оценяване} на интерактивни компютърни системи, предназначени за използване от хора.
\bigbreak
Основно значение има взаимодействието между един или повече потребители с една или повече програми, които те използват. 

% lection 9
\subsection{Подходи}

Основните подходи са:
\begin{enumerate}
    \item \textbf{Потребителя в центъра (User-centеred design (UCD))}, където потребителя е основен източник, а проектантът превежда изискванията му.
    \item \textbf{Действията в центъра (Activity-centеred design (ACD))}, който е насочен не към целите, а стъпките при всяко действие.
    \item \textbf{Системно проектиране}, където има ударение върху крайната система.
    \item \textbf{Артистично проектиране (Genius design)}, където се разчита на опита и творчеството на дизайнера, а ролята на потребителя е да потвърди дизайна.
\end{enumerate}

% lection 9
\subsection{Oсновни процеси}

Основните дейности са:
\begin{enumerate}
    \item \textit{Определяне на нуждите и установяване на изискванията}
    \item \textit{Проектиране на алтернативни решения}
    \item \textit{Избор между алтернативите(оценяване)}
    \item \textit{Създаване на прототип}
\end{enumerate}

% lection 3
\subsection{Анализ на задачите}

\textbf{\textit{Бележка}} - Явно за преподавателите по ПЧМИ думите действие и задача са едно и също нещо.

Основните методи за анализ на задачи са:
\begin{itemize}
    \item \textbf{\textit{Task Analysis}} - простите действия се разглеждат като черни кутии.
    \item \textbf{\textit{Cognitive Task Analysis}} - анализира действия, изискващи значителна умствена дейност от потребителите като решаване на проблеми, внимание и преценка.
    \item \textbf{\textit{Hierarchical Task Analysis}} - всички действия се представят като йерархия, като така по-сложните се представят като съвкупност от по-прости.
\end{itemize}

% UAN, Директно манипулируеми интерфейси -> lection 3
\subsection{Основни техники за специфициране на взаимодействията}

Взаимодействието между потребителя и интерфейса може да бъде специфицирано чрез средства като \textbf{BNF граматики и диаграми}, \textbf{многозначни граматики}, \textbf{дървета от диалогови кутии}, \textbf{избор на менюта}, \textbf{диаграми за преход} и \textbf{графи на състоянията (автомати?)}.

Категориите софтуерни инструменти за спецификация на потребителските взаимодействия са:
\begin{itemize}
    \item \textit{Средства с общо предназначение} като \textbf{PowerPoint} и \textbf{Visio}.
    \item \textit{Средства за спецификация на софтуерни системи} базирани на \textbf{UML} и подобни.
    \item \textit{Специфични системи за интерфейси} като \textbf{LucidChart}.
    \item \textit{Системи за прототипиране} като \textbf{Balsamiq}.
\end{itemize}

% Powerpoint, visio or something after UAN -> lection 3
\subsection{Инструментални средства}

\textit{\textbf{User Action Notation (UAN)}} е инструмент за специфициране на потребителските действия, която \textbf{не зависи от потребителя или технологията на интерфейса}.
Тя цели да покаже \textbf{връзката между действията и интерфейсните елементи}, описвайки интерфейсите \textbf{прецизно, подробно и еднозначно}.
\bigbreak
Основна градивна единица е \textbf{действие на потребител в даден контекст}, като то се описва \textbf{сценарии} чрез \textbf{спомагателни бележки} и \textbf{диаграма на преходите между състоянията в даден сценарий при конкретни условия}. 
\bigbreak
Чрез \textbf{UAN} могат да се репрезентират \textbf{direct manipulation interfaces (DMI)}.
\textit{\textbf{DMI}} е стил на интеракция между човек и машина, при който обектите на интерфейса могат да се манипулират директно.
Например потребител да преоразмери триъгълник на екрана използвайки мишката.

% ----------------------------------------------------------
\section{Проектиране на графичен интерфейс}

\subsection{Интерактивни стилове и техники}


\subsection{Отчитане на психологичните особености на потребителите}
\subsection{Концептуални модели и метафори}

% Методи за създаване на интерфейси - Storyboarding, Wizard of OZ -> lection 3
\subsection{Методи и средства за реализация}

% ----------------------------------------------------------
\section{Разработка на използваем графичен интерфейс}

\subsection{Техники базирани на експерименти}

% ----------------------------------------------------------
\section{Разработка на мултимедиен графичен интерфейс}

\subsection{Проектиране на цветове}
\subsection{Проектиране на звуци}
\subsection{Проектиране на текст}
\subsection{Проектиране на графика}
\subsection{Проектиране на анимация}

% ----------------------------------------------------------
\section{Особености при създаване на интегриран интерфейс}

\subsection{Методи за моделиране насочени  към  крайния  потребител}
\subsection{Екранен  дизайн}
\subsection{Oбработка  на взаимодействията}
\subsection{Интерактивнни методи за проектиране}
\subsection{Прототипиране}

\end{document}
