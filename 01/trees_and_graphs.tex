
\documentclass[fleqn,12pt]{article}

\usepackage[margin=15mm]{geometry}
\usepackage[utf8]{inputenc}
\usepackage[bulgarian]{babel}
\usepackage[unicode]{hyperref}
\usepackage{amsfonts}
\usepackage{amssymb}
\usepackage{enumitem, hyperref}
\usepackage{indentfirst}

\usepackage{amsmath}
\DeclareMathOperator{\cotg}{cotg}
\DeclareMathOperator{\LCS}{LCS}
\DeclareMathOperator{\longer}{longer}

\title{Тема 1\\ Графи. Дървета. Обхождане на графи.}
\author{v1.4}
\date{18 юли 2021}

\begin{document}

\maketitle

\tableofcontents

\section{Предисловие}
Дефинициите са взети от учебника на Манев, който е споменат в конспекта. За съжаление, не са най-удобните за 
доказване на теоремата за Ойлеровия цикъл в \textbf{мулти}граф - дефиницията на цикъл е прекалено ограничаваща, 
и заради това се получава абсурда, че Ойлеровият цикъл е всъщност \textit{специален} вид контур, а не цикъл (в общия случай). 

Може би ще е най-удобно да се пропусне ограничението $v_{p-1} \neq v_{p+1}$ в дефиницията на път и да се замести е $e_{p-1} \neq e_{p+1}$.
За просто графи функционална разлика няма, но така се позволява Ойлеровият цикъл реално да се дефинира като цикъл. За минимално объркване най-добре да 
се включат картинки.

\section{Дефиниции за краен ориентиран (мулти) граф и краен неориентиран (мулти) граф.}
\subsection{Краен ориентиран мултиграф}
Краен ориентиран мултиграф се нарича тройката $G = (V, E, f_G)$, където
\begin{itemize}
	\item $V = \{ v_1, v_2, \dots, v_n \}$ - върхове
	\item $E = \{ e_1, e_2, \dots, e_m \}$ - ребра
	\item $f_G : E \rightarrow (u,v) : u,v \in V$ - свързваща функция
\end{itemize}

\subsection{Неориентиран мултиграф}
\textit{Май нямаме "официална" дефиниция. } Като ориентирания, но ребрата нямат посока: $f_G : E \rightarrow \{u,v\} : u,v \in V$.
Неориентираните мултиграфи ще наричаме просто мултиграфи.

\subsection{Oриентиран граф}
Нека $G = (V, E, f_G)$ е ориентиран мултиграф. Казваме, че $G$ е просто ориентиран граф, ако $f_G$ е инективна. 
Такива графи означаваме $G = (V, E), E \subseteq V^2$.

\subsection{Неориентиран граф}
Нека $G = (V, E)$ е ориентиран граф. Казваме, че $G$ е неориентиран, ако релацията $E$ е симетрична и \textbf{анти}рефлексивна (т.е. без примки).

\textbf{NB: } Когато броим ребрата на неориентиран граф, ще броим само едната посока. Например, ако имаме граф с два върха $v_0, v_1$ 
и едно ребро между тях, ще считаме, че $|E| = 1$, въпреки че в реалност $E = \{ (v_0, v_1), (v_1, v_0) \}$. Това е от значение за 
твърденията за дървета и цикли.

\section{Дефиниции за маршрут (контур) в ориентиран мултиграф и път (цикъл) в неориентиран мултиграф.}
\subsection{Маршрут в (ориентиран) мултиграф}
Нека $G = (V, E, f_G)$ е краен ориентиран мултиграф. Маршрут в крайния мултиграф се нарича всяка
редица от вида $v_0, e_0, v_1, e_1, \dots , e_{L-1} , v_L$, където:
\begin{itemize}
	\item $v_i \in V$
	\item $e_j \in E$
	\item $f(e_k) = (v_k, v_{k+1})$
\end{itemize} 
 
$L$ се нарича дължина на маршрута. Ще използваме понятието за маршрут и в \textbf{неориентиран} мултиграф, с единствената разлика че 
променяме последното условие на $f(e_k) = \{v_k, v_{k+1}\}$, за да е съгласувано с дефиницията ни.

\subsection{Контур в (ориентиран) мултиграф}
Нека $v_0, e_0, v_1, e_1, \dots , e_{L-1} , v_L$ е маршрут в (ориентирания) мултиграф $G$. Ако $v_0 = v_L$, то маршрута наричаме \textbf{контур}.

\subsection{Път в неориентиран мултиграф}
Нека $G = (V, E, f_G)$ е краен неориентиран мултиграф. Път в графа се нарича всяка редица от вида
$v_0, e_0, v_1, e_1, \dots , e_{L-1} , v_L$, за която $\forall p \in \{ 0, 1, \dots, L - 1 \}$ е изпълнено:
\begin{itemize}
	\item $v_p \in V$
	\item $e_p \in E$
	\item $f(e_p) = (v_p, v_{p+1})$
	\item $v_{p-1} \neq v_{p+1}$
\end{itemize}

Последното ограничение разграничава пътищата от маршрутите. \textit{Пример:} $v_0, e_0, v_1, e_0, v_0, e_1, v_2$ е маршрут, но не и път.
Ако мултиграфът е просто граф, то тогава пропускаме да споменем ребрата - те са еднозначно определение от последователността на 
върховете.

\subsection{Цикъл в мултиграф}
Нека $v_0, e_0, v_1, e_1, \dots , e_{L-1} , v_L$ е път в неориентиран (мулти)граф. Казваме, че редицата е цикъл, ако $v_0 = v_L$.
Заради допълнителното ограничение минималната дължина на цикъл е 3.

\subsection{Нетривиален контур}
Нека $v_0, e_0, v_1, e_1, \dots, e_{L-1} , v_L$ е контур. Ако навсякъде е изпълнено $e_i \neq e_{i+1}$, ще казваме че това е \textbf{нетрививален контур}.

\textbf{NB: } Това не е официална дефиниция. Ще я ползваме в доказателство за Ойлеров цикъл, защото:
\begin{itemize}
	\item Контурът е прекалено общ - позволява подмаршрути от вида $v_0, e_0, v_1, e_0, v_0$, т.е. да отидем в някой връх и веднага да се върнем по същото ребро 
	\item Цикълът е прекалено ограничаващ - не позволява $v_0, e_0, v_1, e_1, v_0$, където $e_0 \neq e_1$.
\end{itemize}

Ако се окаже, че горните дефиниции са други или се допуска \textit{гъвкавост} при дефинирането, то тази дефиниция може да се елиминира.

\section{Свързаност и свързани компоненти на граф.} 

\subsection{Свързани компоненти на неориентиран граф}
Нека $G = (V, E)$ е неориентиран граф. 
С $R \subseteq V^2$ обозначаваме релацията $ (u, v) \in R \Leftrightarrow \exists $ път $u, w_0, w_1, \dots, v$ в $G$.
Очевидно $R$ е релация на еквивалентност. Свързани компоненти на $G$ наричаме класовете на еквивалентност на $R$.

\subsection{Свързан граф}
Нека $G(V, E)$ e неориентиран граф. Казваме, че $G$ е свързан, ако всички негови върхове
се намират в една свързана компонента.

\section{Дефиниция на дърво и кореново дърво.}

\subsection{Дърво}
Нека $G(V, E)$ e неориентиран граф. Казваме, че $G$ е дърво, ако е свързан и не съдържа цикли.

\subsection{Кореново дърво}
Кореново дърво наричаме двойката $D(V, E)$, която дефинираме индуктивно:
\begin{itemize}
	\item Двойката $(\{r\}, \varnothing)$ е кореново дърво с корен $r$ и единствено листо $r$.
	\item Нека $D(V,E)$ е кореново дърво с листа $L_0, L_1, \dots, L_k$, $u \in V, v \notin V$. Тогава $D'(V \cup \{v\}, E \cup \{ (v, u)\})$ е кореново дърво.
	Ако $u = L_i$, то листата на $D'$ са $L_0, L_1, \dots, L_{i-1}, v, L_{i+1}, \dots, L_k$. В противен случай листата на $D'$ са 
	$L_0, L_1, \dots, L_k, v$.
\end{itemize}

\section{Доказателство, че всяко кореново дърво е дърво и |V|=|E|+1.}
\subsection{Всяко кореново дърво е дърво}
\textbf{Твърдение:} Всяко кореново дърво $D(V,E)$ е дърво, т.е. свързан ацикличен граф.

\textbf{Доказателство (свързаност)}: Трябва да докажем, че съществува път между всяка двойка върхове $(u, v) \in V^2$. Достатъчно е да докажем, че 
за всяко $v \in V$ съществува път до коренът $r$. Ако това е вярно, то път между всеки два върха може да се построи с обединяване на двата пътя до корена 
(единият наобратно). Ще докажем, че от всеки връх има път до корена, използвайки индуктивната дефиниция.

\textit{Индукционно предположение:} От всеки връх на $D(V,E)$ (без корена) има път до корена $r$. 
\begin{itemize}
	\item Минималното дърво $(\{r\}, \varnothing)$ има единствен връх самият корен, като редицата $r$ можем да считаме за "път до корена".
	\item Нека $D(V,E)$ е кореново дърво и $u \in V, v \notin V$. Разглеждаме кореновото дърво $D'(V \cup \{v\}, E \cup \{ (v, u)\})$.
	От ИП знаем, че от всяко $w \in V$ има път до корена. Има ребро от върха $v \notin V$ към $u \in V$ и има път от $u$ до корена $\Rightarrow$
	има път от $v$ до корена $\Rightarrow$ От всеки връх от $V \cup \{v\}$ има път до корена $\Rightarrow$ ИП е изпълнено за $D'$.
\end{itemize}

\textbf{Доказателство (ацикличност)}: Нека допуснем, че съществува $D(V,E)$ с цикъл. Тогава съществува поддърво $D'(V', E'), V' \subseteq V, E' \subseteq E$, 
което съдържа листо със степен $>1$ - това листо е връх, участващ в цикъла, с максимална дълбочина (може да са повече от един). Това следствие се вижда лесно на картинка. 
Но по индуктивната дефиниция за кореново дърво всички листа са със степен 1. Следователно допускането ни, че съществува кореново дърво с цикъл е грешно $\Rightarrow$ всички коренови дървета
са ациклични.
\vspace{10mm}

\subsection{|V|=|E|+1}
\textbf{Твърдение:} За всяко кореново дърво $D(V,E)$ е изпълнено $|V| = |E| + 1$.

\textbf{Доказателство:} Отново ще използваме индуктивната дефиниция за кореново дърво.
\begin{itemize}
	\item Двойката $(\{r\}, \varnothing)$ е кореново дърво, $|V| = 1, |E| = 0$.
	\item Нека $D(V,E)$ е кореново дърво, $u \in V, v \notin V$. Тогава $D'(V', E'), V' = V \cup \{v\}, E' = E \cup \{ (v, u)\}$ е кореново дърво.
	Имаме $\Rightarrow |V'| = |V| + 1, |E'| = |E| + 1$, но знаем, че $|V| = |E| + 1 \Leftrightarrow |V| + 1 = |E| + 1 + 1 \Leftrightarrow |V'| = |E'| + 1$.
\end{itemize}

\section{Покриващо дърво на граф.}
\subsection{Покриващо дърво}
Нека $G = (V, E)$ е неориентиран граф. Покриващо дърво на $G$ наричаме всяко дърво $D(V, E')$, където $E' \subseteq E$.

\subsection{Покриващо дърво и свързан граф} 

\textbf{Твърдение:} Неориентиран граф $G = (V, E)$ е свързан $\Leftrightarrow$ съществува покриващо дърво $D(V, E')$.

\textbf{Доказателство ($\Leftarrow$):} Понеже $D$ е дърво, то $D$ е свързан. $D$ е подграф на $G$, но със същото множество върхове $V$, следователно $G$ също е свързан.

\textbf{Доказателство ($\Rightarrow$):} Можем да премахнем всички цикли от $G$ чрез прост алгоритъм: за всеки цикъл премахваме едно участващо ребро
и повтаряме докато има цикли. Всяко премахнато ребро \textbf{не} нарушава свързаността на графа, защото е част от цикъл - все още има 
път от всеки връх в цикъла до всеки друг.
Понеже $G$ е свързан в началото, след като приключим резултантния граф $G'(V, E' \subseteq E)$ също ще е свързан $\Rightarrow$ той е дърво.

\subsection{Следствия (a surprise tool that will help us later)}
\subsubsection{Достатъчно условие, за съществуване на цикъл в граф}
\textbf{Твърдение: } Ако граф съдържа повече от $|V| - 1$ ребра, то той съдържа цикъл.

\textbf{Доказателство: } Ако графът е свързан, то можем да го разглеждаме като покриващо дърво с поне едно добавено ребро.
Добавянето на каквото и да е ребро към дърво създава цикъл. Ако графът не е свързан, то съществува поне една свързана компонента,
за която е изпълнено $|E'| > |V'| - 1$ - в противен случай няма как за целия граф да е изпълнено $|E| > |V| - 1$.
Щом такава компонента съществува, разглеждаме я като подграф. Той е свързан $\Rightarrow$ попада в първия случай и съдържа цикъл.

\subsubsection{Достатъчно условие, за съществуване на нетривиален контур в мултиграф}
\textbf{Твърдение: } Ако неориентиран мултиграф съдържа повече от $|V| - 1$ ребра, то той съдържа нетрививален контур.

\textbf{Доказателство: } Ако мултиграфът може да се представи като граф (т.е. няма повече от едно ребро между два върха), то той
съдържа цикъл според предното следствие, а всеки цикъл е нетривиален контур.

Ако пък мултиграфът е \textit{строг} мултиграф, тогава съществуват два върха, между които има повече от две ребра. Тази комбинация от два върха и две ребра
образуват нетривален контур.

\section{Обхождане на граф в ширина и дълбочина.}

\subsection{Обхождане в ширина} 
Нека $G = (V, E)$ е свързан граф. 
\begin{enumerate}
	\item Избираме начален връх $v$, $L_0 \leftarrow \{ v \}$, $l \leftarrow 0$ и наричаме $v$ обходен.
	\item Дефинираме множеството $A_l \leftarrow \{ u \in V | u \notin L_l, L_{l-1}, \dots, L_0 \text{ и } \exists w \in L_l : (w, u) \in E \}$.
	\item Ако $A_l = \varnothing$, графът е обходен и приключваме
	\item Ако $A_l \neq \varnothing$, тогава $L_{l+1} \leftarrow A_l$, $ l \leftarrow l + 1$ и повтаряме стъпка 2.
\end{enumerate}

\subsection{Построяване на покриващо дърво чрез обхождане в ширина} 
Нека $G = (V, E)$ е свързан граф.
Покриващо дърво чрез BFS можем да построим, като започнем с $D_0(V_0, \varnothing)$, $V_0 = \{ v \}$ и на всяка стъпка 3 
от алгоритъма построяваме $D_l(V_l, E_l)$, където $V_l = V_{l - 1} \cup A_l$, 
$E_l = E_{l - 1} \cup \{ (w, u) \in E : w \in L_l, u \in A_l \}$. На всяка стъпка проверяваме дали 
$V = L_0 \cup L_1 \cup \dots \cup L_l$ - ако това е вярно, тогава всички върхове са обходени и $D_l(V_l, E_l)$ е покриващо дърво.
Ако BFS алгоритъмът спре по стъпка 3, но горното условие не е вярно, то $G$ не е свързан.
\vspace{10mm}

\subsection{Обхождане в дълбочина} 
Нека $G = (V, E)$ е свързан граф, $r \in V$ - начален връх и обходен, 
$p(t) : V \rightarrow V$ - функция на предшествениците, $p(r)$ - неопределено, $t \rightarrow r$ - текущ връх.
\begin{enumerate}
	\item \label{dfs:checkAdj} Проверяваме за връх $u \in V$, който е необходен и е съсед на $t$.
	\begin{enumerate}
		\item Ако няма такъв връх, проверяваме дали $t = r$.
		\begin{enumerate}
			\item Ако $t = r$, то всички върхове са обходени - приключваме
			\item Ако не, $t \leftarrow p(t)$ и се връщаме на \ref{dfs:checkAdj}
		\end{enumerate}
		\item \label{dfs:addV} Ако има такъв връх, то $t \leftarrow u$, $u$ става обходен и се връщаме на \ref{dfs:checkAdj}
	\end{enumerate}
\end{enumerate} 

\subsection{Построяване на покриващо дърво чрез обхождане в дълбочина} 
Нека $G = (V, E)$ е свързан граф.
Покриващо дърво на $G$ с корен $v$ можем да построим чрез алгоритъма DFS. Достъчно е да започнем
с $D_0(V_0, E_0), V_0 = \{v\}, E_0 = \varnothing$, и на всяка стъпка \ref{dfs:addV} да построим
$D_{i + 1}(V_{i + 1}, E_{i + 1}), V_{i + 1} = V_i \cup \{ v \}, E_{i + 1} = E_i \cup (t, v)$.

\section{Ойлерови обхождания на мултиграф.}

\subsection{Ойлеров път} 
Нека $G(V,E,f_G)$ е мултиграф. Казваме, че маршрут в $G$ е Ойлеров път, ако всяко ребро
участва точно веднъж.

\subsection{Ойлеров цикъл} 
Нека $G(V,E,f_G)$ е мултиграф. Казваме, че контур в $G$ е Ойлеров цикъл, ако той е Ойлеров път.
\textbf{Наблюдение: } Всеки Ойлеров цикъл е нетривиален контур.

\subsection{Ойлеров граф} 
Казваме, че $G(V,E,f_G)$ е Ойлеров граф, ако в него съществува Ойлеров цикъл.

\section{Теореми за съществуване на Ойлеров цикъл (с доказателство) и Ойлеров път.}

\subsection{Теорема за Ойлеров цикъл (Теорема на Ойлер)} 
Нека $G(V,E,f_G)$ е свързан мултиграф. Тогава $G$ е Ойлеров $\Leftrightarrow$
всички върхове на $G$ имат четна степен.

\textbf{Доказателство ($\Rightarrow$): } Щом съществува Ойлеров цикъл, то всеки връх присъства в него поне веднъж.
За всяко срещане на върха в цикъла се използат 2 различни ребра, имащи край в него - едно за влизане и едно за излизане.
Следователно ако връх $v \in V$ се среща $k$ в цикъла, тогава той има степен $2k$. Не е възможно за някои връх да има други ребра, 
защото по дефиниция цикълът минава през всяко ребро точно веднъж.

\textbf{Доказателство ($\Leftarrow$): } Ойлеровият цикъл можем да построим итеративно. 
Разглеждаме мултиграфът $G$ - знаем, че всеки връх има четна степен $\Rightarrow |E| \geq 2 |V| / 2 \Leftrightarrow |E| \geq |V| > |V| - 1 \Rightarrow G$ съдържа нетривиален контур.  
Нека $L_0$ е един такъв контур. Ако $L_0$ съдържа всички ребра, то той е Ойлеров цикъл. 

Ако не съдържа всички ребра, разглеждаме мултиграфа $G'(V', E', f_G)$ - получен чрез премахването на ребрата на $L_0$. 
Премахваме и всички върхове, останали със степен 0. За всяка свързана компонента $C_i \subseteq G'$ отново ще е вярно,
че всеки връх има четна степен, следователно отново ще съществува нетривиален. Следователно можем да разбием оригиналния
$G$ на нетривиални контури с различни ребра, които ще включват всички ребра. 

Остава да докажем, че е възможно всички тези контури да се обедният в един. Нека допуснем противното: нека съществува контур,
който не споделя нито един връх с друг контур. Това би означавало, че $G$ има повече от една свързана компонента 
$\Rightarrow G$ не е свързан, което е противоречие с условието. Следователно за всеки контур съществува поне един друг,
с който споделят поне един връх. Щом два контура споделят връх, то можем да ги обединим в един. Прилагайки тази процедура
$N - 1$ пъти, където $N$ е броя намерени контури, ще получим един контур, който съдържа всички ребра - Ойлеров цикъл. 
\vspace{10mm}

\subsection{Теорема за Ойлеров път} 
Нека $G(V,E,f_G)$ е свързан мултиграф. Тогава $G$ съдържа Ойлеров път, но не и Ойлеров цикъл, единствено когато
всички върхове без 2 са от четна степен, и останалите 2 имат нечетна степен.

\textbf{Доказателство: } Нека мултиграфа изпълнява условието за степените на върховете. Добавяме имагинерно ребро между двата върха с нечетна степен. В новият граф съществува Ойлеров цикъл.
Ако от него премахнем имагинерното ребро, получаваме Ойлеров път в оригиналния граф, имащ начало и край във върховете с нечетна степен.

Обратно, нека в мултиграфа има Ойлеров път, но не и Ойлеров цикъл. Тогава има два \textit{специални} върха - началото и края на пътя. Те трябва да 
са различни, защото иначе това би било Ойлеров цикъл. Можем да добавим имагинерно ребро между тях - така ще се получи Ойлеров цикъл. Според прената теорема 
в новият граф всички върхове ще са от четна степен. Ако сега премахнем имагинерното ребро, всички върхове без специалните остават със същата степен.
Специалните в новият граф са от четна степен, следователно в оригиналния ще са от нечетна (ако от кое да е четно число извадим 1, получаваме нечетно).

\end{document}
