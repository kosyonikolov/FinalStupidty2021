
\documentclass[fleqn,12pt]{article}

\usepackage[margin=15mm]{geometry}
\usepackage[utf8]{inputenc}
\usepackage[bulgarian]{babel}
\usepackage[unicode]{hyperref}
\usepackage{amsfonts}
\usepackage{amssymb}
\usepackage{enumitem, hyperref}

\usepackage{amsmath}
\DeclareMathOperator{\cotg}{cotg}
\DeclareMathOperator{\LCS}{LCS}
\DeclareMathOperator{\longer}{longer}

\title{Булеви функции. Пълнота}
\author{v0.1}
\date{24 май 2021}

\begin{document}

\maketitle

\tableofcontents

\begin{flushleft}

\section{Дефиниция на булева функция (БФ) и формула над множество БФ}

\subsection{Булева функция}

\subsection{Формула над множество БФ}

\section{БФ с 1 и 2 променливи}

\section{Свойства на булеви функции}

\section{Дефиниция на пълно множество БФ}

\section{Теорема за разбиване на БФ по част от променливите и теорема на Бул}

\subsection{Теорема за разбиване на БФ}

\subsection{Теорема на Бул}

\section{Теорема на Пост}


\end{flushleft}
\end{document}
