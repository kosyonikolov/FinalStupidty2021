
\documentclass[fleqn,12pt]{article}

\usepackage[margin=15mm]{geometry}
\usepackage[utf8]{inputenc}
\usepackage[bulgarian]{babel}
\usepackage[unicode]{hyperref}
\usepackage{amsfonts}
\usepackage{amssymb}
\usepackage{enumitem, hyperref}

\usepackage{amsmath}
\DeclareMathOperator{\cotg}{cotg}
\DeclareMathOperator{\LCS}{LCS}
\DeclareMathOperator{\longer}{longer}
\renewcommand{\arraystretch}{1.3}         % because math expressions

\title{Булеви функции. Пълнота}
\author{v0.1}
\date{24 май 2021}

\begin{document}

\maketitle

\tableofcontents

\begin{flushleft}

\section{Дефиниция на булева функция (БФ) и формула над множество БФ}

\subsection{Булева функция}
Нека $n \in \mathbb{N}, n \geq 1$ и означим $J_2 = \{ 0, 1 \}$. 
Булева функция на $n$ променливи наричаме всяка функция $f : J_2^n \rightarrow J_2$.
Множеството от всички булеви функции на $n$ променливи означаваме
$\mathcal{F}_2^n$, а множеството на всички булеви функции $\mathcal{F}_2 = \cup_{n \in \mathbb{N}} \mathcal{F}^2_n$.

\subsection{Формула над множество БФ}
Нека $X = \{ x_0, x_1, \dots \}$ - изброимо множество от променливи за
всяка булева функция от $\mathcal{F}_2$, $F = \{ f_i \} \subseteq \mathcal{F}_2, I \subseteq \mathbb{N}$
и $H = \{ f, x, (, ), \text{запетая}  \} \cup I $ - азбука.
Формула над множеството $F$, дефинираме като всяка дума $w \in H*$, удовлетворяваща:
\begin{itemize}
    \item $f_i \in F \Rightarrow f_i(x_1,x_2,\dots,x_n)$ е формула над $F$
    \item Нека $f_i \in F$ и $\varphi_1, \varphi_2, \dots, \varphi_n$ са формули над $F$ или променливи от $X$.
    Тогава $f_i(\varphi_1, \varphi_2, \dots, \varphi_n) \in H*$ е формула над $F$.
    \item Няма други формули освен горедефинираните.
\end{itemize}

\section{БФ с 1 и 2 променливи}

\subsection{Булеви функции с 1 променлива}
Съществуват 4 булеви функции на една променлива:
\begin{center}
\begin{tabular}{ |c|c|c|c|c| } 
    \hline
    $x$ & $\widetilde{0}$ & $x$ & $\overline{x}$ & $\widetilde{1}$ \\ 
    \hline
    0 & 0 & 0 & 1 & 1 \\ 
    1 & 0 & 1 & 0 & 1 \\ 
    \hline
\end{tabular}
\end{center}

\subsection{Булеви функции с 2 променлива}
Съществуват 16 булеви функции на две променливи, от които именуваните са в тази таблица:
\begin{center}
\begin{tabular}{ |c|c|c|c|c|c|c|c|c| } 
    \hline
    $x$ & $y$ & $x \vee y$ & $x \wedge y$ & $x \oplus y$ & $x \rightarrow y$ & $y \rightarrow x$ & $x \downarrow y$ & $x | y$ \\ 
    \hline
    0 & 0 & 0 & 0 & 0 & 1 & 1 & 1 & 1 \\ 
    0 & 1 & 1 & 0 & 1 & 1 & 0 & 0 & 1 \\ 
    1 & 0 & 1 & 0 & 1 & 0 & 1 & 0 & 1 \\ 
    1 & 1 & 1 & 1 & 0 & 1 & 1 & 0 & 0 \\ 
    \hline
\end{tabular}
\end{center}
Имената им са както следва:
\begin{itemize}
    \item $x \vee y$ - дизюнкиця
    \item $x \wedge y$ - конюнкция. За кратко записваме $xy = x \wedge y$
    \item $x \oplus y$ - изключващо или
    \item $x \rightarrow y$ и $y \rightarrow x$ - импликации
    \item $x \downarrow y$ - стрелка на Пиърс
    \item $x|y$ - черта на Шефер
\end{itemize}

\section{Свойства на булеви функции}

Нека $x,y,z \in J_2$. В сила са следните свойства:
\begin{enumerate}
    \item Комутативност: $x \vee y = y \vee x, x \wedge y = y \wedge x, x \oplus y = y \oplus x$
    \item Асоциативност: $(x \vee y) \vee z = x \vee (y \vee z), (x \wedge y) \wedge z = x \wedge (y \wedge z), (x \oplus y) \oplus z = x \oplus (y \oplus z)$
    \item Дистрибутивност: $(x \wedge y) \vee z = (x \vee z) \wedge (y \vee z)$, $(x \vee y) \wedge z = xz \vee yz$, $(x \oplus y) \wedge z = xz \oplus yz$
    \item Идемпотентност: $x \wedge x = x$, $x \wedge x = x$, $x \oplus x = \widetilde{0}$
    \item Свойства на отрицанието: $x\overline{x} = \widetilde{0}$, $x \vee \overline{x} = \widetilde{1}$, $x \oplus \overline{x} = \overline{1}$, $\overline{\overline{{x}}} = x$
    \item Свойства на константите: $x\widetilde{0} = 0$, $x\widetilde{1} = x$, $x\vee \widetilde{0} = x$, $x \vee \widetilde{1} = 1$, $x \oplus \widetilde{0} = x$, $x \oplus \widetilde{1} = \overline{x}$
    \item Закони на де Морган: $\overline{x \vee y} = \overline{x} \wedge \overline{y}$, $\overline{x \wedge y} = \overline{x} \vee \overline{y}$.  
\end{enumerate}


\section{Дефиниция на пълно множество БФ}

\section{Теорема за разбиване на БФ по част от променливите и теорема на Бул}

\subsection{Теорема за разбиване на БФ}

\subsection{Теорема на Бул}

\section{Теорема на Пост}


\end{flushleft}
\end{document}
