
\documentclass[fleqn,12pt]{article}

\usepackage[margin=15mm]{geometry}
\usepackage[utf8]{inputenc}
\usepackage[bulgarian]{babel}
\usepackage[unicode]{hyperref}
\usepackage{amsfonts}
\usepackage{amssymb}
\usepackage{enumitem, hyperref}

\usepackage{amsmath}
\DeclareMathOperator{\cotg}{cotg}
\DeclareMathOperator{\LCS}{LCS}
\DeclareMathOperator{\longer}{longer}

\title{Булеви функции. Пълнота}
\author{v0.1}
\date{24 май 2021}

\begin{document}

\maketitle

\tableofcontents

\begin{flushleft}

\section{Дефиниция на булева функция (БФ) и формула над множество БФ}

\subsection{Булева функция}
Нека $n \in \mathbb{N}, n \geq 1$ и означим $J_2 = \{ 0, 1 \}$. 
Булева функция на $n$ променливи наричаме всяка функция $f : J_2^n \rightarrow J_2$.
Множеството от всички булеви функции на $n$ променливи означаваме
$\mathcal{F}_2^n$, а множеството на всички булеви функции $\mathcal{F}_2 = \cup_{n \in \mathbb{N}} \mathcal{F}^2_n$.

\subsection{Формула над множество БФ}
Нека $X = \{ x_0, x_1, \dots \}$ - изброимо множество от променливи за
всяка булева функция от $\mathcal{F}_2$, $F = \{ f_i \} \subseteq \mathcal{F}_2, I \subseteq \mathbb{N}$
и $H = \{ f, x, (, ), \text{запетая}  \} \cup I $ - азбука.
Формула над множеството $F$, дефинираме като всяка дума $w \in H*$, удовлетворяваща:
\begin{itemize}
    \item $f_i \in F \Rightarrow f_i(x_1,x_2,\dots,x_n)$ е формула над $F$
    \item Нека $f_i \in F$ и $\varphi_1, \varphi_2, \dots, \varphi_n$ са формули над $F$ или променливи от $X$.
    Тогава $f_i(\varphi_1, \varphi_2, \dots, \varphi_n) \in H*$ е формула над $F$.
    \item Няма други формули освен горедефинираните.
\end{itemize}

\section{БФ с 1 и 2 променливи}

\section{Свойства на булеви функции}

\section{Дефиниция на пълно множество БФ}

\section{Теорема за разбиване на БФ по част от променливите и теорема на Бул}

\subsection{Теорема за разбиване на БФ}

\subsection{Теорема на Бул}

\section{Теорема на Пост}


\end{flushleft}
\end{document}
