\documentclass[fleqn,12pt]{article}

\usepackage[margin=15mm]{geometry}
\usepackage[utf8]{inputenc}
\usepackage[bulgarian]{babel}
\usepackage[unicode]{hyperref}
\usepackage{amsfonts}
\usepackage{amssymb}
\usepackage{enumitem, hyperref}
\usepackage{upgreek}
\usepackage{graphicx}
\usepackage{mathtools}
\usepackage{pgfplots}


\title{Oпределен  интеграл.  Дефиниция  и  свойства.  Интегруемост  на непрекъснати функции. Теорема на Нютон-Лайбниц}
\author{v1.0}
\date{16 юни 2021}

\begin{document}

\maketitle

\tableofcontents
\pagebreak

\begin{flushleft}

\section{Разбиване на интервал}
Разбиване $\tau$ на интервала $[a,b]$ наричаме системата от точки $\{x_i\}_{i=0}^n : а = x_0 < x_1 < ... < x_n = b$. В следствие от разбиването
се образуват n непрепокриващи се подинтервала на $[a,b] : [a, x_1], [x_1, x_2], ... [x_{n-1}, b]$, като дължината на интервал $i \in [1,n]$ е $\Delta_i = x_i - x_{i-1}$.

\section{Разбиване съдържащо друго разбиване}
Нека $x,y$ са две разбивания на интервала $[a,b]$, съответно на системите от точки $\{x_i\}_{i=0}^{n}$ и  $\{y_i\}_{i=0}^{m}$. Казваме че разбиването $y$ съдържа в себе си разбиването $x$ ($x \prec y$), ако
$\{x_i\}_{i=0}^{n} \subseteq  \{y_i\}_{i=0}^{m}, n \leq m$.

\section{Междинни точки на разбиване}
Нека $\tau$ e разбиване на интервала $[a,b]$. Тогава точките ${\xi_i}_{i=1}^n : \xi_i \in [x_{i-1},x_i]$ ще наричаме междинни точки на това разбиване.

\section{Диаметър на разбиване}
Диаметър на разбиването $\tau$ дефинираме дължината на най-дългият интервал, който се образува от разбиването, т.е $d(\tau)=max_{i\in[1,n]} \Delta_i$.

\section{Риманова сума}
Римановата сума на функцията $f(x)$, съответстваща на разбиването $\tau$ на интервала [a,b] и на избраните междинни точки $\{\xi_i\}_{i=1}^n$, се определя от формулата: 
$R(f;\tau;\{\xi_i\})=\sum_{n = 1}^{n} f(\xi)(x_i-x_{i-1})$.  

\section{Риманов интеграл}
Стойността $I$ наричаме граница на римановите суми $R_\tau(f)$ на функцията $f$ при $d(\tau)\rightarrow0$, ако $\forall\epsilon>0, \exists \delta>0 :$ за всяко разбиване $\tau$
на интервала $[a,b]$ с $d(\tau)<\delta$ е изпълнено че $|R_f(x) - I|<\epsilon$ (при произволен избор на междинни точки). Всяка функцията $f$, за която съществува такава граница на римановите суми в $[a,b]$,
се нарича интегруема (по Риман) в $[a,b]$, a границата $I$ се нарича определен интеграл от $f$ в интервала $[a,b]$ и се бележи с $I = \int_{a}^{b}  f(x)\,dx$ 

\section{Всяка интегруема по Риман функция е ограничена}
\textbf{Доказателство: } Нека функцията $f$ e интегруема по Риман в интервала $[a,b]$ и $I = \int_{a}^{b}  f(x)\,dx$. Нека изберем $\epsilon=1$ тогава $\exists \delta>0$, такова че 
за всяко разбиване $\tau$ на интервала $[a,b]$ с $d(\tau)<\delta$ е изпълнено че $|R_\tau(f) - I|<1$, т.е $I-1<R_f(x)<I+1$. От което следва, че множеството от римановите суми на функцията
$f$ е ограничено когато $d(\tau)<\delta$. Остава да докажем, че $f$ е ограничена в интервала $[a,b]$.
Да допуснем че $f$ не е ограничена в $[a,b]$. Тогава съществува поне един интервал от разбиването $\tau$, в който $f$ е неограничена - нека такъв интервал е например $[x_{j-1},x_j]$.
Нека в този интервал вземем редица от точки $\{\xi_j^{(k)}\}_k : |f(\xi_j^{(k)})| > k, \forall k \in [0,\infty]$, следователно $\lim_{k \rightarrow +\infty} f(\xi_j^{(k)}) = +\infty$.
Нека за останите интервали фиксираме прозиволни междинни точки $\{\xi_i\}_{i=1,i \neq j}^n : \xi_i \in [x_{i-1},x_i]$ и тогава $\sum_{i=1,i \neq j}^{n} f(\xi_i)(x_i - x_{i-1})$ е с фиксирана стойност.\\
Добавяйки към тази сума събираемото $f(\xi_j^{(k)})(x_j - x_{j-1})$ получаваме римановата сума на $f$ в $[a,b]$ : $R(f;\tau;\xi_1,\xi_2, ... ,\xi_{j-1},\xi_j^{(k)},\xi_{j+1}, ... ,\xi_n)$. Тогава е изпълнено, че:\\
$\lim_{x \rightarrow \infty} R(f;\tau;\xi_1,\xi_2, ... ,\xi_{j-1},\xi_j^{(k)},\xi_{j+1}, ... ,\xi_n) = \lim_{x \rightarrow \infty} (f(\xi_j^{(k)})(x_j - x_{j-1}) + \sum_{i=1,i \neq j}^{n} f(\xi_i)(x_i - x_{i-1})) = +\infty$ \\
Което означава че множеството от риманови суми на $f$ в $[a,b]$ е неограничено, включително когато $d(\tau)<\delta$. Следователно противоречие с допускането, че $f$ e неограничена в $[a,b]$.
Теоремата е доказана. 

\section{Малка сума на Дарбу}
Нека $f$ е ограничена в интервала $[a,b]$ и $\tau$ е разбиване на интервала $[a,b]$ на системата от точки $\{x_i\}_{i=0}^{n}$.
Тогава "малка сума на Дарбу" дефинираме като $\underline{s}(f,[a,b],\tau)=\sum_{i=1}^{n} m_i(x_i-x_{i-1})$, където $m_i=\inf_{x\in[x_{i-1},x_i]}f(x)$.

(Тук да поставя диаграма)

\section{Голяма сума на Дарбу}
Нека $f$ е ограничена в интервала $[a,b]$ и $\tau$ е разбиване на интервала $[a,b]$ на системата от точки $\{x_i\}_{i=0}^{n}$.
Тогава "голяма сума на Дарбу" дефинираме като $\overline{S}(f,[a,b],\tau)=\sum_{i=1}^{n} M_i(x_i-x_{i-1})$, където $M_i=\sup_{x\in[x_{i-1},x_i]}f(x)$.

(Тук да поставя диаграма)

\section{Характеристики на сумите на Дарбу}
\subsection{Връзка между сумите на Дарбу и Римановите суми}
Нека $f$ e интегруема по Риман в интервала $[a,b]$ и $\tau$ е разбиване на интервала $[a,b]$ на системата от точки $\{x_i\}_{i=0}^{n}$. Toгава е в сила, че:\\
$\underline{s}(f,[a,b],\tau) \leq R_f(\tau) \leq \overline{S}(f,[a,b],\tau)$

\subsection{Свойство 1: Малките суми нарастват, а не намаляват}
Нека $x,y$ са разбивания на интервала $[a,b]$, такива че разбиването $y$ строго съдържа в себе си разбиването $x$. Тогава е в сила, че:
$underline{s}(f,[a,b],x) < underline{s}(f,[a,b],y)$\\
\textbf{Доказателство: } Нека системата от точки на разбиването $y$ съдържа точно една точка в повече от системата от точки на разбиването $x$.
Нека тази точка е $x^{*} \in [x_{j-1},x_j], j\in [0,n]$. Нека разгледаме израза:
$\underline{s}(f,[a,b],y)-\underline{s}(f,[a,b],x) = (m_1(x^{*} - x_{j-1}) + m_2(x_j - x^{*}) + \sum_{i=0,i \neq j}^{n} m_i(x_i - x_{i-1})) - (\sum_{i=0}^{n} m_i(x_i - x_{i-1}))$,\\ kъдето
$m_1 =\inf_{x\in[x_{j-1},x^{*}]}f(x), m_2 =\inf_{x\in[x^{*},x_j]}f(x)$.\\
$\underline{s}(f,[a,b],y) - \underline{s}(f,[a,b],x) = m_1(x^{*} - x_{j-1}) + m_2(x_j - x^{*}) - m_j(x_j - x_{j-1}) = m_1(x^{*} - x_{j-1}) + m_2(x_j - x^{*}) - m_j(x_j + x^{*} - x^{*} - x_{j-1})$
$ = (m_1-m_j)(x^{*} - x_{j-1}) + (m_2-m_j)(x_j - x^{*})$. Но понеже $m_1 \geq m_j, m_2 \geq m_j$ понеже $m_1,m_2$ са минимумите на $f$ в подинтервали на $[x_{j-1},x_j]$.
Следователно $\underline{s}(f,[a,b],y) - \underline{s}(f,[a,b],x) = (m_1-m_j)(x^{*} - x_{j-1}) + (m_2-m_j)(x_j - x^{*}) \geq 0$.

\subsection{Свойство 2: Големите суми намаляват, а не нарастват}
Нека $x,y$ са разбивания на интервала $[a,b]$, такива че разбиването $y$ строго съдържа в себе си разбиването $x$. Тогава е в сила, че:
$\overline{S}(f,[a,b],y) < \overline{S}(f,[a,b],x)$\\
\textbf{Доказателство: } Нека системата от точки на разбиването $y$ съдържа точно една точка в повече от системата от точки на разбиването $x$.
Нека тази точка е $x^{*} \in [x_{j-1},x_j], j\in [0,n]$. Нека разгледаме израза:
$\overline{S}(f,[a,b],y) - \overline{S}(f,[a,b],x) = (M_1(x^{*} - x_{j-1}) + M_2(x_j - x^{*}) + \sum_{i=0,i \neq j}^{n} M_i(x_i - x_{i-1})) - (\sum_{i=0}^{n} M_i(x_i - x_{i-1}))$,\\ kъдето
$M_1 =\sup_{x\in[x_{j-1},x^{*}]}f(x), M_2 =\sup_{x\in[x^{*},x_j]}f(x)$.\\
$\overline{S}(f,[a,b],y) - \overline{S}(f,[a,b],x) = M_1(x^{*} - x_{j-1}) + M_2(x_j - x^{*}) - m_j(x_j - x_{j-1}) = M_1(x^{*} - x_{j-1}) + M_2(x_j - x^{*}) - M_j(x_j + x^{*} - x^{*} - x_{j-1})$
$ = (M_1-M_j)(x^{*} - x_{j-1}) + (M_2-M_j)(x_j - x^{*})$. Но понеже $M_1 \leq M_j, M_2 \leq M_j$ понеже $M_1,M_2$ са максимумите на $f$ в подинтервали на $[x_{j-1},x_j]$.
Следователно $\overline{S}(f,[a,b],y) - \overline{S}(f,[a,b],x) = (M_1-M_j)(x^{*} - x_{j-1}) + (M_2-M_j)(x_j - x^{*}) \leq 0$.

\subsection{Свойство 3: Всяка малка сума на Дарбу е по-малка или равна на всяка голяма}
\textbf{Доказателство: }
Нека $f$ e ограничена в интервала $[a,b]$ и $x,y$ са произволни разбивания на интервала $[a,b]$. Нека $z$ е също разбиване на интервала $[a,b]$,
съдържащо в себе си разбиванията $x,y$ ($x \preceq z, y \preceq z$). Тогава :
от свойство 1 следва, че $\underline{s}(f,[a,b],x) \leq \underline{s}(f,[a,b],z)$
от свойство 2 следва, че $\overline{S}(f,[a,b],z) \leq \overline{S}(f,[a,b],y)$
Следователно $\underline(s)(f,[a,b],x) \leq \underline(s)(f,[a,b],z) \leq \overline(S)(f,[a,b],z) \leq \overline(s)(f,[a,b],y)$.\\
Понеже избрахме $x,y$ да са произволни разбивания на $[a,b]$, то твърдението е изпълнено. 

\section{Интегруемост по Дарбу}
От Свойство 3 на сумите на Дарбу знаем, че всяка малка сума на Дарбу е по-малка или равна на всяка голяма, т.е малките суми на Дарбу имат горна граница, а горните суми на Дарби долна граница.
Нека дефинираме $\underline{I}=\inf_\tau \underline{s}(f,[a,b],\tau)$ и $\overline{I}=\sup_\tau \overline{S}(f,[a,b],\tau)$,
тогава $\underline{I}$ и  $\overline{I}$ наричаме съответно долен и горен интеграл на Дарбу. Ако $\underline{I}=\overline{I}=I$, то функцията $f$ е интегруема по Дарбу в интервала $[a,b]$.

\section{Дадена функция е интегруема по Риман тогава и само тогава, когато за всяко $\epsilon > 0$ съществуват голяма и малка сума на Дарбу $\overline{S}$ и $\underline{s}$ такива, че $\overline{S} − \underline{s} < \epsilon$}
\textbf{Доказателство (==>):}
Нека $f$ e интегруема по Риман

\section{Всяка непрекъсната функция в краен и затворен интервал е интегруема по Риман}
\section{Основни свойства на Римановия интеграл}
\section{Теорема за средните стойности}
\section{Теорема на Лайбниц-Нютон}
\section{Формула на Лайбниц-Нютон (използване на теоремата за изчисляване на определен интеграл)}


\end{flushleft}
\end{document}