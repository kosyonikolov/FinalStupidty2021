\documentclass[fleqn,12pt]{article}

\usepackage[margin=15mm]{geometry}
\usepackage[utf8]{inputenc}
\usepackage[bulgarian]{babel}
\usepackage[unicode]{hyperref}
\usepackage{amsfonts}
\usepackage{amssymb}
\usepackage{enumitem, hyperref}
\usepackage{upgreek}
\usepackage{graphicx}
\usepackage{mathtools}




\title{Oпределен  интеграл.  Дефиниция  и  свойства.  Интегруемост  на непрекъснати функции. Теорема на Нютон-Лайбниц}
\author{v1.0}
\date{16 юни 2021}

\begin{document}

\maketitle

\tableofcontents
\pagebreak

\begin{flushleft}

\section{Разбиване на интервал}
Разбиване $\tau$ на интервала $[a,b]$ наричаме системата от точки $\{x_i\}_{i=0}^n : а = x_0 < x_1 < ... < x_n = b$. В следствие от разбиването
се образуват n непрепокриващи се подинтервала на $[a,b] : [a, x_1], [x_1, x_2], ... [x_{n-1}, b]$, като дължината на интервал $i \in [1,n]$ е $\Delta_i = x_i - x_{i-1}$.

\section{Междинни точки на разбиване}
Нека $\tau$ e разбиване на интервала $[a,b]$. Тогава точките ${\xi_i}_{i=1}^n : \xi_i \in [x_{i-1},x_i]$ ще наричаме междинни точки на това разбиване.

\section{Диаметър на разбиване}
Диаметър на разбиването $\tau$ дефинираме дължината на най-дългият интервал, който се образува от разбиването, т.е $d(\tau)=max_{i\in[1,n]} \Delta_i$.

\section{Риманова сума}
Римановата сума на функцията $f(x)$, съответстваща на разбиването $\tau$ на интервала [a,b] и на избраните междинни точки $\{\xi_i\}_{i=1}^n$, се определя от формулата: 
$R(f;\tau;\{\xi_i\})=\sum_{n = 1}^{n} f(\xi)(x_i-x_{i-1})$.  

\section{Риманов интеграл}
Стойността $I$ наричаме граница на римановите суми $R_\tau(f)$ на функцията $f$ при $d(\tau)\rightarrow0$, ако $\forall \epsilon>0 \exists \delta>0 :$ за всяко разбиване $\tau$
на интервала $[a,b]$ с $d(\tau)<\delta$ е изпълнено че $|R_f(x) - I|<\epsilon$ (при произволен избор на междинни точки). Всяка функцията $f$, за която съществува такава граница на римановите суми в $[a,b]$,
се нарича интегруема (по Риман) в $[a,b]$, a границата $I$ се нарича определен интеграл от $f$ в интервала $[a,b]$ и се бележи с $I = \int_{a}^{b}  f(x)\,dx$ 

\section{Всяка интегруема по Риман функция е ограничена}
\textbf{Доказателство: } Нека функцията $f$ e интегруема по Риман в интервала $[a,b]$ и $I = \int_{a}^{b}  f(x)\,dx$. Нека изберем $\epsilon=1$ тогава $\exists \delta>0$, такова че 
за всяко разбиване $\tau$ на интервала $[a,b]$ с $d(\tau)<\delta$ е изпълнено че $|R_\tau(f) - I|<1$, т.е $I-1<R_f(x)<I+1$. От което следва, че множеството от римановите суми на функцията
$f$ е ограничено когато $d(\tau)<\delta$. Остава да докажем, че $f$ е ограничена в интервала $[a,b]$.
Да допуснем че $f$ не е ограничена в $[a,b]$. Тогава съществува поне един интервал от разбиването $\tau$, в който $f$ е неограничена - нека такъв интервал е например $[x_{j-1},x_j]$.
Нека в този интервал вземем редица от точки $\{\xi_j^{(k)}\}_k : |f(\xi_j^{(k)})| > k, \forall k \in [0,\infty]$, следователно $\lim_{k \rightarrow +\infty} f(\xi_j^{(k)}) = +\infty$.
Нека за останите интервали фиксираме междинни точки $\{\xi_i\}_{i=1,i \neq j}^n : \xi_i \in [x_{i-1},x_i]$ и тогава $\sum_{i=1,i \neq j}^{n} f(\xi_i)(x_i - x_{i-1})$ е с фиксирана стойност.\\
В този ред на мисли римановата сума на $f$ в $[a,b]$ e: \\
$R(f;\tau;\xi_1,\xi_2, ... ,\xi_{j-1},\xi_j^{(k)},\xi_{j+1}, ... ,\xi_n) = f(\xi_j^{(k)})(x_j - x_{j-1}) + \sum_{i=1,i \neq j}^{n} f(\xi_i)(x_i - x_{i-1}) = +\infty$ \\
Което означава че множеството от риманови суми на $f$ в $[a,b]$ е неограничено, включително когато $d(\tau)<\delta$. Следователно противоречие с допускането, че $f$ e неограничена в $[a,b]$.
Теоремата е доказана. 

\section{Малка сума на Дарбу}
\section{Голяма сума на Дарбу}
\section{Характеристики на сумите на Дарбу}
\section{Дадена функция е интегруема по Риман тогава и само тогава, когато за всяко 𝜀 > 0 съществуват голяма и малка сума на Дарбу 𝑆 и 𝑠 такива, че 𝑆 − 𝑠 < 𝜀}
\section{Всяка непрекъсната функция в краен и затворен интервал е интегруема по Риман}
\section{Основни свойства на Римановия интеграл}
\section{Теорема за средните стойности}
\section{Теорема на Лайбниц-Нютон}
\section{Формула на Лайбниц-Нютон (използване на теоремата за изчисляване на определен интеграл)}


\end{flushleft}
\end{document}