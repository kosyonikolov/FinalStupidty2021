
\documentclass[fleqn,12pt]{article}

\usepackage[margin=15mm]{geometry}
\usepackage[utf8]{inputenc}
\usepackage[bulgarian]{babel}
\usepackage[unicode]{hyperref}
\usepackage{amsfonts}
\usepackage{amssymb}
\usepackage{enumitem, hyperref}
\usepackage{upgreek}
\usepackage{indentfirst}

\usepackage{amsmath}
\DeclareMathOperator{\cotg}{cotg}
\DeclareMathOperator{\LCS}{LCS}
\DeclareMathOperator{\longer}{longer}

\title{Софтуерно инженерство и неговото място като дял от знанието. Софтуерен процес и модели на софтуерни процеси. Концепция за многократна употреба.}

\author{v0.1}
\date{24 юни 2021}

\begin{document}

\maketitle

\tableofcontents

\clearpage

\section{Софтуерното  инженерство}
\subsection{Какво  е  софтуер?}
Един софтуерът може да се определи като съвкупност от:
\begin{itemize}
	\item Инструкции (програми), които при изпълнение осигуряват желаните характеристики, функционалностти и производителност
	\item Структури от данни, които дават възможност на програмите да манипулират информацията адекватно
	\item Документация, описваща работата и използването на програмите
\end{itemize}

Софтуера се обособява, чрез следните характеристики:
\begin{itemize}
	\item Софтуера се разработва, не се произвежда
	\item Софтуера не се износва
	\item Софтуера е сложен и притежава голямо ниво на абстрактност
\end{itemize}

\subsection{Видове  софтуер}
Имаме няколко вида софтуер:
\begin{itemize}
	\item Системен софтуер
	\item Приложен софтуер
	\item Научен софтуер
	\item Вграден софтуер
	\item Продуктова линия
	\item Уеб приложения
	\item Изкуствен интелект
	\item Системи от системи
	\item Наследен софтуер
\end{itemize}

\subsection{Същност и обхват на софтуерното инженерство}
Инженерството се състой от анализ, проектиране, конструиране, верифициране и управление на технически (или социални) единици. Софтуерното инженерство е дисциплина, която се занимава с всички аспекти на проектирането и разработката на \textit{висококачествен} софтуер.

Софтуерните инженери би трябвало да възприемат в своята работа систематизиран и организиран подход на разработка. Също така трябва да използват подходящи средства в зависимост от решавания проблем, съществуващите ограничения и начичните ресурси

Ако един софтуер се разработва с цел решаването на даден проблем, то можем да кажем, че софтуерното инженерство се занимава с:
\begin{itemize}
	\item Разбирането и анализирането на проблема
	\item Намиране на решение за проблема
	\begin{itemize}
		\item Конструиране на решението от части, които засягат различни аспекти на проблема. Този процес се нарича синтез.
	\end{itemize}
	\item Решаване на проблема с използване на различни
	\begin{itemize}
		\item Методи или техники (формални процедури за произвеждане на някакъв резултат)
		\item Средства (инструменти или автоматизирани системи)
		\item Процедури (комбинация от средства и техники)
		\item Парадигми (определени подходи или философии)
	\end{itemize}
\end{itemize}

\section{Софтуерен  процес}

Софтуерния процес е последователност от стъпки включващи дейности, ограничения и ресурси, които се следват за създаването на продукт или система в даден срок и с високо качество.

\subsection{Фази и  основни  дейности}
\subsubsection{Фази на софтуерния процес:}
\begin{itemize}
	\item Анализ и дефиниране на изискванията
	\item Проектиране на системата
	\item Проектиране на програмата
	\item Писане на програмата
	\item Тестване на единици (unit testing)
	\item Интеграционно тестване (integration testing)
	\item Тестване на системата
	\item Доставяне на системата
	\item Поддръжка
\end{itemize}

\subsubsection{Основни дейности на софтуерния процес:}
\begin{itemize}
	
	\item Комуникация
	\begin{itemize}
		\item събиране и разбиране на изисквания за функционалността на софтуера и за ограниченията вурху разработката му
	\end{itemize}
	
	\item Планиране
	\begin{itemize}
		\item Създава се план за бъдещата работа по разработката на софтуера
		\item Описват се
		\begin{itemize}
			\item техническите рискове, които трябва да се имат предвид
			\item потенциални рискове
			\item необходими ресурси
			\item работните продукти, които ще се произведат
			\item времеви график на работата
		\end{itemize}
	\end{itemize}
	
	\item Моделиране - има 2 части: Анализ и проектиране
	\begin{itemize}
		\item Анализ:
		\begin{itemize}
			\item Работни задачи: Събиране на изисквания, Уточняване, Договаряне, Специфициране и Валидиране
			\item Работни продукти: Модел на анализа и Спецификация на изискванията
		\end{itemize}
		
		\item Проектиране:
		\begin{itemize}
			\item Работни задачи: Дизайн на данните, архитектурата, интерфейсите и нивото на компонентите.
			\item Работни продукти: Модел и спецификация на дизайна
		\end{itemize}
	\end{itemize}

	\item Конструиране
	\begin{itemize}
		\item Генериране на код:
		\item Тестване:
		\begin{itemize}
			\item тестване на самостоятелни компоненти
			\item тестване на интегрираната система (модулите + подсистемите + системата)
			\item потребителско (бета) тестване
		\end{itemize}
	\end{itemize}


	\item Внедряване
	\begin{itemize}
		\item Софтуерът се предоставя на клиента:
		\item Клиентът оценява продукта като дава обратна връзка и препоръки:
	\end{itemize}
\end{itemize}

\subsubsection{Допълнителни дейности на софтуерния процес:}
\begin{itemize}
	\item Следене и управление на софт. продукт
	\item Управление на риска
	\item Осигуряване на качеството
	\item Формални технически прегледи
	\item Измерване на системата
	\item Управление на софтуерната конфигурация
	\item Управление на повторното използване
	\item Подготовка и генериране на работни продукти
\end{itemize}

\subsection{Модели на софтуерния процес}
Един модел представлява опростено описание на начина на разработка на софтуера, представено от определена гледна точка. Те биват няколко вида:

\begin{itemize}
	\item Описателни модели - как се разработва софтуера
	\item Предписателни модели - как би трябвало да се разработва софтуера.
	\begin{itemize}
		\item те дефинират специално множество от дейности, задачи, milestone-ове и работни продукти, които са необходими за създаването на софтуер с високо качество.
	\end{itemize}
\end{itemize}

\subsection{Езици за  моделиране}
Език за моделиране на процеси е език, адаптиран или създаден с цел представянето на процеси. Те се използват за представяне по прецизен и изчерпателен начин следните характеристики на софтуерния процес:
 
\begin{itemize}
	\item Дейностите, които трябва да се извършат, за да се постигнат целите на процеса
	\item Ролите на хората, участващи в процеса
	\item Структурата и същността на артефактите, които се създават и поддържат
	\item Средствата, които се използват
\end{itemize}

Класификация на езиците за моделиране на процеси:
\begin{itemize}
	\item Дескриптивни (логически) - използват правила и тригери (Sentinel/Latin)
	\item Мрежово базирани - представляват процесите на базата на мрежи на Петри (SPADE, FUNSOFT nets)
	\item Императивни - базирани на езиците за програмиране (APPL/A, JIL)
\end{itemize}

\subsection{Шаблони за описание}
Шаблон е описание на общо решение на общ проблем, на базата на което може да се извлече детайлно решение на специфичен проблем.
Шаблон за описание на процес представлява структурирано описание на процес, което е метод за описание на важните характеристики на софтуерния процес. Шаблоните могат да бъдат дефинирани на различни нива на абстракция. Важна характеристика за шаблоните е, че описва какво да се направи, но не \textbf{как} да се направи.

\section{Сравнителен анализ на описателни модели на софтуерен процес}
\subsection{Модел на водопада}
TODO

\subsection{Прототипен модел}
TODO

\subsection{Модел на бързата разработка}
TODO

\subsection{Спираловиден модел}
TODO

\section{Концепция за многократна употреба}
TODO

\end{document}
