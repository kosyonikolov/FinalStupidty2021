
\documentclass[fleqn,12pt]{article}

\usepackage[margin=15mm]{geometry}
\usepackage[utf8]{inputenc}
\usepackage[bulgarian]{babel}
\usepackage[unicode]{hyperref}
\usepackage{amsfonts}
\usepackage{amssymb}
\usepackage{enumitem, hyperref}
\usepackage{upgreek}
\usepackage{indentfirst}

\usepackage{amsmath}
\DeclareMathOperator{\cotg}{cotg}
\DeclareMathOperator{\LCS}{LCS}
\DeclareMathOperator{\longer}{longer}

\title{Модели на разпределени софтуерни архитектури. Среди и протоколи за разпределени приложения.}

\author{v0.1}
\date{25 юни 2021}

\begin{document}

\maketitle
\tableofcontents
\pagebreak

\section{Параметри на паралелната и разпределената обработка}

\subsection{Метрика и методи за анализ}

\subsection{Модели на разпределените софтуерни архитектури и техните структури, организация, компоненти и приложение}

\subsection{Процедурни модели}
\subsection{Обектни модели}
\subsection{Потокови модели}
\subsection{Контекстни модели}
\subsection{Йерархични модели}
\subsection{Асинхронни модели}
\subsection{Интерактивни модели}

\section{Организация на разпределените приложения}

\subsection{Клиент-сървър и двуслойни архитектури}
\subsection{Трислойни архитектури}
\subsection{N-слойни архитектури}
\subsection{Peer-to-Peer архитектури}
\subsection{Сървъри за приложения и web-сървъри}
\subsection{Метасистеми и грид}
\subsection{Сервизно-ориентирани, моделно-ориентирани и аспектно-ориентирани архитектури}
\subsection{Софтуерни агенти}

\end{document}
