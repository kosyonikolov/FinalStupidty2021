
\documentclass[fleqn,12pt]{article}

\usepackage[margin=1cm]{geometry}
\usepackage[utf8]{inputenc}
\usepackage[bulgarian]{babel}
\usepackage{amsfonts}
\usepackage{amssymb}

\usepackage{amsmath}
\DeclareMathOperator{\cotg}{cotg}
\DeclareMathOperator{\LCS}{LCS}
\DeclareMathOperator{\longer}{longer}

\begin{document}
\pagenumbering{gobble}

\Large\center\textbf{Графи. Дървета. Обхождане на графи.}



\begin{flushleft}

\section{Дефиниции за краен ориентиран (мулти)граф и краен неориентиран (мулти)граф.}
Краен ориентиран мултиграф се нарича тройката $G = (V, E, f_G)$, където
\begin{itemize}
	\item $V = \{ v_1, v_2, \dots, v_n \}$ - върхове
	\item $E = \{ e_1, e_2, \dots, e_m \}$ - ребра
	\item $f_G : E \rightarrow V^2$ - свързваща функция
\end{itemize}


\section{Дефиниции за маршрут (контур) в ориентиран мултиграф и път (цикъл) в неориентиран мултиграф.}
\section{Свързаност и свързани компоненти на граф.} 
\section{Дефиниция на дърво и кореново дърво.}
\section{Доказателство, че всяко кореново дърво е дърво и |V|=|E|+1.}
\section{Покриващо дърво на граф.}
\section{Обхождане на граф в ширина и дълбочина.}
\section{Ойлерови обхождания на мултиграф.}
\section{Теореми за съществуване на Ойлеров цикъл (с доказателство) и Ойлеров път.}

\textbf{МК 2} \hspace{5mm}\textit{ФН:} \hspace{5mm} \hspace{20mm} \textit{Група: } \hspace{10mm} \textit{Име: } \\
\vspace{5mm}

\textbf{1. TupleSum:} Даден е масив $A[1..n]$ от положителни и отрицателни цели числа. Можем да групираме съседни числа във двойки, като всяко число може да участва в точно една двойка. Сума на двойките наричаме сумата от произведението на числата във всяка двойка. Каква е максималната такава сума? \\

По-формално: Множество от "двойки" \hspace{1mm} $T$ можем да дефинираме така: $T \subset \{ i \in \mathbb{N} | 1 \leq i < N \}$ и $\forall i \in T \hspace{2mm} (i + 1 \notin T)$. Ако сумата, породена от множество двойки $T$ отбележим с $S(T) = \sum_{i \in T} A[i] . A[i+1]$, то търсим $M$, такова че:

\begin{itemize}
	\item съществува множество двойки $T'$, такова че $S(T')=M$
	\item не съществува множество двойки $T''$, такова че $S(T'') > M$
\end{itemize}

Съставете възможно най-бърз алгоритъм, решаващ задачата, и намерете сложността му. Приложете го върху примера. Може ли да се модифицира така, че да връща и самите двойки? \\
\vspace{5mm}
\textit{Пример: } \begin{tabular}{|c|c|c|c|c|c|c|c|c|c|} \hline -5 & 7 & 4 & -2 & -2 & -8 & 7 & -5 & 4 & 2 \\ \hline \end{tabular}

\vspace{10mm}
\textbf{2. Uni-X: } Геодезичния Робото-Архитектурен Факултет на унивеситета X предлага курсове, които се делят на три групи:

\begin{itemize}
	\item входни - могат да се запишат веднага след влизането във факултета
	\item междинни - всеки има списък с "необходими" курсове. За да се запише курса, е необходимо \textbf{поне един} курс от "необходимите" да е завършен.
	\item изходни - това са междинни курсове, завършването на един от които води до успешно завършване на факултета
\end{itemize}

Общо курсовете са $N$ на брой и са съответно номерирани с числата от 1 до $N$. Знаете множеството $F$ - изходните курсове (естествено $F \subseteq \{1,2, \dots, N\}$). За всеки курс имате $T_i \in \mathbb{N}$ - време за завършване, и множество $R_i$ - "необходимите" курсове за записване на курса $i$. Всички курсове, за които $R_i$ е празно, са входни. Винаги има поне един входен и поне един изходен курс (но е позволено входен и изходен да съвпадат), и винаги е възможно да се завърши. Съставете възможно най-бърз алгоритъм, намиращ минималното време за завършване на факултета. Изразете сложността му чрез $N$ и $M$, където $M = \sum_{i=1}^N |R_i|$.

\vspace{10mm}

\textbf{3. Copy Paste:} Студентите от гореспоменатия факултет имат контролно по Дизайн и Анализ на Градоустройство, което по традиция е в седмициа с още 4 контролни. Затова студентите са измислили следната система: За всяко контролно учи само един човек, а останалите се опитват да препишат от него - включително чрез посредници (напр. А преписва от Б, който преписва от В). Понеже това е все пак университет по архитектура, залата е със изключително странна конфигурация, която затруднява преписването. След множество проведени експерименти (т.е. контролни и изпити) студентите са установили колко добре от всяко място се преписва и от кои съседни. След един провал, състоящ се от цикъл на преписване на все по-грешни решения, студентите се уговорили да преписват само така, че да не може да се получи цикъл. \\
\qquad Иван влиза в залата последен и трябва да си избере място. Той знае множеството на местата $V$, множеството на свободните $K \subset V$, насочените "връзки" на преписване $E$ (между места в залата), колко загуби(неточности) внася всяка връзка $c : E \rightarrow \mathbb{Q} $ и естествено $S$ - мястото, където е седнал единствения човек, който е учил за контролното. Съставате максимално бърз алгоритъм, който помага на Иван да си избере оптимално място (с минимални загуби при преписване).

\end{flushleft}
\end{document}
