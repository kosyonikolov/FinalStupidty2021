
\documentclass[fleqn,12pt]{article}

\usepackage[margin=15mm]{geometry}
\usepackage[utf8]{inputenc}
\usepackage[bulgarian]{babel}
\usepackage[unicode]{hyperref}
\usepackage{amsfonts}
\usepackage{amssymb}
\usepackage{enumitem, hyperref}
\usepackage{upgreek}

\usepackage{amsmath}
\DeclareMathOperator{\cotg}{cotg}
\DeclareMathOperator{\LCS}{LCS}
\DeclareMathOperator{\longer}{longer}
\renewcommand{\arraystretch}{1.3}   

\title{Теореми  за  средните  стойности  (Рол,  Лагранж  и  Коши).  Формула  на Тейлър}
\author{v0.1}
\date{3 юни 2021}

\begin{document}
    
\maketitle

\tableofcontents

\begin{flushleft}
    
\section{Помощни теореми}

\subsection{Определение за диференцируемост на функция}
Функцията $f(x)$ е диференцируема в точката $x_0$ точно тогава, когато притежава лява и дясна производни в $x_0$ и те съвпадат.
\\$\lim_{h \to +0} \frac{f(x_0 + h)-f(x_0)}{h} = \lim_{h \to -0} \frac{ f(x_0 + h)-f(x_0)}{h}$

\subsection{Теорема на Вайерщрас}
Нека a<b и функцията $f:[a,b]\rightarrow\mathbb{R}$ е непрекъсната. Тогава е изпълнено, че:
\begin{itemize}
    \item $f(x)$ е ограничена в [a,b]
    \item $f(x)$ има минимум и максимум в [a,b] 
\end{itemize}

\subsection{Теорема на Ферма}
Ако $f(x)$ има локален екстремум в точка $x_0$ и $f(x)$ е диференцируема в точка $x_0$, то $f'(x_0)=0$

\textbf{Доказателство}
Нека $f(x)$ има локален минимум в т. $x_0$ (Аналогично се доказва за локален максимум, разглеждаме $-f(x)$),
тогава съществува околност на т. $x_0$ от вида $(x_0 - \delta,x_0 + \delta), \delta > 0$, в която стойностите на $f(x)$ са по-големи или равни на $f(x_0)$
е по-големи от стойността на $f(x)$ в т. $x_0$. Понеже $f(x)$ е диференцируема в т. $x_0$, то лявата и дясната производни в т. $x_0$ съвпадат.
\begin{itemize}
    \item $f'(x_0) = \lim_{h \to +0} \frac{ f(x_0 + h)-f(x_0)}{h} \geq 0$ защото $f(x_0 + h) \geq f(x_0)$
    \item $f'(x_0) = \lim_{h \to -0} \frac{ f(x_0 + h)-f(x_0)}{h} \leq 0$ защото $f(x_0 + h) \geq f(x_0)$
\end{itemize}
Следователно, за да съвпадат лявата и дясната производни, то $f'(x_0) = 0$

\section{Теорема на Рол}
Нека $a<b$ и $f:[a,b]\rightarrow\mathbb{R}$ е непрекъсната в $[a,b]$ и диференцируема в $(a,b)$. Ако $f(a) = f(b)$, то съществува точка $c \in (a,b)$ такава, че $f'(c)=0$

\textbf{Доказателство:}
В интервала $[a,b]$ $f(x)$ има минимум и максимум (от теоремата на Вайерщрас).
Следователно $\exists x_0,x_1 \in [a,b]: f(x_0)=\inf_{x \in [a,b]} f(x), f(x_1)=\sup_{x \in [a,b]} f(x)$
Тогава можем да разгледаме следните случаи.

Ако $f(x_0) = f(x_1)$, то $f(x)$ приема една и съща стойност в интервала $[a,b]$. Следователно $\forall x \in (a,b), f'(x)=0$ 

В противен случай, понеже $f(a) = f(b)$, то поне една от точките $x_0,x_1$ принадлежи на $(a,b)$. Б.о.о допускаме, че $x_0 \in (a,b)$.
Понеже $f(x)$ е диференцируема в $(a,b)$ и има локален екстремум $x_0 \in (a,b)$, то $f'(x_0) = 0$ (От теоремата на Ферма) 

С това вариантите се изчерпват и теоремата е доказана

\section{Теорема на Лагранж}
Нека $a<b$ и $f:[a,b]\rightarrow\mathbb{R}$ e непрекъсната в $[a,b]$ и диференцируема в $(a,b)$.
Тогава съществува т. $c \in (a,b): f'(c) = \frac{f(b)-f(a)}{b-a}$

\textbf{Доказателство:}
Нека разгледаме функцията $h(x)=f(x) - \frac{f(b)-f(a)}{b-a}*x$. 
\begin{itemize}
    \item $h(a)=\frac{f(a)*b - f(a)*a - f(b)*a + f(a)*a}{b-a} = \frac{f(a)*b-f(b)*a}{b-a} = \frac{f(b)*b - f(b)*a - f(b)*b + f(a)*b}{b-a} = h(b)$
    \item Функциите $f(x)$ и $-\frac{f(b)-f(a)}{b-a}*x$ са непрекъснати в $[a,b]$, следователно функцията $h(x)$ също е непрекъсната в $[a,b]$.
    \item Функциите $f(x)$ и $-\frac{f(b)-f(a)}{b-a}*x$ са диференцируеми в $(a,b)$, следователно функцията $h(x)$ също е диференцируема в $(a,b)$.
\end{itemize}
От теоремата на Рол, следва че $\exists c \in (a,b): h'(c) = f'(c) - \frac{f(b)-f(a)}{b-a} = 0, f'(c) = \frac{f(b)-f(a)}{b-a}$.
Теоремата е доказана

\section{Теорема на Коши}

\section{Теорема на Тейлър}

\end{flushleft}

\end{document}