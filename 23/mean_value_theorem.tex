
\documentclass[fleqn,12pt]{article}

\usepackage[margin=15mm]{geometry}
\usepackage[utf8]{inputenc}
\usepackage[bulgarian]{babel}
\usepackage[unicode]{hyperref}
\usepackage{amsfonts}
\usepackage{amssymb}
\usepackage{enumitem, hyperref}
\usepackage{upgreek}

\renewcommand{\arraystretch}{1.3}   

\title{Теореми  за  средните  стойности  (Рол,  Лагранж  и  Коши).  Формула  на Тейлър}
\author{v0.1}
\date{3 юни 2021}

\begin{document}
    
\maketitle

\tableofcontents

\begin{flushleft}
    
\section{Помощни теореми}

\subsection{Определение за диференцируемост на функция}
Функцията $f(x)$ е диференцируема в точката $x_0$ точно тогава, когато притежава лява и дясна производни в $x_0$ и те съвпадат.
\[\lim_{h \to +0} \frac{f(x_0 + h)-f(x_0)}{h} = \lim_{h \to -0} \frac{ f(x_0 + h)-f(x_0)}{h}\]

\subsection{Теорема на Вайерщрас}
Нека a<b и функцията $f:[a,b]\rightarrow\mathbb{R}$ е непрекъсната. Тогава е изпълнено, че:
\begin{itemize}
    \item $f(x)$ е ограничена в [a,b]
    \item $f(x)$ има минимум и максимум в [a,b] 
\end{itemize}

\subsection{Теорема на Ферма}
Ако $f(x)$ има локален екстремум в точка $x_0$ и $f(x)$ е диференцируема в точка $x_0$, то $f'(x_0)=0$
\bigbreak
\textbf{Доказателство}
Нека $f(x)$ има локален минимум в т. $x_0$ (Аналогично се доказва за локален максимум, разглеждаме $-f(x)$),
тогава съществува околност на т. $x_0$ от вида $(x_0 - \delta,x_0 + \delta), \delta > 0$, в която стойностите на $f(x)$ са по-големи или равни на $f(x_0)$
Понеже $f(x)$ е диференцируема в т. $x_0$, то лявата и дясната производни в т. $x_0$ трябва да съвпадат.
\[f'(x_0) = \lim_{h \to +0} \frac{ f(x_0 + h)-f(x_0)}{h} \geq 0, f(x_0 + h) \geq f(x_0)\]
\[f'(x_0) = \lim_{h \to -0} \frac{ f(x_0 + h)-f(x_0)}{h} \leq 0, f(x_0 + h) \geq f(x_0)\]
Следователно, за да съвпадат лявата и дясната производни, то $f'(x_0) = 0$

\section{Теореми  за  средните  стойности}
\subsection{Теорема на Рол}
Нека $a<b$ и $f:[a,b]\rightarrow\mathbb{R}$ е непрекъсната в $[a,b]$ и диференцируема в $(a,b)$. Ако $f(a) = f(b)$, то $\exists c \in (a,b): f'(c)=0$
\bigbreak
\textbf{Доказателство:}
В интервала $[a,b]$ $f(x)$ има минимум и максимум (от теоремата на Вайерщрас).
Следователно $\exists x_0,x_1 \in [a,b]: f(x_0)=\inf_{x \in [a,b]} f(x), f(x_1)=\sup_{x \in [a,b]} f(x)$
Тогава можем да разгледаме следните случаи.
\begin{itemize}
    \item Ако $f(x_0) = f(x_1)$, то $f(x)$ приема една и съща стойност в интервала $[a,b]$. Следователно $\forall x \in (a,b), f'(x)=0$ 

    \item В противен случай, понеже $f(a) = f(b)$, то поне една от точките $x_0,x_1$ принадлежи на $(a,b)$. Б.о.о допускаме, че $x_0 \in (a,b)$. Понеже $f(x)$ е диференцируема в $(a,b)$ и има локален екстремум $x_0 \in (a,b)$, то $f'(x_0) = 0$ (От теоремата на Ферма) 
\end{itemize}
С това вариантите се изчерпват и теоремата е доказана

\subsection{Теорема на Лагранж}
Нека $a<b$ и $f:[a,b]\rightarrow\mathbb{R}$ e непрекъсната в $[a,b]$ и диференцируема в $(a,b)$.
Тогава съществува т. $c \in (a,b): f'(c) = \frac{f(b)-f(a)}{b-a}$
\bigbreak
\textbf{Доказателство:}
Нека разгледаме функцията $h(x)=f(x) - \frac{f(b)-f(a)}{b-a}*x$. 
\begin{itemize}
    \item $h(a)=\frac{f(a)*b - f(a)*a - f(b)*a + f(a)*a}{b-a} = \frac{f(a)*b-f(b)*a}{b-a} = \frac{f(b)*b - f(b)*a - f(b)*b + f(a)*b}{b-a} = h(b)$
    \item Функциите $f(x)$ и $-\frac{f(b)-f(a)}{b-a}*x$ са непрекъснати в $[a,b]$, следователно функцията $h(x)$ също е непрекъсната в $[a,b]$.
    \item Функциите $f(x)$ и $-\frac{f(b)-f(a)}{b-a}*x$ са диференцируеми в $(a,b)$, следователно функцията $h(x)$ също е диференцируема в $(a,b)$.
\end{itemize}
От теоремата на Рол, следва че $\exists c \in (a,b): h'(c) = f'(c) - \frac{f(b)-f(a)}{b-a} = 0, f'(c) = \frac{f(b)-f(a)}{b-a}$.
Теоремата е доказана

\subsection{Теорема на Коши}
Нека $a<b$ и $f:[a,b]\rightarrow\mathbb{R}$, $g:[a,b]\rightarrow\mathbb{R}$ са непрекъснати в $[a,b]$ и са диференцируеми в $(a,b)$, $\forall x \in (a,b): g'(x) \neq 0$.
Тогава е изпълнено, че $\exists c \in (a,b): \frac{f'(c)}{g'(c)}=\frac{f(b)-f(a)}{g(b)-g(a)}$.
\bigbreak
\textbf{Доказателство:}
Понеже $g'(x) \neq 0$ в интервала $(a,b)$, то теоремата на Рол не е изпълнена за $g(x)$ в $[a,b]$ и следователно $g(b) \neq g(a)$.
Нека разгледаме функцията $h(x)=f(x) - \frac{f(b)-f(a)}{g(b)-g(a)}*g(x)$.
\begin{itemize}
    \item $h(a)=\frac{f(a)*g(b) - f(a)*g(a) - f(b)*g(a) + f(a)*g(a)}{g(b)-g(a)} = \frac{f(a)*g(b)-f(b)*g(a)}{g(b)-g(a)} = \frac{f(b)*g(b) - f(b)*g(a) - f(b)*g(b) + f(a)*g(b)}{g(b)-g(a)} = h(b)$
    \item Функциите $f(x)$ и $-\frac{f(b)-f(a)}{g(b)-g(a)}*g(x)$ са непрекъснати в $[a,b]$, следователно функцията $h(x)$ също е непрекъсната в $[a,b]$.
    \item Функциите $f(x)$ и $-\frac{f(b)-f(a)}{g(b)-g(a)}*g(x)$ са диференцируеми в $(a,b)$, следователно функцията $h(x)$ също е диференцируема в $(a,b)$.
\end{itemize}
От теоремата на Рол, следва че $\exists c \in (a,b): h'(c) = f'(c) - \frac{f(b)-f(a)}{g(b)-g(a)}*g'(c) = 0, \frac{f'(c)}{g'(c)} = \frac{f(b)-f(a)}{g(b)-g(a)}$.
Теоремата е доказана


\subsection{Теорема на Тейлър}
Нека $a<b$ и $f:[a,b]\rightarrow\mathbb{R}$ e непрекъсната в $[a,b]$ и n+1-кратно диференцируема в $(a,b)$. Тогава за произволна точка $\xi \in (a,b)$ в достатъчна малка нейна околност е изпълнено, че:
\[f(x)=\sum_{k = 0}^{n} \frac{f^{(k)}(\xi)}{k!}*(x-\xi)^k + R(x), \lim_{x \rightarrow \xi} R(x) = 0\]
където $R(x)$ се нарича остатъчен член.
\bigbreak
\textbf{Доказателство:}
Нека $\xi,x \in [a,b]$ и $U=[min(\xi,x),max(\xi,x)]$ и нека $\phi(t) = f(x) - \sum_{k = 0}^{n} \frac{f^{(k)}(t)}{k!}*(x-t)^{k}$.
Тогава $\phi(\xi)=R(x)$, т.е чрез $\phi$ можем да изразим и остатъчния член.
\bigbreak
$\phi'(t)=0 - \sum_{k=0}^{n} \frac{f^{(k+1)}(t)}{k!}*(x-t)^{k} +  \sum_{k=0}^{n}\frac{f^{(k)}(t)}{(k-1)!}*(x-t)^{k-1} =$
$=\sum_{k=1}^{n}\frac{f^{(k)}(t)}{(k-1)!}*(x-t)^{k-1} - \sum_{k=0}^{n} \frac{f^{(k+1)}(t)}{k!}*(x-t)^{k} = -\frac{f^{(n+1)}(t)}{n!}*(x-t)^n$
\bigbreak
За да изразим остатъчния член в различни формати, то ще използваме помощна функция $\psi:U\rightarrow\mathbb{R}$, която е непрекъсната в $U$ и диференцируема във вътрешността на $U$ като $\psi'(t) \neq 0 : t \in U$.
От теоремата на Коши следва че: $\exists c \in$ вътрешността на $U: \frac{\phi'(c)}{\psi'(c)}=\frac{\phi(x)-\phi(\xi)}{\psi(x)-\psi(\xi)}$.
\bigbreak
Понеже $\phi(\xi) = R(x), \phi(x) = 0$, то $R(x) = -\frac{\phi'(c)*(\psi(x)-\psi(\xi))}{\psi'(c)} = \frac{\psi(x)-\psi(\xi)}{\psi'(c)}*\frac{f^{(n+1)}(c)}{n!}*(x-c)^n$\newline
От непрекъснатостта на $\psi(x)$ следва че: $\lim_{x->\xi} R(x) = \lim_{x->\xi} \frac{\psi(x)-\psi(\xi)}{\psi'(c)}*\frac{f^{(n+1)}(c)}{n!}*(x-c)^n = 0$
\bigbreak
Казваме, че $R(x)$ e във форма на Лагранж ако $\psi(t)=(x-t)^{n+1}, \psi'(t)=-(n+1)*(x-t)^n$
Следователно $R(x)=\frac{0-(x-\xi)^{n+1}}{-(n+1)*(x-c)^n}*\frac{f^{(n+1)}(c)}{n!}*(x-c)^n = \frac{f^{(n+1)}(c)}{(n+1)!}*(x-\xi)^{n+1}$


\end{flushleft}

\end{document}