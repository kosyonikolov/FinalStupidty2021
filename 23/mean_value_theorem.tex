\documentclass[fleqn,12pt]{article}

\usepackage[margin=15mm]{geometry}
\usepackage[utf8]{inputenc}
\usepackage[bulgarian]{babel}
\usepackage[unicode]{hyperref}
\usepackage{amsthm}
\usepackage{amssymb}
\usepackage{mathtools}
\usepackage[unicode]{hyperref}
\usepackage{enumitem, hyperref}
\usepackage{indentfirst}

\renewcommand{\arraystretch}{1.3}   

\title{Теореми  за  средните  стойности  (Рол,  Лагранж  и  Коши).  Формула  на Тейлър}
\author{v1.0}
\date{18 юни 2021}

\begin{document}
    
\maketitle

\tableofcontents
\pagebreak
    
\section{Помощни теореми}

\subsection{Определение за диференцируемост на функция}
Функцията $f(x)$ е диференцируема в точката $x_0$ точно тогава, когато притежава лява и дясна производни в $x_0$ и те съвпадат.
\[f'(x) = \lim_{h \to +0} \frac{f(x_0 + h)-f(x_0)}{h} = \lim_{h \to -0} \frac{ f(x_0 + h)-f(x_0)}{h}\]

\subsection{Теорема на Вайерщрас}
Нека $a<b$ и функцията $f:[a,b]\rightarrow\mathbb{R}$ е непрекъсната. Тогава е изпълнено, че $f(x)$ има точна горна и долна граница в $[a,b]$
т.е $\exists x_0,x_1 \in [a,b]: f(x_0)=\inf_{x \in [a,b]} f(x), f(x_1)=\sup_{x \in [a,b]} f(x)$.

\subsection{Oпределение за околност на точка}
Всеки отворен интервал $(a,b)$, който съдържа дадена точка $x$, ще наричаме околност на $x$. Ще казваме, че $x$ е вътрешна за $(a,b)$, ако
съществува околност $(x-\delta,x+\delta) \subset (a,b), \delta > 0$.

\subsection{Теорема на Ферма}
Ако $f(x)$ има локален екстремум в точка $x_0$ и $f(x)$ е диференцируема в точка $x_0$, то $f'(x_0)=0$
\bigbreak
\textbf{Доказателство}
Нека $f(x)$ има локален минимум в т. $x_0$ (Аналогично се доказва за локален максимум, разглеждаме $-f(x)$),
тогава съществува околност $(x_0 - \delta,x_0 + \delta), \delta > 0$, такава че $|x-x_0| < \delta \Rightarrow f(x) \geq f(x_0)$.
Нека $|h| < |\delta|$ и разгледаме отношението $L(h)$:
\[ L(h) = \frac{f(x_0 + h) - f(x_0)}{h} \]

За $h \in (0, \delta)$ получаваме $f(x_0 + h) \geq f(x_0) \Leftrightarrow f(x_0 + h) - f(x_0) \geq 0 \Leftrightarrow L(h) \geq 0$.
При $h \in (-\delta, 0)$ отново имаме $f(x_0 + h) \geq f(x_0) \Leftrightarrow f(x_0 + h) - f(x_0) \geq 0$,
но понеже $h < 0 \Rightarrow L(h) \leq 0$.

Понеже $f(x)$ е диференцируема в т. $x_0$, то в $x_0$ трябва да съществува границата
\[ \lim_{h \to 0} \frac{ f(x_0 + h)-f(x_0)}{h} = L(h) = f'(x_0) \]

Следователно, за да съществува производната, трябва $L(h) = 0$ за $|h| < |\delta| \Rightarrow f'(x_0) = 0$.

\section{Теореми  за  средните  стойности}
\subsection{Теорема на Рол}
Нека $a<b$ и $f:[a,b]\rightarrow\mathbb{R}$ е непрекъсната в $[a,b]$ и диференцируема в $(a,b)$. Ако $f(a) = f(b)$, то $\exists c \in (a,b): f'(c)=0$
\bigbreak
\textbf{Доказателство:}
В интервала $[a,b]$ $f(x)$ е ограничена отгоре и отдолу (от теоремата на Вайерщрас).
Следователно $\exists x_0,x_1 \in [a,b]: f(x_0)=\inf_{x \in [a,b]} f(x), f(x_1)=\sup_{x \in [a,b]} f(x)$
Тогава можем да разгледаме следните случаи.
\begin{itemize}
    \item Ако $f(x_0) = f(x_1)$, то $f(x)$ приема една и съща стойност в интервала $[a,b]$. Следователно $\forall x \in (a,b), f'(x)=0$ 

    \item В противен случай, понеже $f(a) = f(b)$, то поне една от точките $x_0,x_1$ принадлежи на $(a,b)$. Б.о.о допускаме, че $x_0 \in (a,b)$. Понеже $f(x)$ е диференцируема в $(a,b)$ и има локален екстремум $x_0 \in (a,b)$, то $f'(x_0) = 0$ (От теоремата на Ферма) 
\end{itemize}
С това вариантите се изчерпват и теоремата е доказана

\subsection{Теорема на Лагранж}
Нека $a<b$ и $f:[a,b]\rightarrow\mathbb{R}$ e непрекъсната в $[a,b]$ и диференцируема в $(a,b)$.
Тогава съществува т. $c \in (a,b): f'(c) = \frac{f(b)-f(a)}{b-a}$.
\bigbreak
\textbf{Доказателство:}
Нека разгледаме функцията $h(x)=f(x) + g(x)$, където $g(x) = \frac{f(b)-f(a)}{b-a}x$. Функциите $f(x)$ и $g(x)$ са непрекъснати в $[a,b]$, 
следователно и $h(x)$ е непрекъсната в $[a,b]$. Двете функции са и диференцируеми в $(a,b) \Rightarrow h(x)$ също е диференцируема в $(a,b)$.
Ще покажем, че $h(a) = h(b)$: 
\[ h(a)=\frac{f(a)b - f(a)a - f(b)a + f(a)a}{b-a} = \frac{f(a)b-f(b)a}{b-a} = \frac{f(b)b - f(b)a - f(b)b + f(a)b}{b-a} = h(b) \]

От теоремата на Рол, следва че $\exists c \in (a,b)$, такова че  $h'(c) = 0 \Leftrightarrow f'(c) - \frac{f(b)-f(a)}{b-a} = 0 \Leftrightarrow$
$f'(c) = \frac{f(b)-f(a)}{b-a}$, с което теоремата е доказана.

\subsection{Теорема на Коши}
Нека $a<b$ и $f,g:[a,b]\rightarrow\mathbb{R}$ са непрекъснати в $[a,b]$ и са диференцируеми в $(a,b)$, $\forall x \in (a,b): g'(x) \neq 0$.
Тогава е изпълнено, че $\exists c \in (a,b): \frac{f'(c)}{g'(c)}=\frac{f(b)-f(a)}{g(b)-g(a)}$.
\bigbreak
\textbf{Доказателство:}
Първо ще покажем, че $g(b) \neq g(a)$. Ако това не е изпълено, съществува $c \in (a,b): g'(c) = 0$ - по теорема на Рол. Но по условие $g'(x) \neq 0$ за $x \in (a,b) \Rightarrow$ противоречие.
Нека разгледаме функцията $h(x)=f(x) - \frac{f(b)-f(a)}{g(b)-g(a)}g(x)$.
\[ h(a)=\frac{f(a)g(b) - f(a)g(a) - f(b)g(a) + f(a)g(a)}{g(b)-g(a)} = \frac{f(a)g(b)-f(b)g(a)}{g(b)-g(a)} =  \]
\[ = \frac{f(b)g(b) - f(b)g(a) - f(b)g(b) + f(a)g(b)}{g(b)-g(a)} = h(b) \]
\begin{itemize}
    \item Функциите $f(x)$ и $-\frac{f(b)-f(a)}{g(b)-g(a)}g(x)$ са непрекъснати в $[a,b]$, следователно функцията $h(x)$ също е непрекъсната в $[a,b]$.
    \item Функциите $f(x)$ и $-\frac{f(b)-f(a)}{g(b)-g(a)}g(x)$ са диференцируеми в $(a,b)$, следователно функцията $h(x)$ също е диференцируема в $(a,b)$.
\end{itemize}

От теоремата на Рол, следва че $\exists c \in (a,b): h'(c) = f'(c) - \frac{f(b)-f(a)}{g(b)-g(a)}g'(c) = 0, \frac{f'(c)}{g'(c)} = \frac{f(b)-f(a)}{g(b)-g(a)}$.
Теоремата е доказана.


\subsection{Теорема на Тейлър}
Нека $a<b$ и $f:[a,b]\rightarrow\mathbb{R}$ e непрекъсната в $[a,b]$ и n+1-кратно диференцируема в $(a,b)$. 
Тогава за произволна точка $\xi \in (a,b)$ в достатъчна малка нейна околност е изпълнено, че:
\[f(x)=\sum_{k = 0}^{n} \frac{f^{(k)}(\xi)}{k!}(x-\xi)^k + R(x) \text{, } \lim_{x \rightarrow \xi} R(x) = 0\]
където $R(x)$ се нарича остатъчен член.

\textbf{Доказателство:}
Нека $\xi,x \in (a,b)$ и нека $\phi(t) = f(x) - \sum_{k = 0}^{n} \frac{f^{(k)}(t)}{k!}(x-t)^{k}$.
Тогава $\phi(\xi)=R(x)$, т.е чрез $\phi$ можем да изразим и остатъчния член.

\[\phi'(t)=0 - \sum_{k=0}^{n} \frac{f^{(k+1)}(t)}{k!}(x-t)^{k} +  \sum_{k=0}^{n}\frac{f^{(k)}(t)}{(k-1)!}(x-t)^{k-1} = \]
\[ =\sum_{k=1}^{n}\frac{f^{(k)}(t)}{(k-1)!}(x-t)^{k-1} - \sum_{k=0}^{n} \frac{f^{(k+1)}(t)}{k!}(x-t)^{k} = -\frac{f^{(n+1)}(t)}{n!}(x-t)^n \]

За да изразим остатъчния член в различни формати, то ще дефинираме множеството $U=[\min(\xi,x),\max(\xi,x)]$ и ще използваме помощна функция $\psi:U\rightarrow\mathbb{R}$, която е непрекъсната в $U$
и диференцируема във вътрешността на $U$ като $\psi'(t) \neq 0 : t \in U$.\\
От теоремата на Коши следва че съществува точка $c$ от вътрешността на $U: \frac{\phi'(c)}{\psi'(c)}=\frac{\phi(x)-\phi(\xi)}{\psi(x)-\psi(\xi)}$.
Забелязваме, че
\[ \phi(t) = f(x) - \frac{f^{(0)}(t)}{0!}(x-t)^0 - \sum_{k=1}^{n}\frac{f^{(k)}(t)}{k!}(x-t)^k = \] 
\[ = f(x) - f(t) - \sum_{k=1}^{n}\frac{f^{(k)}(t)}{k!}(x-t)^k \]

Тогава за $\phi(x)$ получаваме $\phi(x) = f(x) - f(x) - \sum_{k=1}^{n}\frac{f^{(k)}(x)}{k!}(x-x)^k = 0$

Щом $\phi(\xi) = R(x)$ и $\phi(x) = 0$, то 
\[R(x) = -\frac{\phi'(c)(\psi(x)-\psi(\xi))}{\psi'(c)} = \frac{\psi(x)-\psi(\xi)}{\psi'(c)}\frac{f^{(n+1)}(c)}{n!}(x-c)^n\]

От непрекъснатостта на $\psi(x)$ следва че: $\lim_{x\to\xi} R(x) = \lim_{x\to\xi} \frac{\psi(x)-\psi(\xi)}{\psi'(c)}\frac{f^{(n+1)}(c)}{n!}(x-c)^n = 0$.
Следователно в околност на $\xi$ e изпънено, че:
\[
\begin{cases}
    f(x) = \sum_{k = 0}^{n} \dfrac{f^{(k)}(t)}{k!}(x-t)^{k} + \phi(\xi) = \sum_{k = 0}^{n} \frac{f^{(k)}(t)}{k!}(x-t)^{k} + R(x)\\
    \lim_{x\to\xi} R(x) = 0
\end{cases}
\]
c което доказахме теоремата. 

Казваме, че формулата на Тейлър e във форма на Лагранж, ако 
\[ \psi(t)=(x-t)^{n+1}, \psi'(t)=-(n+1)(x-t)^n \Rightarrow \]
\[ \Rightarrow R(x)=\frac{0-(x-\xi)^{n+1}}{-(n+1)(x-c)^n}\frac{f^{(n+1)}(c)}{n!}(x-c)^n = \frac{f^{(n+1)}(c)}{(n+1)!}(x-\xi)^{n+1} \]

\end{document}