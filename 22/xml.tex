
\documentclass[fleqn,12pt]{article}

\usepackage[margin=15mm]{geometry}
\usepackage[utf8]{inputenc}
\usepackage[bulgarian]{babel}
\usepackage[unicode]{hyperref}
\usepackage{amsfonts}
\usepackage{amssymb}
\usepackage{enumitem, hyperref}
\usepackage{upgreek}
\usepackage{indentfirst}
\usepackage{graphicx}

\usepackage{amsmath}

\graphicspath{ {./img/} }

\title{Тема 22 \\Използване на XML за структуриране, валидация, обработка и представяне на документно съдържание.}

\author{v0.1}
\date{30 юни 2021}

\begin{document}

\maketitle
\tableofcontents
\pagebreak

\section{Добре структуриран XML}

\subsection{Основни концепции}
\subsection{Йерархии}
\subsection{Синтактични правила}
\subsection{XML пространства от имена}

\section{XML валидация чрез Document Type Defintions (DTD)}

\subsection{Цели на валидирането}
\subsection{DTD структура}
\subsection{Синтаксис}

\section{XML валидация чрез XML Schema}

\subsection{Спецификации}
\subsection{Типове данни}
\subsection{Фасети}
\subsection{Структури}
\subsection{Сравнение с DTD}

\section{Алокиране, манипулиране и представяне на XML съдържание}

\subsection{Използване на XSLT (eXtensible StyleSheet Language Transformations)}
\subsection{XPath}

\section{Обработка на XML документи}

\subsection{Основни интерфейси и начини на използване на DOM (Document Object Model)}
\subsection{Основни интерфейси и начини на използване на SAX (Simple API for XML)}
\subsection{Сравнение между DOM и SAX}


\end{document}
