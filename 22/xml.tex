
\documentclass[fleqn,12pt]{article}

\usepackage[margin=15mm]{geometry}
\usepackage[utf8]{inputenc}
\usepackage[bulgarian]{babel}
\usepackage[unicode]{hyperref}
\usepackage{amsfonts}
\usepackage{amssymb}
\usepackage{enumitem, hyperref}
\usepackage{upgreek}
\usepackage{indentfirst}
\usepackage{graphicx}

\usepackage{amsmath}
\usepackage{listings}
\usepackage{xcolor}

\definecolor{codegreen}{rgb}{0,0.6,0}
\definecolor{codegray}{rgb}{0.5,0.5,0.5}
\definecolor{codepurple}{rgb}{0.58,0,0.82}
\definecolor{backcolour}{rgb}{0.95,0.95,0.92}

\lstdefinestyle{mystyle}{
    backgroundcolor=\color{backcolour},   
    commentstyle=\color{codegreen},
    keywordstyle=\color{magenta},
    numberstyle=\tiny\color{codegray},
    stringstyle=\color{codepurple},
    basicstyle=\ttfamily\footnotesize,
    breakatwhitespace=false,         
    breaklines=true,                 
    captionpos=b,                    
    keepspaces=true,                 
    numbers=left,                    
    numbersep=5pt,                  
    showspaces=false,                
    showstringspaces=false,
    showtabs=false,                  
    tabsize=2
}
\lstset{style=mystyle}

\graphicspath{ {./img/} }

\title{Тема 22 \\Използване на XML за структуриране, валидация, обработка и представяне на документно съдържание.}

\author{v0.1}
\date{30 юни 2021}

\begin{document}

\maketitle
\tableofcontents
\pagebreak

\section{Добре структуриран XML}

\subsection{Основни концепции}

\textbf{XML таг} наричаме маркер (име на елемент), което е отделено със символите < и >. Примерен таг е \textbf{<hello-world>}.
\bigbreak

\textbf{XML елемент} се състои от отварящ таг, съдържание на елемента и затварящ таг. Примерен елемент е \textbf{<b>how bold of you</b>}.
\bigbreak

\textbf{XML атрибути} наричаме двойки име/стойност, асоциирани с елемент.
Добавят се в отварящия таг със стойности, поставени между ' ' или '' ''.
Тяхната практичнта стойност е обвързана с конфигурация на елементите.
Пример за поставяне на атрибут е \textbf{<div color='red'></div>}.
\bigbreak

\textbf{XML документ} е текст под формата на един или повече елементи.
\bigbreak

\textbf{XML markup} се състои от всички тагове на даден XML документ.
\bigbreak

\textbf{XML инструкциите} са конструкции предоставящи метаданни на външни приложения. Записваме ги като \textbf{<?$\dots$?>}. Пример за това са инструкциите при php.
\bigbreak

\textbf{XML декларация} наричаме конструкцията \\\textbf{<?xml version='1.0'} \textbf{encoding='UTF-16'} \textbf{standalone='yes' ?>}, която указва версията на XML спецификацията на документа, кодирането на документа и дали е асоцииран с външно DTD.


\subsection{Йерархии}

XML документите могат да имат \textbf{дървовидно} представяне, като отделните възли са XML елементи.
Всеки документ има точно един \textbf{root} елемент, който няма родител, в който се съдържат всички други елементи.

Структурата на XML документите е следната:
\begin{itemize}
    \item \textbf{Пролог} - декларации, стилове и тип на документа;
    \item \textbf{Екземпляр} - конкретна елемента йерархия;
    \item \textbf{Допълнения} - коментари, CDATA секции и инструкции за обработка;
\end{itemize}


\subsection{Синтактични правила}

XML документите трябва да следват следните правила, за да бъдат правилно структурирани:
\begin{enumerate}
    \item \textbf{Затвореност} - всеки затваряш таг трябва да има съвпадащ затварящ таг или да бъде self-closing;
    \item \textbf{Без застъпване} - таговете не могат да се застъпват, като елементите трябва да са подходящо вложени. Некоректно е например \textbf{<a><b></a></b>}.
    \item \textbf{One root to rule them all} - xml документите могат да имат само 1 корен;
    \item \textbf{XML конвенции} - имената на таговете трябва да започват с малка буква, главна буква или ''-'', като не могат да започват с думите xml, XML, Xml, xmL, \dots;
    \item \textbf{Whitespaces} - XML запазва празните пространства в PCDATA;
    \item \textbf{Всеки атрибут трябва задължително да има стойност}, дори и тя да е празен низ;
    \item \textbf{Без запазени XML символи в съдържанието} - те се вмъкват чрез предварително дефинирани entities или \textbf{<![CDATA[\dots]]>} секции.
\end{enumerate}

\subsection{XML пространства от имена}

Пространствата от имена (namespaces) са абстракна нотация за група от имена.
Едно име може да принадлежи само на едно пространство от имена.
Целта е да се осигури уникалност на имената на XML елементите посредством използването на URI (Uniform Resource Identifier).
За осигуряване на уникалността на префиксите, означаващи пространства от имена се използва само URL (Uniform Resource Locator).
URL не е нужно да е реален, т.е. можем да си използваме каквито си искаме.
\bigbreak
Пространствата от имена за дадена елементна йерархия се задават чрез атрибута xmlns върху корена ѝ, като те важат за всички деца на корена.
Може да съществува едно пространство от имена, наречено такова по подразбиране, което се дефинириа с xmlns без префикс.
Тъй като атрибутите са обвързани с елементите, към които са дефинирани, ако елемент е зададен с префикс на дадено пространство от имена, то не е задължително атрибутите му да са зададени чрез него.

\begin{lstlisting}[language=XML, caption=Doom namespace]
<?xml version="1.0" encoding="utf-8" ?>

<config xmlns="http://config.com" xmlns:doom="http://app.com/myapp">
    <doom:levels>
        <doom:level>
            <doom:npc name="cacodemon" type="regular">
                <doom:count>54<doom:count>
                <doom:power>100<doom:power>
                <doom:health>150<doom:health>
            <doom:npc>
            <doom:npc name="mancubus" type="regular">
                <doom:count>108<doom:count>
                <doom:power>150<doom:power>
                <doom:health>90<doom:health>
            <doom:npc>
        <doom:level>
    </doom:levels>
</config>
\end{lstlisting}


\section{XML валидация чрез Document Type Defintions (DTD)}

\subsection{Цели на валидирането}
\subsection{DTD структура}
\subsection{Синтаксис}

\section{XML валидация чрез XML Schema}

\subsection{Спецификации}
\subsection{Типове данни}
\subsection{Фасети}
\subsection{Структури}
\subsection{Сравнение с DTD}

\section{Алокиране, манипулиране и представяне на XML съдържание}

\subsection{Използване на XSLT (eXtensible StyleSheet Language Transformations)}
\subsection{XPath}

\section{Обработка на XML документи}

\subsection{Основни интерфейси и начини на използване на DOM (Document Object Model)}
\subsection{Основни интерфейси и начини на използване на SAX (Simple API for XML)}
\subsection{Сравнение между DOM и SAX}


\end{document}
