
\documentclass[fleqn,12pt]{article}

\usepackage[margin=15mm]{geometry}
\usepackage[utf8]{inputenc}
\usepackage[bulgarian]{babel}
\usepackage[unicode]{hyperref}
\usepackage{amsfonts}
\usepackage{amssymb}
\usepackage{enumitem, hyperref}
\usepackage{upgreek}
\usepackage{indentfirst}
\usepackage{graphicx}

\usepackage{amsmath}

\graphicspath{ {./img/} }

\title{Тема 22 \\Използване на XML за структуриране, валидация, обработка и представяне на документно съдържание.}

\author{v0.1}
\date{30 юни 2021}

\begin{document}

\maketitle
\tableofcontents
\pagebreak

\section{Добре структуриран XML}

\subsection{Основни концепции}

\textbf{XML таг} наричаме маркер (име на елемент), което е отделено със символите < и >. Примерен таг е \textbf{<hello-world>}.
\bigbreak

\textbf{XML елемент} се състои от отварящ таг, съдържание на елемента и затварящ таг. Примерен елемент е \textbf{<b>how bold of you</b>}.
\bigbreak

\textbf{XML атрибути} наричаме двойки име/стойност, асоциирани с елемент.
Добавят се в отварящия таг със стойности, поставени между ' ' или '' ''.
Тяхната практичнта стойност е обвързана с конфигурация на елементите.
Пример за поставяне на атрибут е \textbf{<div color='red'></div>}.
\bigbreak

Възможно е някои елементи да се състоят от единствен таг, където съдържанието е посочено чрез атрибути. Например \textbf{<self-contained content='Hello world!'>}.
\bigbreak

\textbf{XML документ} е текст под формата на един или повече елементи.
\bigbreak

\textbf{XML markup} се състои от всички тагове на даден XML документ.
\bigbreak

\textbf{XML инструкциите} са конструкции предоставящи метаданни на външни приложения. Записваме ги като \textbf{<?$\dots$?>}. Пример за това са инструкциите при php.
\bigbreak

\textbf{XML декларация} наричаме конструкцията \\\textbf{<?xml version='1.0'} \textbf{encoding='UTF-16'} \textbf{standalone='yes'>}, която указва версията на XML спецификацията на документа, кодирането на документа и дали е асоцииран с външно DTD.


\subsection{Йерархии}
\subsection{Синтактични правила}
\subsection{XML пространства от имена}

\section{XML валидация чрез Document Type Defintions (DTD)}

\subsection{Цели на валидирането}
\subsection{DTD структура}
\subsection{Синтаксис}

\section{XML валидация чрез XML Schema}

\subsection{Спецификации}
\subsection{Типове данни}
\subsection{Фасети}
\subsection{Структури}
\subsection{Сравнение с DTD}

\section{Алокиране, манипулиране и представяне на XML съдържание}

\subsection{Използване на XSLT (eXtensible StyleSheet Language Transformations)}
\subsection{XPath}

\section{Обработка на XML документи}

\subsection{Основни интерфейси и начини на използване на DOM (Document Object Model)}
\subsection{Основни интерфейси и начини на използване на SAX (Simple API for XML)}
\subsection{Сравнение между DOM и SAX}


\end{document}
