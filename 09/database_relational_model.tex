\documentclass[fleqn,12pt]{article}

\usepackage[margin=15mm]{geometry}
\usepackage[utf8]{inputenc}
\usepackage[bulgarian]{babel}
\usepackage[unicode]{hyperref}
\usepackage{amsfonts}
\usepackage{amssymb}
\usepackage{indentfirst}
\usepackage{enumitem, hyperref}
\usepackage{blindtext}
\usepackage{multicol}

\usepackage{amsmath}
\usepackage{listings}
\usepackage{xcolor}


\definecolor{codegreen}{rgb}{0,0.6,0}
\definecolor{codegray}{rgb}{0.5,0.5,0.5}
\definecolor{codepurple}{rgb}{0.58,0,0.82}
\definecolor{backcolour}{rgb}{0.95,0.95,0.92}

\lstdefinestyle{mystyle}{
    backgroundcolor=\color{backcolour},   
    commentstyle=\color{codegreen},
    keywordstyle=\color{magenta},
    numberstyle=\tiny\color{codegray},
    stringstyle=\color{codepurple},
    basicstyle=\ttfamily\footnotesize,
    breakatwhitespace=false,         
    breaklines=true,                 
    captionpos=b,                    
    keepspaces=true,                 
    numbers=left,                    
    numbersep=5pt,                  
    showspaces=false,                
    showstringspaces=false,
    showtabs=false,                  
    tabsize=2
}
\lstset{style=mystyle}

\title{Бази от данни. Релационен модел на данните}
\author{v1.0}
\date{14 юни 2021}

\begin{document}

\maketitle

\tableofcontents
\pagebreak

\section{Релационен модел}

\textit{\textbf{Релационният модел}} е модел за управление на данни, където те са представени като кортежи групирани в релации.
База от данни, която използва релационния модел се нарича релационна база от данни, а СУБД управляващо релационна база от данни е релационна СУБД.
\bigbreak
Основната цел на този модел е да предостави \textbf{декларативен} начин за представяне на данните и заявките към тях, скривайки реалната имплементация.
Следователно изборът на структурите от данни и алгоритмите за това е резервирана функционалност за релационните СУБД-та.

\subsection{Домейн}
% the egg and chicken problem

\textbf{\textit{Домейн (Област)}} наричаме множество от допустими стойности за някаква величина.
\bigbreak
Релационният модел изисква стойностите на всички компоненти на кортежите да бъдат \textbf{атомарни}.
Обичайно домейнът дефинира скаларния тип на данните, както и някои допълнителни ограничения.

\subsection{Релация}

\textbf{\textit{Релация}} по дефиниция на Код наричаме всяко подмножество на декартово произведение $D_1 \times \dots \times D_n, n \in \mathbb{N}$.
В контекста на базите от данни релация може да се разглежда като таблица, където всеки ред е наредена n-торка съдържаща запис от данни, а стълбовете съдържат елементи само от една и съща област.

\subsection{Кортеж}

\textbf{\textit{Кортеж}} наричаме всеки елемент на $R$, където $R$ е релация, т.е. кортеж е наредена n-торка.
Щом $R$ е множество, то следователно кортежите в едно множество $R$ трябва да са уникални.
\bigbreak
Множеството от всички кортежи на дадена релация наричаме \textit{\textbf{екземпляр на релацията}}.
Екземплярите на релациите се променят при променяне, добавяне и премахване на кортежи в тях.
Броят на кортежите в един екземпляр на релация наричаме негова кардиналност.
\bigbreak
Релационна база от данни наричаме множество от релации.
Всяко множество от конкретни екземпляри на релации се нарича текущо състояние на релационната база от данни.

\subsection{Атрибути}

Нека $R$ е някоя релация и $R \subseteq D_1 \times \dots \times D_n, n \in \mathbb{N}$.
Всеки стълб на табличното представяне на $R$ притежава име, което наричаме \textbf{\textit{атрибут}}.
Следователно всеки домейн $D_i$ е асоцииран с някой атрибут $A_i$, където $i \in \mathbb{N}, i \in [1, n]$.
\bigbreak
От семантична/практична гледна точка атрибутите се използват за описване на характеристиките на складираните данни.

\subsection{Схема на релация и релационна база от данни}

Името на една релация и множеството от нейните атрибути наричаме \textbf{\textit{схема на релацията}}.
Прието е схемата на релация $R$ с атрибути $\{A_1, \dots, A_n\}$ да означаваме като $R(A_1, \dots, A_n)$.
\bigbreak
Въпреки, че атрибутите в една схема са множество, а не списък, е практично да се въведе $``$стандартна$''$ наредба.
При схема $R(A_1, \dots, A_n)$ стандартната наредба е $A_1, \dots, A_n$.
\bigbreak
При релационния модел, дизайнът на една база от данни се състои от множество от схеми на релации, което наричаме \textbf{\textit{схема на релационната база от данни (или просто схема на базата от данни)}}.

\subsection{Реализация на релационна база от данни}

Първата фаза от проектирането на една релационна БД обичайно се фокусира над изграждането на даннов модел.
Тя може да се раздели на следните стъпки:
\begin{enumerate}
    \item Дефиниране на това каква информация ще се съхранява.
    \item Дефиниране на зависимости (връзки) между информационните елементи.
    \item Определяне на ограничения над данните като например ключове посредством зависимостите, възможни стойности и прочие.
\end{enumerate}

Важно е при изграждането на данновия модел да се моделира коректно и просто реалния свят, да се избягват излишества и да се подберат правилните елементи и връзки между тях.
Обичайно продуктът от тази фаза се изразява в \textbf{модел Същност/Връзка (Entity/Relationship (E/R))}.
\bigbreak
Следващата фаза е данновият модел да се транслира до релационна схема на БД.
При транслацията от \textbf{E/R} модел към релационна схема на БД първата подфаза се състои от:
\begin{enumerate}
    \item превръщане на всяка същност в релация със същите атрибути.
    \item заместване на всяка връзка с релация, където атрибутите са ключовете на свързаните същности.
\end{enumerate}

Обичайно това е достатъчно, но понякога има изключения като например, че:
\begin{itemize}
    \item транслацията на $``$слабите$''$ същности към релации не е тривиална.
    \item $``Isa''$ връзките изискват специално внимание.
    \item понякога е добре да слеем две релации в една, като тези произлизащи от същност \textbf{E} и нейна връзка \textbf{R} с друга същност \textbf{F} където:
    \begin{itemize}
        \item връзката \textbf{R} е тип много към 1 съответно от \textbf{E} към \textbf{F}.
        \item релациите от \textbf{E} и \textbf{R} съдържат атрибутите от ключа на същността \textbf{E}.
        \item новополучената релация съдържа атрибутите на \textbf{E} и \textbf{R}, като в това число и ключа на \textbf{F}.
    \end{itemize}
\end{itemize}

Възможен е и алтернативен подход, който включва директно проектиране на данновия модел като релационен, но той е ограничен и неудобен.


\subsection{Видове операции върху релационната база данни}

Релационните СУБД-та най-често използват езика \textbf{SQL} за извършване на операции в бази от данни.
Той е декларативен и се състои от две части \textbf{Data Definition Language (DDL)} и \textbf{Data Manipulation Language (DML)}.
\bigbreak
\textbf{DDL} се използва за дефиниция на схемата на базата от данни, като това включва добавяне и изтриване на релации, модифициране на вече съществуващи такива чрез добавяне/промяна/изтриване на атрибути, както и поставяне на ограничения.
Имплементират се посредством \textbf{DDL} становища съдържащи операциите \textbf{CREATE} (създай), \textbf{DROP} (премахни) и \textbf{ALTER} (промени).
\bigbreak
\textbf{DML} се използва за манипулация на кортежите чрез операциите селектиране, добавяне, промяна и изтриване, които се имплементират съответно чрез \textbf{DML} становища съдържащи операциите \textbf{SELECT, INSERT, UPDATE и DELETE}.

\subsection{Заявки към релационната база данни}

\bigbreak
Примерни \textbf{DDL} заявки са:

\begin{lstlisting}[language=SQL, caption=DLL queries example]
-- Create a relation Actors(actor_id, name, age) with a primary key actor_id.
CREATE TABLE Actors(
    actor_id INTEGER PRIMARY_KEY,
    name VARCHAR(50) NOT_NULL,
    age INTEGER NOT_NULL,
);

-- Add an optional attribute called height.
ALTER TABLE Actors ADD height INTEGER;

-- Delete the height attrib.
ALTER TABLE Actors DROP COLUMN height;

-- Delete the Actors relation.
DROP TABLE Actors;
\end{lstlisting}


Примерни \textbf{DML} заявки са:

\begin{lstlisting}[language=SQL, caption=DML queries example]
-- Assume a relation Actors(actor_id, name, age) with a key {actor_id}
-- Assume a relation Movies(movie_id, name, length_in_min) with a key {movie_id}
-- Assume a relation Roles(movie_id, actor_id, name) with a key {movie_id, actor_id}

-- Extract the names and role names of all actors aged upon 30 that have participated in Titanic.
SELECT Actors.actor_name, Roles.name FROM Actors
JOIN Roles ON Actors.actor_id=Roles.actor_id
JOIN Movies ON Roles.movie_id=Movies.movie_id
WHERE Actors.age > 30 AND Movies.name="Titanic";

-- Add a Movie named The Flintstones with a length of 2 hours.
INSERT INTO Movies (movie_id, name, length_in_min)
VALUES (3, "The Flintstones", 125);

-- Update the movie named Aladin to have a lenth of 3 hours.
UPDATE Movies SET length_in_min=180 WHERE name="Aladin";

-- Delete the move named Terminator.
DELETE FROM Movies WHERE name="Terminator";
\end{lstlisting}


\section{Релационна алгебра}

\textit{\textbf{Релационната алгебра (РА)}} е алгебра, където операндите са релации.
Тя дефинира множество от стандартни операции за изпълнение на заявки върху релационни бази от данни.
Релационната алгебра не е пълна по Тюринг, т.е. не можем да извършваме всичко с нея, но това позволява допълнителни оптимизации от страна на СУБД, които я имплементират.

\subsection{Основни операции}

Казваме, че релациите $R(A_1, \dots, A_n)$ и $S(B_1, \dots, B_m)$ за $n, m \in \mathbb{N}$ са \textbf{\textit{съвместими}} когато:
\begin{itemize}
    \item $m = n$.
    \item $dom(A_i) = dom(B_i), i \in \mathbb{N}, i \in [1, n]$, където $dom$ е функция връщаща домейна, асоцииран с атрибута, който ѝ е подаден като аргумент.
    \item редът на атрибутите в двете релации съвпада.
\end{itemize}
\bigbreak

\textbf{\textit{Обединение (Union)}} на две съвместими релации $R$ и $S$ наричаме релацията $U$, която съдържа всички кортежи на $R$ и $S$.
Бележим $U = R \cup S$.
Тя е бинарна, комутативна и асоциативна операция.
\bigbreak

\textbf{\textit{Разлика (Difference)}} на две съвместими релации $R$ и $S$, където от $R$ изваждаме $S$, наричаме релацията $D$ която съдържа елементите на $R$, които не се съдържат в $S$.
Бележим като $D = R \setminus S$.
Тя е бинарна и некомутативна.
\bigbreak

\textbf{\textit{Проекция (Projection)}} на $n$-мерната релация $R(A_1, \dots, A_n), n \in \mathbb{N}$ наричаме релацията \\
$P(A_{i_1}, \dots, A_{i_k})$, $k \in \mathbb{N}, k \leq n$, където $\{A_{i_1}, \dots, A_{i_k}\} \subseteq \{A_1, \dots, A_n\}$.
Тя се получава чрез отстраняване на всички атрибути на $R$, които не фигурират в конфигурацията на $P$.
Следователно може да се каже, че тя е намаляване в ширина (по брой атрибути).
Като страничен ефект може да настъпи и намаляване в дълбочина (по брой кортежи) при премахване на всички атрибути, отличаващи поне два кортежа.
При релационни СУБД данните обичайно се пазят като мултимножества и такава загуба на информация не може да настъпи.
\bigbreak
Бележим проекция на $R(A_1, \dots, A_n)$ в $P(A_{i_1}, \dots, A_{i_k})$ чрез функцията $\pi_{A_{i_1}, \dots, A_{i_k}}(R)$.
Тази операция е аналог на \textbf{SELECT} в \textbf{SQL}.
\bigbreak

\textbf{\textit{Селекция (Selection)}} върху релация $R(A_1, \dots, A_n), n \in \mathbb{N}$ според предиката $E$ е релацията $(R:E)(A_1, \dots, A_n)$, където $R:E = \{c \in R | E(c) = true\}$.
Бележим с $\sigma_{[predicate]}(R)$.
Тази операция е аналог на \textbf{WHERE} в \textbf{SQL}.
\bigbreak

\textbf{\textit{Преименуване (Renaming)}} на релация $R(A_1, \dots, A_n), n \in \mathbb{N}$ е релация $R'(B_1, \dots, B_n)$, такава че $R = R'$, но не е задължително да е изпълнено $A_i = B_i$ за $i \in \mathbb{N}, i \in [0, 1]$. 
Накратко е операция, която сменя имената на атрибутите в дадена релация без да променя домейните им.
Бележим като $\rho_{[new names]}(R)$.
Тази операция е аналог на \textbf{AS} в \textbf{SQL}.

\textbf{\textit{Декартово произведение (Cartesian product)}} на две релации $R(X)$ и $S(Y)$, където $X$ и $Y$ са множества от атрибути наричаме релацията $(R \times S)(Z)$, където:
\begin{equation}
    \begin{cases}
        Z = X \circ Y\\
        R \times S = \{r \circ s | r \in R, s \in S\}
    \end{cases}
\end{equation}
където $\circ$ е конкатенация.
На прост език съдържа комбинацията на всеки кортеж от първото множество с всеки такъв от второто.


\subsection{Допълнителни операции}

\textbf{\textit{Сечение (Intersection)}} на две съвместими релации $R$ и $S$ наричаме релацията $I$, която съдържа само общите кортежи на $R$ и $S$.
Бележим $I = R \cap S$.
Тя е бинарна, комутативна и асоциативна операция.
\bigbreak

\textbf{\textit{Частно (Division)}} на две релации $N(X)$ и $D(Y)$, където $X$ и $Y$ са множества от атрибути и $Y \subset X$, наричаме релацията $R$, за която е изпълнено, че:
\begin{itemize}
    \item съдържа само атрибутите $X \setminus Y$.
    \item съдържа само кортежите на $N$, които за една комбинация на $X \setminus Y$ са съпоставени спрямо всички кортежи на $D$.
\end{itemize}


Формалната дефиниция e $(N \div D)(X \setminus Y): \, N \div D = \{t | t \in \pi_{X \setminus Y} (N) :\, \forall d \in D \,$ е в сила $\, t \circ d \in N \}$\\
Например:\\

\begin{multicols}{3}
\begin{center}
\begin{tabular}{ |c|c| }
    \hline
    \multicolumn{2}{|c|}{StudentSports} \\
    \hline
    student\_id & sport \\
    \hline
    1 & badminton \\
    2 & cricket \\
    2 & badminton \\
    4 & cricket \\
    \hline
\end{tabular}
\end{center}

\begin{center}
\begin{tabular}{ |c| } 
    \hline
    \multicolumn{1}{|c|}{Sports} \\
    \hline
    sport \\
    \hline
    badminton \\
    cricket \\
    \hline
\end{tabular}
\end{center}

\begin{center}
\begin{tabular}{ |c| } 
    \hline
    \multicolumn{1}{|c|}{StudentSports $\div$ Sports} \\
    \hline
    student\_id \\
    \hline
    2 \\
    \hline
\end{tabular}
\end{center}

\end{multicols}

\textbf{\textit{Съединение (Theta join)}} на две релации $R(X)$ и $S(Y)$, където $X$ и $Y$ са множества от атрибути, по предиката $C$ е релацията $T = \{t \in R \times S | C(t) = true\}$.
На кратко това са всички кортежи от декартовото произведение, които отговарят на някакво условие.
\bigbreak
Бележим с $T = X \underset{C}{\Theta} Y$.

\textbf{\textit{Естествено съединение (Natural join)}} на две релации $R(X)$ и $S(Y)$, където $X$ и $Y$ са множества от атрибути, наричаме релация $N = \{t \in R \times S | \exists tp \in \pi_{X \cap Y}(\{t\}): tp \in \pi_{X \cap Y}(R)$ и $tp \in \pi_{X \cap Y}(S)\}$.
На кратко е частен случай на съединението между две множества, където условието е кортежите да съвпадат по стойности в атрибутите с еднакви имена.
\bigbreak
Бележим с $N = R \bowtie S$. Тази операция е аналог на \textbf{JOIN} в \textbf{SQL}.

\end{document}
