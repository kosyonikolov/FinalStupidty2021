
\documentclass[fleqn,12pt]{article}

\usepackage[margin=15mm]{geometry}
\usepackage[utf8]{inputenc}
\usepackage[bulgarian]{babel}
\usepackage[unicode]{hyperref}
\usepackage{amsfonts}
\usepackage{amssymb}
\usepackage{enumitem, hyperref}
\usepackage{upgreek}
\usepackage{indentfirst}
\usepackage{array}
\usepackage{listings}

\usepackage{amsmath}
\DeclareMathOperator{\cotg}{cotg}
\DeclareMathOperator{\LCR}{LCR}
\DeclareMathOperator{\longer}{longer}

\title{Тема 26\\Полиноми на една променлива. Теорема за деление с остатък. Най-голям
общ делител на полиноми – тъждество на Безу и алгоритъм на Евклид.
Зависимост между корени и коефициенти на полиноми (формули на Виет).}

\author{v0.1}
\date{29 юни 2021}

\begin{document}

\maketitle

\tableofcontents

\section{Предварителни дефиниции}
Нека $R$ е непразно множество, в което са въведени операциите събиране и умножение.
Дефинираме следните свойства за елементи $a,b,c \in R$:
\begin{enumerate}
    \item Асоциативност на събирането: $(a + b) + c = a + (b + c)$
    \item Съществува неутрален елемент на събирането: $\exists e_0 \in M$, такъв че $\forall a \in M$ е изпълнено $a + e_0 = e_0 + a = a$
    \item Съществува противоположен елемент за събирането: $\forall a \in M$ $\exists (-a) \in M$ : $a + (-a) = (-a) + a = e_0$
    \item Комутативност на събирането: $a + b = b + a$
    \item Асоциативност на умножението: $(ab)c = a(bc)$
    \item Дистрибутивни закони на умножението: $a(b + c) = ab + ac$, $(a + b)c = ac + bc$
    \item Съществува неутрален елемент на умножението: $\exists e_1 \in M$ : $\forall a \in M$ е изпълнено $a . e_1 = e_1 . a = a$
    \item Комутативност на умножението: $ab = ba$
    \item Съществува обратен елемент на умножението: $\forall a \neq e_0 \in M$ $\exists a^{-1} \in M$ : $a . a^{-1} = a^{-1} . a = e_1$
\end{enumerate}

\subsection{Абелева група}
Ако са изпълнени свойства 1 до 4, казваме че $(R, +)$ е \textbf{Абелева група}.

\subsection{Пръстен}
Ако са изпълнени свойста 1 до 6, казваме, че $R$ е \textbf{пръстен}.
Ако допълнително е изпълнено и свойство 7, казваме че $R$ е \textbf{пръстен с единица}.
Ако са изпълнени 1 до 6 и 8, казваме че $R$ е \textbf{комутативен пръстен}.
Ако са изпълнени 1 до 8, казваме че $R$ e \textbf{комутативен пръстен с единица}.

\subsection{Подпръстен}
Нека $R$ е пръстен и $R' \subseteq R$. Казваме, че $R'$ е подпръстен на $R$ и 
обозначаваме $R' \leq R$, ако $R'$ е затворен относно събиране, изваждане и умножение.
\textbf{Тривиални} подпръстени на $R$ са $\{e_0\}, R$.

\subsection{Хомоморфизъм на пръстени}
Нека $R$ и $R'$ са пръстени. Казваме, че $\varphi : R \rightarrow R'$ е \textbf{хомоморфизъм на пръстени},
ако за всички $a,c \in R$ е изпълнено:
\begin{itemize}
    \item $\varphi(a + c) = \varphi(a) + \varphi(c)$
    \item $\varphi(ac) = \varphi(a) . \varphi(c)$
\end{itemize}

\subsection{Изоморфизъм на пръстени}
Нека $\varphi : R \rightarrow R'$ е хомоморфизъм на пръстени.
Казваме, че $\varphi$ е \textbf{изоморфизъм на пръстени}, ако $\varphi$ е биекция.
Отбелязваме с $R \cong R'$.

\subsection{Идеал на пръстен}
Нека $R$ е комутативен пръстен. Множество $I \subseteq R$ се нарича
\textbf{(двустранен) идеал} на пръстена, ако за всички $a,b \in I$ е изпълнено:
\begin{itemize}
    \item $a - b \in I \Leftrightarrow a + (-b) \in I$
    \item $r.a = a.r \in I$, $\forall r \in R$
\end{itemize}

Записваме $I \trianglelefteq R$.

\subsection{Главен идеал}
Нека $R$ е комутативен пръстен. \textbf{Главен идеал}, породен от елемента $a \in R$,
дефинираме като множеството $<a> = \{r.a | r \in R\} \trianglelefteq R$.

\subsection{Тяло}
Ако са изпълнени 1 до 7 и 9, казваме че $R$ е \textbf{тяло}.

\subsection{Поле}
Ако са изпълнени всички свойства (1 до 9), казваме че $R$ е \textbf{поле}.

\section{Дефиниции}
\subsection{Полином с коефициенти над поле}
\subsection{Степен на полином}
\subsection{Корени на полином}

\section{Теорема за деление с остатък}

\section{Схема на Хорнер}

\section{Идеали}
\subsection{Дефиниция}
\subsection{Всеки идеал в F[x] е главен идеал}

\section{Принцип за сравняване на коефициенти}

\section{Най-голям общ делител на два полинома}
\subsection{Определение}
\subsection{Tеорема за съществуване на най-голям общ делител на два полинома с коефициенти над поле}
\subsection{Тъждество на Безу}
\subsection{Алгоритъм на Евклид}

\section{Формули на Виет за полином от степен n с коефициенти от поле}


\end{document}
