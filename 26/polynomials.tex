
\documentclass[fleqn,12pt]{article}

\usepackage[margin=15mm]{geometry}
\usepackage[utf8]{inputenc}
\usepackage[bulgarian]{babel}
\usepackage[unicode]{hyperref}
\usepackage{amsfonts}
\usepackage{amssymb}
\usepackage{enumitem, hyperref}
\usepackage{upgreek}
\usepackage{indentfirst}
\usepackage{array}
\usepackage{listings}

\usepackage{amsmath}
\DeclareMathOperator{\cotg}{cotg}
\DeclareMathOperator{\LCS}{LCS}
\DeclareMathOperator{\longer}{longer}

\title{Тема 26\\Полиноми на една променлива. Теорема за деление с остатък. Най-голям
общ делител на полиноми – тъждество на Безу и алгоритъм на Евклид.
Зависимост между корени и коефициенти на полиноми (формули на Виет).}

\author{v0.1}
\date{29 юни 2021}

\begin{document}

\maketitle

\tableofcontents

\section{Дефиниции}
\subsection{Полином с коефициенти над поле}
\subsection{Степен на полином}
\subsection{Корени на полином}

\section{Теорема за деление с остатък}

\section{Схема на Хорнер}

\section{Идеали}
\subsection{Дефиниция}
\subsection{Всеки идеал в F[x] е главен идеал}

\section{Принцип за сравняване на коефициенти}

\section{Най-голям общ делител на два полинома}
\subsection{Определение}
\subsection{Tеорема за съществуване на най-голям общ делител на два полинома с коефициенти над поле}
\subsection{Тъждество на Безу}
\subsection{Алгоритъм на Евклид}

\section{Формули на Виет за полином от степен n с коефициенти от поле}


\end{document}
