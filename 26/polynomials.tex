
\documentclass[fleqn,12pt]{article}

\usepackage[margin=15mm]{geometry}
\usepackage[utf8]{inputenc}
\usepackage[bulgarian]{babel}
\usepackage[unicode]{hyperref}
\usepackage{amsfonts}
\usepackage{amssymb}
\usepackage{enumitem, hyperref}
\usepackage{upgreek}
\usepackage{indentfirst}
\usepackage{array}
\usepackage{listings}

\usepackage{amsmath}
\DeclareMathOperator{\cotg}{cotg}
\DeclareMathOperator{\LCR}{LCR}
\DeclareMathOperator{\longer}{longer}

\title{Тема 26\\Полиноми на една променлива. Теорема за деление с остатък. Най-голям
общ делител на полиноми – тъждество на Безу и алгоритъм на Евклид.
Зависимост между корени и коефициенти на полиноми (формули на Виет).}

\author{v0.1}
\date{29 юни 2021}

\begin{document}

\maketitle

\tableofcontents

\section{Предварителни дефиниции}
Нека $R$ е непразно множество, в което са въведени операциите събиране и умножение.
Дефинираме следните свойства за елементи $a,b,c \in R$:
\begin{enumerate}
    \item \label{prop:add_assoc} Асоциативност на събирането: $(a + b) + c = a + (b + c)$
    \item \label{prop:add_neutral} Съществува неутрален елемент на събирането: $\exists e_0 \in M$, такъв че $\forall a \in M$ е изпълнено $a + e_0 = e_0 + a = a$
    \item \label{prop:add_inverse} Съществува противоположен елемент за събирането: $\forall a \in M$ $\exists (-a) \in M$ : $a + (-a) = (-a) + a = e_0$
    \item \label{prop:add_commute} Комутативност на събирането: $a + b = b + a$
    \item \label{prop:mul_assoc} Асоциативност на умножението: $(ab)c = a(bc)$
    \item \label{prop:mul_distr} Дистрибутивни закони на умножението: $a(b + c) = ab + ac$, $(a + b)c = ac + bc$
    \item \label{prop:mul_neutral} Съществува неутрален елемент на умножението: $\exists e_1 \in M$ : $\forall a \in M$ е изпълнено $a . e_1 = e_1 . a = a$
    \item \label{prop:mul_commute} Комутативност на умножението: $ab = ba$
    \item \label{prop:mul_inverse} Съществува обратен елемент на умножението: $\forall a \neq e_0 \in M$ $\exists a^{-1} \in M$ : $a . a^{-1} = a^{-1} . a = e_1$
\end{enumerate}

\subsection{Абелева група}
Ако са изпълнени свойства 1 до 4, казваме че $(R, +)$ е \textbf{Абелева група}.

\subsection{Пръстен}
Ако са изпълнени свойста 1 до 6, казваме, че $R$ е \textbf{пръстен}.
Ако допълнително е изпълнено и свойство 7, казваме че $R$ е \textbf{пръстен с единица}.
Ако са изпълнени 1 до 6 и 8, казваме че $R$ е \textbf{комутативен пръстен}.
Ако са изпълнени 1 до 8, казваме че $R$ e \textbf{комутативен пръстен с единица}.

\subsection{Подпръстен}
Нека $R$ е пръстен и $R' \subseteq R$. Казваме, че $R'$ е подпръстен на $R$ и 
обозначаваме $R' \leq R$, ако $R'$ е затворен относно събиране, изваждане и умножение.
\textbf{Тривиални} подпръстени на $R$ са $\{e_0\}, R$.

\subsection{Хомоморфизъм на пръстени}
Нека $R$ и $R'$ са пръстени. Казваме, че $\varphi : R \rightarrow R'$ е \textbf{хомоморфизъм на пръстени},
ако за всички $a,c \in R$ е изпълнено:
\begin{itemize}
    \item $\varphi(a + c) = \varphi(a) + \varphi(c)$
    \item $\varphi(ac) = \varphi(a) . \varphi(c)$
\end{itemize}

\subsection{Изоморфизъм на пръстени}
Нека $\varphi : R \rightarrow R'$ е хомоморфизъм на пръстени.
Казваме, че $\varphi$ е \textbf{изоморфизъм на пръстени}, ако $\varphi$ е биекция.
Отбелязваме с $R \cong R'$.

\subsection{Идеал на пръстен}
Нека $R$ е комутативен пръстен. Множество $I \subseteq R$ се нарича
\textbf{(двустранен) идеал} на пръстена, ако за всички $a,b \in I$ е изпълнено:
\begin{itemize}
    \item $a - b \in I \Leftrightarrow a + (-b) \in I$
    \item $r.a = a.r \in I$, $\forall r \in R$
\end{itemize}

Записваме $I \trianglelefteq R$.

\subsection{Главен идеал}
Нека $R$ е комутативен пръстен. \textbf{Главен идеал}, породен от елемента $a \in R$,
дефинираме като множеството $<a> = \{r.a | r \in R\} \trianglelefteq R$.

\subsection{Тяло}
Ако са изпълнени 1 до 7 и 9, казваме че $R$ е \textbf{тяло}.

\subsection{Поле}
Ако са изпълнени всички свойства (1 до 9), казваме че $R$ е \textbf{поле}.

\section{Полином}
\subsection{Дефиниция}
Нека $A$ е комутативен пръстен с единица.
Дефинираме множеството 
\[ A[x] = \{(a_0, a_1, \dots, a_n, 0, 0, \dots | a_i \in A) \] 
с
допълнителното условие, че \textbf{краен} брой елементи на всяка редица са различни от нула.
Въвеждаме операциите събиране и умножение по следния начин за всички $(a_0, \dots, a_n, \dots), (b_0, \dots, b_n, \dots) \in A[x]$:
\[(a_0, \dots, a_n, \dots) + (b_0, \dots, b_n, \dots) = (a_0 + b_0, a_1 + b_1, \dots) \]
\[(a_0, \dots, a_n, \dots) . (b_0, \dots, b_n, \dots) = (c_0, c_1, \dots), \hspace{3mm}
c_k = \sum_{i=0}^{k} a_i . b_{k - i} \]

Елементите на $A[x]$ наричаме \textbf{полиноми}. Така дефинираното $A[x]$ е \textbf{комутативен пръстен с единица}.

\textbf{Доказателство:} Очевидно нулевият полином е $(0, 0, \dots) \Rightarrow$ свойство \ref{prop:add_neutral} е изпълнено.
Полиномът $(1, 0, 0, \dots)$ изпълнява свойство \ref{prop:mul_neutral}. Лесно се вижда, че свойства \ref{prop:add_assoc}, \ref{prop:add_commute}, 
\ref{prop:add_inverse} и \ref{prop:mul_commute} са изпълнени. Свойства \ref{prop:mul_assoc} и \ref{prop:mul_distr} не са трудни за доказване, 
но искат писане на много суми, поради което засега ще ги пропуснем\dots

\subsection{Свойства}
Съществува биекция, изпращаща всяко $a \in A$ в редицата $(a, 0, \dots) \in A[x]$.
Следователно $A \leq A[x]$.

Нека обозначим $x = (0, 1, 0, \dots) \in A[x]$. Тогава лесно се вижда, че
$x^n$ съдържа единствена единица на $n+1$-во място (броейки от 1), или на индекс точно $n$.
Вижда се и че $a.x^n = (0, 0, \dots, a, 0, \dots)$ - отново на същата позиция.
Следователно всеки полином $f = (a_0, a_1, \dots, a_n, \dots)$ можем да представим като сума:
$f = a_0 + a_1 x + a_2 x^2 + \dots + a_n x^n$.
Това ни позволява да дефинираме $A[x]$ по еквивалентен начин:
\[A[x] = \{ f = a_0 + a_1 x + \dots a_n x^n | a_i \in A, n \in \mathbb{N} \}\]

\subsection{Степен на полином}
Нека $f = a_0 + a_1 x + a_2 x^2 + \dots + a_n x^n$ е полином. Степен на полинома $f$ 
наричаме числото $n$, т.е. последният ненулев елемент на редицата. Обозначаваме $\deg f = n$.
Полиномите от вида $(a, 0, \dots), a \neq 0$ имат степен 0. Дефинираме нулевият полином да има степен $-\infty$.

\subsection{Полином с коефициенти над поле}
С $F[x]$ обозначаваме пръстен от полиноми над поле $F$ с пролемнлива $x$.

\subsection{Корени на полином}
TODO

\section{Теорема за деление с остатък}
Нека $f,g \in F[x]$, различни от нулевия полином. 
Тогава съществуват \textbf{единствени} полиноми $q,r \in F[x]$,
които наричаме частно и остатък, такива че
\[ f = g.q + r, \hspace{2mm} \deg r < \deg g \]

\section{Схема на Хорнер}

\section{Всеки идеал в F[x] е главен идеал}

\section{Принцип за сравняване на коефициенти}

\section{Най-голям общ делител на два полинома}
\subsection{Определение}
\subsection{Tеорема за съществуване на най-голям общ делител на два полинома с коефициенти над поле}
\subsection{Тъждество на Безу}
\subsection{Алгоритъм на Евклид}

\section{Формули на Виет за полином от степен n с коефициенти от поле}


\end{document}
