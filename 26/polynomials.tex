
\documentclass[fleqn,12pt]{article}

\usepackage[margin=15mm]{geometry}
\usepackage[utf8]{inputenc}
\usepackage[bulgarian]{babel}
\usepackage[unicode]{hyperref}
\usepackage{amsfonts}
\usepackage{amssymb}
\usepackage{enumitem, hyperref}
\usepackage{upgreek}
\usepackage{indentfirst}
\usepackage{array}
\usepackage{listings}

\usepackage{amsmath}
\DeclareMathOperator{\cotg}{cotg}
\DeclareMathOperator{\LCR}{LCR}
\DeclareMathOperator{\longer}{longer}

\title{Тема 26\\Полиноми на една променлива. Теорема за деление с остатък. Най-голям
общ делител на полиноми – тъждество на Безу и алгоритъм на Евклид.
Зависимост между корени и коефициенти на полиноми (формули на Виет).}

\author{v0.1}
\date{29 юни 2021}

\begin{document}

\maketitle

\tableofcontents

\section{Предварителни дефиниции}
Нека $R$ е непразно множество, в което са въведени операциите събиране и умножение.
Дефинираме следните свойства за елементи $a,b,c \in R$:
\begin{enumerate}
    \item \label{prop:add_assoc} Асоциативност на събирането: $(a + b) + c = a + (b + c)$
    \item \label{prop:add_neutral} Съществува неутрален елемент на събирането: $\exists e_0 \in M$, такъв че $\forall a \in M$ е изпълнено $a + e_0 = e_0 + a = a$
    \item \label{prop:add_inverse} Съществува противоположен елемент за събирането: $\forall a \in M$ $\exists (-a) \in M$ : $a + (-a) = (-a) + a = e_0$
    \item \label{prop:add_commute} Комутативност на събирането: $a + b = b + a$
    \item \label{prop:mul_assoc} Асоциативност на умножението: $(ab)c = a(bc)$
    \item \label{prop:mul_distr} Дистрибутивни закони на умножението: $a(b + c) = ab + ac$, $(a + b)c = ac + bc$
    \item \label{prop:mul_neutral} Съществува неутрален елемент на умножението: $\exists e_1 \in M$ : $\forall a \in M$ е изпълнено $a . e_1 = e_1 . a = a$
    \item \label{prop:mul_commute} Комутативност на умножението: $ab = ba$
    \item \label{prop:mul_inverse} Съществува обратен елемент на умножението: $\forall a \neq e_0 \in M$ $\exists a^{-1} \in M$ : $a . a^{-1} = a^{-1} . a = e_1$
\end{enumerate}

\subsection{Абелева група}
Ако са изпълнени свойства 1 до 4, казваме че $(R, +)$ е \textbf{Абелева група}.

\subsection{Пръстен}
Ако са изпълнени свойста 1 до 6, казваме, че $R$ е \textbf{пръстен}.
Ако допълнително е изпълнено и свойство 7, казваме че $R$ е \textbf{пръстен с единица}.
Ако са изпълнени 1 до 6 и 8, казваме че $R$ е \textbf{комутативен пръстен}.
Ако са изпълнени 1 до 8, казваме че $R$ e \textbf{комутативен пръстен с единица}.

\subsubsection{Област (на цялост)}
Един комутативен пръстен с единица наричаме \textbf{област (на цялост)}, ако в него
няма ненулеви делители на нулата.

\subsection{Подпръстен}
Нека $R$ е пръстен и $R' \subseteq R$. Казваме, че $R'$ е подпръстен на $R$ и 
обозначаваме $R' \leq R$, ако $R'$ е затворен относно събиране, изваждане и умножение.
\textbf{Тривиални} подпръстени на $R$ са $\{e_0\}, R$.

\subsection{Хомоморфизъм на пръстени}
Нека $R$ и $R'$ са пръстени. Казваме, че $\varphi : R \rightarrow R'$ е \textbf{хомоморфизъм на пръстени},
ако за всички $a,c \in R$ е изпълнено:
\begin{itemize}
    \item $\varphi(a + c) = \varphi(a) + \varphi(c)$
    \item $\varphi(ac) = \varphi(a) . \varphi(c)$
\end{itemize}

\subsection{Изоморфизъм на пръстени}
Нека $\varphi : R \rightarrow R'$ е хомоморфизъм на пръстени.
Казваме, че $\varphi$ е \textbf{изоморфизъм на пръстени}, ако $\varphi$ е биекция.
Отбелязваме с $R \cong R'$.

\subsection{Идеал на пръстен}
Нека $R$ е комутативен пръстен. Множество $I \subseteq R$ се нарича
\textbf{(двустранен) идеал} на пръстена, ако за всички $a,b \in I$ е изпълнено:
\begin{itemize}
    \item $a - b \in I \Leftrightarrow a + (-b) \in I$
    \item $r.a = a.r \in I$, $\forall r \in R$
\end{itemize}

Записваме $I \trianglelefteq R$.

\subsection{Главен идеал}
Нека $R$ е комутативен пръстен. \textbf{Главен идеал}, породен от елемента $a \in R$,
дефинираме като множеството $<a> = \{r.a | r \in R\} \trianglelefteq R$.

\subsection{Тяло}
Ако са изпълнени 1 до 7 и 9, казваме че $R$ е \textbf{тяло}.

\subsection{Поле}
Ако са изпълнени всички свойства (1 до 9), казваме че $R$ е \textbf{поле}.

\section{Полином}
\subsection{Дефиниция}
Нека $A$ е комутативен пръстен с единица.
Дефинираме множеството 
\[ A[x] = \{(a_0, a_1, \dots, a_n, 0, 0, \dots | a_i \in A) \] 
с
допълнителното условие, че \textbf{краен} брой елементи на всяка редица са различни от нула.
Въвеждаме операциите събиране и умножение по следния начин за всички $(a_0, \dots, a_n, \dots), (b_0, \dots, b_n, \dots) \in A[x]$:
\[(a_0, \dots, a_n, \dots) + (b_0, \dots, b_n, \dots) = (a_0 + b_0, a_1 + b_1, \dots) \]
\[(a_0, \dots, a_n, \dots) . (b_0, \dots, b_n, \dots) = (c_0, c_1, \dots), \hspace{3mm}
c_k = \sum_{i=0}^{k} a_i . b_{k - i} \]

Елементите на $A[x]$ наричаме \textbf{полиноми}. Така дефинираното $A[x]$ е \textbf{комутативен пръстен с единица}.

\textbf{Доказателство:} Очевидно нулевият полином е $(0, 0, \dots) \Rightarrow$ свойство \ref{prop:add_neutral} е изпълнено.
Полиномът $(1, 0, 0, \dots)$ изпълнява свойство \ref{prop:mul_neutral}. Лесно се вижда, че свойства \ref{prop:add_assoc}, \ref{prop:add_commute}, 
\ref{prop:add_inverse} и \ref{prop:mul_commute} са изпълнени. Свойства \ref{prop:mul_assoc} и \ref{prop:mul_distr} не са трудни за доказване, 
но искат писане на много суми, поради което засега ще ги пропуснем\dots

\subsection{Свойства}
Съществува биекция, изпращаща всяко $a \in A$ в редицата $(a, 0, \dots) \in A[x]$.
Следователно $A \leq A[x]$.

Нека обозначим $x = (0, 1, 0, \dots) \in A[x]$. Тогава лесно се вижда, че
$x^n$ съдържа единствена единица на $n+1$-во място (броейки от 1), или на индекс точно $n$.
Вижда се и че $a.x^n = (0, 0, \dots, a, 0, \dots)$ - отново на същата позиция.
Следователно всеки полином $f = (a_0, a_1, \dots, a_n, \dots)$ можем да представим като сума:
$f = a_0 + a_1 x + a_2 x^2 + \dots + a_n x^n$.
Това ни позволява да дефинираме $A[x]$ по еквивалентен начин:
\[A[x] = \{ f = a_0 + a_1 x + \dots a_n x^n | a_i \in A, n \in \mathbb{N} \}\]

\subsection{Степен на полином}
Нека $f = a_0 + a_1 x + a_2 x^2 + \dots + a_n x^n$ е полином. Степен на полинома $f$ 
наричаме числото $n$, т.е. последният ненулев елемент на редицата. Обозначаваме $\deg f = n$.
Полиномите от вида $(a, 0, \dots), a \neq 0$ имат степен 0. Дефинираме нулевият полином да има степен $-\infty$.

\subsection{Полином с коефициенти над поле}
С $F[x]$ обозначаваме пръстен от полиноми над поле $F$ с променлива $x$.

\subsection{Корени на полином}
Нека $f \neq 0$ е полином от $F[x]$. Нека $\alpha \in K \geq F$ е такова, че 
$f(x) = (x - \alpha) . g(x)$, като $g(x) \in K[x]$. Казваме, че $\alpha$ е
корен на полинома $f(x)$.

\section{Теорема за деление с остатък}
\label{sec:division}
\subsection{Твърдение}
Нека $f,g \in F[x]$, различни от нулевия полином. 
Тогава съществуват \textbf{единствени} полиноми $q,r \in F[x]$,
които наричаме частно и остатък, такива че
\[ f = g.q + r, \hspace{2mm} \deg r < \deg g \]

\subsection{Доказателство (съществуване)}
Ще разгледаме три случая в зависимост от степента на $g$.

\subsubsection{deg f < deg g}
В този случай можем просто да представим $f = 0.g + f$, т.е. $q=0, r = f$.
Това е валидно, понеже $\deg r = \deg f < \deg g$, както се иска по условие.

\subsubsection{deg g = 0}
В случая $g = a, a \in F$. Можем да представим $f = g . \frac{f}{g}$, т.е. $q = \frac{f}{a}, r = 0$.

\subsubsection{deg f $\geq$ deg g}
Нека $n = \deg f$ и $m = \deg g$. Представяме двата полинома по следния начин:
\[ f = a_0 x^n + a_1 x^{n-1} + \dots + a_{n-1} x + a_n \]
\[ g = b_0 x^m + b_1 x^{m-1} + \dots + b_{m-1} x + b_m \]

Коефициентите пред най-високата степен са с индекс 0, а не $n$ / $m$, но това е нарочно, защото ще дефинираме и
$Q = a_0 b_0^{-1} x^{n-m}$. Разглеждаме полинома $f_1 = f - Q . g$. За него е вярно $\deg f_1 < n$, заради начина, по 
който дефинирахме $Q$. Можем да изразим $f$ чрез $f_1$, като пренаредим горното равенство: $f = f_1 + Q . g$.

Ще използваме силна индукция по $n$, за да докажем, че теоремата е вярна.
\textit{Индукционно предположение}: Всички полиноми $f' : \deg f' < n$ могат да се представят
като $f' = g . q' + r'$, $\deg r' < \deg q$. За база ни служат предходните два случая.

Разглеждаме отново $f_1$. Той изпълнява $\deg f_1 < n \Rightarrow f_1 = g . q_1 + r_1$.
Тогава 
\[f = f_1 + Q . g = g . q_1 + r_1 + Q . g = g . (q_1 + Q) + r_1 \]

Полагаме $q = (q_1 + Q)$ и $r = r_1$. От ИП знаем, че $\deg r_1 = \deg r < \deg g$, 
следователно това е валиден избор на $q$ и $r$. Следователно доказахме, че ИП важи и за $n$, с
което тази част на доказателство завършва. 

\subsection{Доказателство (единственост)}
Нека предположим, че съществуват $q_1 \neq q_2$ и $r_1 \neq r_2$, такива че 
$f = g . q_1 + r_1 = g . q_2 . r_2$ и $\deg r_1 < \deg g$, $\deg r_2 < \deg g$. 
\[ g . q_1 + r_1 = g . q_2 . r_2 \Leftrightarrow g(q_1 - q_2) = r_2 - r_1 \]
\[ q_1 \neq q_2 \Rightarrow \deg g(q_1 - q_2) \geq \deg g \]
\[ \deg (r_2 - r_1) \leq \deg r_2 < q \]

Получихме, че степента отляво трябва да е $\geq \deg g$, а отдясно - по-малка
$\Rightarrow$ противоречие. Следователно допускането, че $q_1 \neq q_2$ и $r_1 \neq r_2$ е невярно.
С това доказахме, че има единствени $q,r$, които отговарят на условието в теоремата.


\section{Схема на Хорнер}

\section{Всеки идеал в F[x] е главен идеал}
\textbf{Доказателство: } Нека $I \trianglelefteq F[x]$. Избираме ненулев полином $g \in I$ с \textbf{минимална степен}.
Ще докажем, че $<g> = I$.
Разглеждаме произволен полином $f \in I$. По теорема \ref{sec:division} можем да представим 
\[ f = q.g + r, \deg r < \deg g \Leftrightarrow r = f - q.g \]

Имаме $g \in I, q \in F[x] \Rightarrow q.g \in I$. Понеже и $f \in I \Rightarrow f - q.g \in I \Leftrightarrow r \in I$.
Но сме избрали $g$ да е ненулев полином с минимална степен от $I$ и имаме $\deg r < \deg g \Rightarrow$ единствената възможност
е $r = 0 \Rightarrow f = q.g \Rightarrow f \in <g>$. Теоремата е доказана.

\section{Принцип за сравняване на коефициенти}

\section{Най-голям общ делител на два полинома}
\subsection{Делимост}
Нека $f,g \in F[x], g \neq 0$. Казваме, че $g$ дели $f$ и означаваме $g \mid f$,
ако съществува $h \in F[x] : f = g.h$. Ако $g$ не дели $f$, означаваме $g \nmid f$.

\subsubsection{Свойства}
Нека $a,b \in F, a \neq 0$, $f,g,h \in F[x]$. В сила са следните свойства:
\begin{enumerate}
    \item $a.g \mid b.f$
    \item $g \mid f \Rightarrow a.g \mid b.f$
    \item $f \mid g$ и $g \mid f \Rightarrow f = c.g, c \in F, c\neq 0$
    \item $f \mid g$ и $g \mid h \Rightarrow f \mid h$
    \item $f \mid g_i$ за $i = 1 \dots k \Rightarrow f \mid (t_1 g_1 + \dots + t_k g_k), t_i \in F[x]$
    \item $f \mid (g_1 + g_2)$ и $f \mid g_1 \Rightarrow f \mid g_2$ 
\end{enumerate} 

\subsection{НОД на два полинома}
Нека $f,g \in F[x], g \neq 0$. Най-голям общ делител на $f$ и $g$ наричаме 
полином $d = (f,g), d \in F[x]$, такъв че:
\begin{enumerate}
    \item $d \mid f$ и $d \mid g$
    \item $d_1 \mid f$ и $d_1 \mid g \Rightarrow d_1 \mid d$
\end{enumerate}

С точност до ненулева констант $d=(f,g)$ е еднозначно определен от $f$ и $g$.

\subsection{Tеорема за съществуване на най-голям общ делител на два полинома с коефициенти над поле}
Нека $f,g \in F[x]$. Тогава съществува $d=(f,g) \in F[x]$.

\textbf{Доказателство: } Първо разглеждаме случая $f = g = 0$. Тогава лесно се вижда, че нулевият полином е НОД.

Сега нека допуснем, че поне единия е ненулев. Дефинираме множеството $S$:
\[ S = \{ M.f + N.g | M,N \in F[x] \} \]

При $M(x) = 1$ и $N(x) = 0$, получаваме, че $f \in S$. Аналогично $g \in S$. 
Тогава $S$ съдържа ненулеви полиноми. Избираме такъв от тях, който има минимална степен
 - нека той бъде $d$. Него можем да представим като
\[ d = M_0.f + N_0.g \]

Ще докажем, че така намереният $d$ е най-голям общ делител.

\subsubsection{Доказателство, че е делител}
Да допуснем, че $d \nmid f$. Тогава $f = d.q + r, \deg r < \deg g, r \neq 0$.
\[ f = d.q + r \Leftrightarrow r = f - d.q = 
f - (M_0 . f + N_0 . g)q = 
(1 - M_0 . q) . f - N_0 . q . g \]

Този израз можем да запишем като $r = M_1 . f + N_1 . g \Rightarrow r \in S$.
Но от допускането имаме $\deg r < \deg d$, което е в противоречие с избора на $d$.
Следователно допускането, че $d \nmid f$ е грешно $\Rightarrow d \mid f$.
Аналогично се доказва, че $d \mid g$.

\subsubsection{Доказателство, че е най-голям делител}
Нека $d_1 \mid f, d_1 \mid g$. Нашетo $d$ има вида $M_0.f + N_0.g$, следователно $d_1 \mid d$, с което
доказахме, че $d$ съществува и е НОД на $f,g$, т.е. $d = (f,g)$.

\subsection{Тъждество на Безу}
\subsection{Алгоритъм на Евклид}

\section{Формули на Виет за полином от степен n с коефициенти от поле}
Нека $f \in F[x]$, $f = a_0 x^n + a_1 x^{n-1} + \dots + a_{n-1} x + a_n$
и $\alpha_1, \alpha_2, \dots, \alpha_n$ са корените на $f(x)$. Дефинираме $I = {1, 2, \dots, n}$. 
Тогава в сила са следните формули за $k=1\dots n$:
\[ \mathfrak{S}_k = \sum_{\{i_1, i_2, \dots, i_k \} \subseteq I} \alpha_{i_1} \alpha_{i_2} \dots \alpha_{i_k} = (-1)^k \frac{a_0}{a_k} \]

\end{document}
