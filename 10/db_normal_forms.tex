
\documentclass[fleqn,12pt]{article}

\usepackage[margin=15mm]{geometry}
\usepackage[utf8]{inputenc}
\usepackage[bulgarian]{babel}
\usepackage[unicode]{hyperref}
\usepackage{amsfonts}
\usepackage{amssymb}
\usepackage{enumitem, hyperref}
\usepackage{upgreek}
\usepackage{indentfirst}

\usepackage{amsmath}
\DeclareMathOperator{\cotg}{cotg}
\DeclareMathOperator{\LCS}{LCS}
\DeclareMathOperator{\longer}{longer}

\title{Бази от данни. Нормални форми.}

\author{v0.1}
\date{24 юни 2021}

\begin{document}

\maketitle
\tableofcontents
\pagebreak


\section{Нормални форми}
\subsection{Проектиране схемите на релационните бази от данни}

\subsection{Функционални зависимости}
\subsection{Ключове}
\subsection{Ограничения}
\subsection{Аксиоми на Армстронг}
\subsection{Аномалии}

\subsection{Първа, втора, трета нормална форма, нормална форма на Бойс-Код}
\subsection{Многозначни зависимости}

\subsection{Аксиоми на функционалните и многозначните зависимости}

\subsection{Съединение без загуба}

\subsection{Четвърта нормална форма}

\subsection{Линкове към задачи}

\end{document}
