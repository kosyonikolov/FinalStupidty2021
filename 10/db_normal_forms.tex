
\documentclass[fleqn,12pt]{article}

\usepackage[margin=15mm]{geometry}
\usepackage[utf8]{inputenc}
\usepackage[bulgarian]{babel}
\usepackage[unicode]{hyperref}
\usepackage{amsfonts}
\usepackage{amssymb}
\usepackage{enumitem, hyperref}
\usepackage{upgreek}
\usepackage{indentfirst}

\usepackage{amsmath}
\DeclareMathOperator{\cotg}{cotg}
\DeclareMathOperator{\LCS}{LCS}
\DeclareMathOperator{\longer}{longer}

\title{Бази от данни. Нормални форми.}

\author{v0.1}
\date{24 юни 2021}

\begin{document}

\maketitle
\tableofcontents
\pagebreak


\section{Нормални форми}
\subsection{Проектиране схемите на релационните бази от данни}

Транслацията на даннов модел (напр. \textbf{същност-връзка}) на база от данни до съвкупност от релации, над чиито атрибути са дефинирани зависимости, се нарича проектиране на схема на релационна база от данни.
Важни цели на този процес са да запази цялостта на данните, както и да нормализира получените релации.
\bigbreak
Нормализацията на релационна база от данни е процесът на реструктуриране на релациите и зависимостите на техните атрибути в нормални форми, които целят \textbf{минимизация на дублирането на данните}, както и запазването на тяхната \textbf{цялост}.
Възможни начини за извършването ѝ са синтез, т.е. създаване на нов релационен дизайн, и декомпозиция на вече съществуващ такъв (по-често използван).

\subsection{Функционални зависимости}

\textbf{\textit{Деф}} - Нека $R(A_1, \dots, A_n), n \in \mathbb{N}$ е релация.
Функционална зависимост (ФЗ) за релацията $R$ наричаме твърдение от вида $A_i \rightarrow A_j$, където $A_i, A_j$ са атрибути на $R$, при което $\forall t, u \in R$ е изпълнено, че ако $t$ и $u$ съвпадат по атрибута $A_i$ на $R$, то те съвпадат и по атрибута $A_j$ на $R$.
Четем $A_i$ еднозначно определя $A_j$.
\bigbreak
\textbf{\textit{Правило за разделяне}} - Можем да разделяме атрибутите от дясната страна на ФЗ, така че само един атрибут да се появява от дясно за всяка ФЗ.
Например ФЗ $A_1, A_2 \rightarrow B_1, B_2$ може да се раздели на $A_1, A_2 \rightarrow B_1$ и $A_1, A_2 \rightarrow B_2$.
\bigbreak
\textbf{\textit{Правило за комбиниране}} - Можем да комбинираме множество от ФЗ с едни и същи леви страни до една ФЗ с комбинирани десни страни на първоначалните ФЗ.
Например ФЗ $A_1, A_2 \rightarrow B_1$ и $A_1, A_2 \rightarrow B_2$ можем да комбинираме в $A_1, A_2 \rightarrow B_1$ и $A_1, A_2 \rightarrow B_1, B_2$.
\bigbreak
\textbf{\textit{Заб}} - Правилата за разделяне и комбиниране са приложими само за десните страни на ФЗ.
\bigbreak
\textbf{\textit{Деф}} - Нека $A_1, \dots, A_n \rightarrow B_1, \dots B_m, n, m \in \mathbb{N}$ е ФЗ.
Тогава казваме, че тя е \textbf{тривиална ФЗ} ако $\{B_1, \dots, B_m\} \subseteq \{A_1, \dots, A_n\}$.

\subsection{Ключове}

\textbf{\textit{Деф}} - Множество от един или повече атрибути $\{A_1, \dots, A_n\}$ на дадена релация $R$ наричаме ключ на $R$ ако:
\begin{enumerate}
    \item $\{A_1, \dots, A_n\}$ функционално определя всички останали атрибути на $R$.
    \item $\{A_1, \dots, A_n\}$ е минимално, т.е. не съществува подмножество на $\{A_1, \dots, A_n\}$, за което да е изпълнено условие 1.
\end{enumerate}
\bigbreak
\textit{\textbf{Пример} - Нека $R(A_1, A_2, B_1, B_2)$ е релация и $A_1, A_2 \rightarrow B_1$ и $A_1, A_2 \rightarrow B_2$ са ФЗ.
Тогава имаме, че $A_1, A_2 \rightarrow A_1, A_2, B_1, B_2$ и следователно $\{A_1, A_2\}$ е ключ на $R$.}
\bigbreak
\textbf{\textit{Заб}} - За дадена релация може да съществуват повече от 1 ключа.
Те се наричат \textbf{кандидат-ключове}.
За всяка релация се избира един ключ, който се нарича \textbf{първичен ключ}.
\bigbreak
\textbf{\textit{Деф}} - Нека $R$ е релация и $K$ е ключ на $R$. Множество от атрибути $A: K \subseteq A$ се нарича \textbf{суперключ} на $R$.
\bigbreak
От множеството от ФЗ, които са в сила в релация $R$, можем да определим кандидат ключовете за дадена релация.
Това се прави чрез намирането на покритието на всички атрибути и комбинация от атрибути за релацията $R$.


\subsection{Ограничения}
\subsection{Аксиоми на Армстронг}
\subsection{Аномалии}

\subsection{Първа, втора, трета нормална форма, нормална форма на Бойс-Код}
\subsection{Многозначни зависимости}

\subsection{Аксиоми на функционалните и многозначните зависимости}

\subsection{Съединение без загуба}

\subsection{Четвърта нормална форма}

\subsection{Линкове към задачи}

\end{document}
