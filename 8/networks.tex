
\documentclass[fleqn,12pt]{article}

\usepackage[margin=15mm]{geometry}
\usepackage[utf8]{inputenc}
\usepackage[bulgarian]{babel}
\usepackage[unicode]{hyperref}
\usepackage{amsfonts}
\usepackage{amssymb}
\usepackage{enumitem, hyperref}
\usepackage{upgreek}

\usepackage{amsmath}
\DeclareMathOperator{\cotg}{cotg}
\DeclareMathOperator{\LCS}{LCS}
\DeclareMathOperator{\longer}{longer}
\renewcommand{\arraystretch}{1.3}         % because math expressions

\title{Компютърни мрежи и протоколи – OSI модел. Протоколи IPv4,  IPv6, TCP, HTTP.}
\author{v1.0}
\date{8 юне 2021}

\begin{document}

\maketitle

\tableofcontents
\pagebreak

\begin{flushleft}

\section{Модели на организация на мрежовите протоколи}
\subsection{OSI модел - най-обща характеристика на нивата}
\textbf{OSI (Open Systems Interconnection)} моделът е първият опит за изграждане теоретичен модел, описващ принципите на комуникация и устройство на мрежовите протоколи.
Той е въведен от организацията \textbf{ISO}. Основна градивна единица са слоевете, като всеки слой предоставя интерфейс и услуги към слоя над него. \textf{OSI} се състои от следните 7 слоя:
\begin{enumerate}
    \item \textbf{Физическия слой} е най-ниският слой от \textbf{OSI} модела. Протоколите от този слой приемат и предават битове от едно устройство в друго, като така те отговарят за
    физическата връзка между върховете в дадена топология. Функциите на физическия слой включват:
    \begin{enumerate}
        \item \textit{синхронизация на битове} - осигурява се чрез общ clock, контролиращ изпраща и получателя.
        \item \textit{честота на изпращане на битове (bit rate control)}
        \item \textit{определяне на мрежовата топология}, т.е. дали е bus, star или mesh
        \item \textit{режим на предаване на битовете} - дали комуникацията е simplex, half-duplex или full-duplex.
        \item \textit{начин на установяване/прекъсване на връзката между устройствата}
    \end{enumerate}
    Примери за устройства от физическия слой са хъбовете, модемите и коаксиалните кабели.
    \item \textbf{Каналният слой (Data Link Layer или DLL)} се грижи за безгрешното предаване на данни от едно устройство до друго утилизирайки физическия слой.
    Каналният слой може да бъде разделен на два подслоя \textbf{Logical Link Control (LLC)} и \textbf{Media Access Control (MAC)}.\\
    Отговорност на \textbf{DLL} е да предава пакетите до правилните хостове използвайки техните \textbf{MAC} адреси, които се намират използвайки \textbf{ARP (Address Resolution Protocol)}.\\
    Освен това когато пакет от мрежовия слой пристигне той бива допълнително разбит на кадри (frames) в зависимост от размера на кадрите дефиниран от \textbf{NIC (Network Interface Card)}.\\
    Функциите на каналния слой включват:
    \begin{enumerate}
        \item \textit{разбиване на кадри} чрез поставяне на специални битови последователности в началото и края на всеки кадър.
        \item \textit{физическо адресиране} чрез добавяне на \textbf{MAC} адресите на получателя и/или изпращача в хедърите на всеки кадър.
        \item \textit{контрол над грешките} чрез откриване и предаване отново на счупени и/или загубени кадри.
        \item \textit{контрол над потока от данни} - честотата на получаване на устройствата може да бъде различна и заради това се налага координация над количеството от данни, което 
        може да се предава за даден интервал от време.
        \item \textit{контрол на достъпа} - \textbf{MAC} подслоя се използва за да се определи кой има контрол над даден комуникационен канал в някой момент когато той се използва между много устройства.
    \end{enumerate}
    \textbf{DDL} обичайно се предоставя от \textbf{NIC} и някои драйвери на устройства. Имплементира се от \textbf{Ethernet}, суичовете и бриджовете.
    \item \textbf{Мрежовият слой} отговаря за предаването на данни между хостове от различни мрежи, потенциално различаващи се по физическите и каналните си слоеве.
    Като част от този процес е възможно данните да бъдат фрагментирани под формата на пакети \textbf{(PDU (Protocol Data Unit))}.
    Функциите на мрежовия слой включват:
    \begin{enumerate}
        \item \textit{Маршрутизация} - намиране на най-кратък път от изпращач до получател.
        \item \textit{Логическа адресация} - начин за идентификация на всяко устройство в мрежата. При протокола \textbf{IP} адресите на изпращача и получателя биват поставени в хедъра на пакета.
    \end{enumerate}
    Примерни протоколи са \textbf{IP}, \textbf{ICMP} и \textbf{IPSec}.
    \item \textbf{Транспортният слой} отговаря за доставката на цяло съобщение от край-до-край.
    Това се получава като протоколите от този слой разбиват съобщенията получени от сесийния слой на по-малки наречени сегменти и ги препращат към мрежовия слой,
    осигурявайки, че всички части са пристигнали цели и в правилна наредба при получателя. Ако сегмент не е стигнал до дестинацията си, то той бива изпратен отново.
    Казваме, че съобщенията биват доставени от край-до-край защото транспортния слой на изпращача комуникира директно с този на получателя.
    Функциите на транспортния слоя включват:
    \begin{enumerate}
        \item \textit{Сегментация и повторно сглобяване} - транспортния слой на изпращача получава съобщението от сесийния слой, разбива го на сегменти, поставяйки в хедъра на всеки метаданни за реда на реасемблиране и ги подава на мрежовия слой.
        Транспортния слой на получателя се грижи за реасемблирането на пакетите на база техните хедъри.
        \item \textit{Адресация на услуги (Service Point Addressing)} - за да бъде доставено съобщението до правилния процес върху дестинацията, транспортният слой добавя порт на дестинацията в хедърите на всеки сегмент.
    \end{enumerate}
    Транспортният слой се имплементира като част от ОС, правеща системни извиквания към процесите. Примери са \textbf{TCP} и \textbf{UDP}.
    \item \textbf{Сесийният слой} предоставя:
    \begin{enumerate}
        \item \textit{Установяване, поддръжка и терминиране на сесии}
        \item \textit{Синхронизация} - процесите могат да добавят контролни точки в данните чрез периодичното им запазване.
        Така след прекъсвания на предаването и/или грешки се избягва загуба или повреда на данните.
        \item \textit{Диалогов контролер} - процесите могат да комуникират в half-duplex или full-duplex режим, като той следи чий ред е да изпраща данни.
        \item \textit{Сигурност (напр. автентикация)}
    \end{enumerate}
    Обичайно се имплементира чрез сокети.
    \item \textbf{Презентационният слой} извлича и манипулира данните от приложния слой за да станат годни за пренос по мрежата. Той предоставя:
    \begin{enumerate}
        \item \textit{Транслация} - Конвертиране на данновия формат. Например от \textbf{ASCII} към \textbf{EBCDIC}.
        \item \textit{Криптиране/Декриптиране}
        \item \textit{Компресия}
    \end{enumerate}
    Примерни протоколи от презентационния слой са \textbf{TLS}, \textbf{SSL}, \textbf{IMAP} и \textbf{FTP}.
    \item \textbf{Приложният слой} представлява съвкупност от протоколи, които се имплементират директно от процесите. Примери са \textbf{HTTP} и \textbf{DNS}.
\end{enumerate}

\subsection{TCP/IP модел}
\subsection{Съпоставка между OSI и TCP/IP моделите}
\section{ICMP}
\subsection{IPv4 адресация – класова и безкласова}
\subsection{Основни характеристики на протокол IPv6}

\section{TCP}
\subsection{TCP – процедура на трикратно договаряне}
\subsection{Хипертекстов протокол HTTP}

\end{flushleft}
\end{document}
