\documentclass[fleqn,12pt]{article}

\usepackage[margin=15mm]{geometry}
\usepackage[utf8]{inputenc}
\usepackage[bulgarian]{babel}
\usepackage[unicode]{hyperref}
\usepackage{amsfonts}
\usepackage{amssymb}
\usepackage{enumitem, hyperref}
\usepackage{upgreek}

\usepackage[dvipsnames]{xcolor}

\title{Процедурно програмиране - основни информационни и алгоритмични структури (на базата на C++)}
\author{v1.0}
\date{19 юни 2021}

\begin{document}

\maketitle

\tableofcontents
\pagebreak

\section{Скаларни типове от данни}
\subsection{Логически тип}
\subsection{Числени типове цял и реален}

\section{Основни структури от данни}
\subsection{Съставни типове от данни}
\subsection{Структура от данни масив}
\subsection{Тип масив}

\section{Тип указател}
\subsection{Дефиниране}
\subsection{Основни операции}
\subsection{Указателна аритметика}
\subsection{Указатели и едномерни масиви}
\subsection{Указатели и двумерни масиви}
\subsection{Указатели и низове}

\section{Функции}
\subsection{Дефиниране на функция}
\subsection{Обръщение към функция}
\subsection{Предаване на параметрите по стойност}
\subsection{Предаване на параметрите чрез указател}
\subsection{Предаване на параметрите чрез псевдоним}
\subsection{Масиви като формални параметри}

\end{document}
