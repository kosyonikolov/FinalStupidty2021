
\documentclass[fleqn,12pt]{article}

\usepackage[margin=15mm]{geometry}
\usepackage[utf8]{inputenc}
\usepackage[bulgarian]{babel}
\usepackage[unicode]{hyperref}
\usepackage{amsfonts}
\usepackage{amssymb}
\usepackage{enumitem, hyperref}
\usepackage{upgreek}
\usepackage{indentfirst}
\usepackage{graphicx}

\usepackage{amsmath}

\graphicspath{ {./img/} }

\title{Тема 21 \\Планиране на проекта – същност и основни елементи, обхват на проекта, времеви и финансови ресурси. Дейности по управление и контрол, методи и средства за създаване на план-график на проекта.}

\author{v0.1}
\date{30 юни 2021}

\begin{document}

\maketitle
\tableofcontents
\pagebreak

\section{Планиране на проекта}

\subsection{Същност и основни елементи на плана на проекта}

\textit{\textbf{Проектът}} представлява ограничено от времето начинание, предназначено да създаде някакъв конкретен резултат, продукт или услуга.
\bigbreak
Основни характеристики са:
\begin{itemize}
    \item \textbf{временност (temporality)} - има ограничена продължителност (начало и край);
    \item \textbf{уникалност на резултата (deliverables)} - създава се нов продукт, знание или документ;
    \item \textbf{прогресивно развитие (progressive elaboration)} - постепенно развитие на итерации;
\end{itemize}

\textit{\textbf{Планът}} определя как проектът се изпълнява, наблюдава, контролира и приключва.
\bigbreak

Той включва всички дейности, които са необходими за дефиниране, интегриране и координиране на включените подпланове за управление на (SSCQ - SCRP (Subzero vs Scorpion analogy)):
\begin{itemize}
    \item \textbf{обхвата и мисията на проекта (Scope)} - изясняват се изискванията;
    \item \textbf{графика (Schedule)} - дефинират се контролни точки (milestones) и се разработва график (schedule) спрямо тях;
    \item \textbf{разходите (Cost)} - естимира се колко време и пари ще заминат;
    \item \textbf{качеството (Quality)} - дефинира се стратегия по качество;
    \item \textbf{подобряване на процесите};
    \item \textbf{човешките ресурси (Staffing)} - колко и какви хора ще трябват;
    \item \textbf{комуникациите (Communication)};
    \item \textbf{риска (Risk)}; 
    \item \textbf{поръчки и доставки (Procurement)};
\end{itemize}

\section{Обхват на проекта}

\textit{\textbf{Обхватът}} е определен от стратегическите цели на проекта като има две вариации на дефиницията на обхват:
\begin{itemize}
    \item \textit{\textbf{Обхват на продукт}} - характеристиките и функциите, които характеризират продукт, услуга или резултат;
    \item \textit{\textbf{Обхват на проекта}} - работата, която трябва да бъде извършена, за да се достави продукт, услуга или резултат с посочените характеристики и функции;
\end{itemize}

% може до добавя дейностите
\bigbreak

Основните процеси от управлението на обхвата са:
\begin{enumerate}
    \item \textbf{Планиране на обхвата и мисията} - подготвяне на план за управление на обхвата и мисията на проекта, който документира как те се определят, верифицират и контролират;  %процес на събиране и документиране на нуждите на ЗЛ за постигане на целите на проекта;
    \item \textbf{Определяне на обхвата и мисията} - процес на разработване на подробно описание на обхвата и мисията на проекта като основа за бъдещи решения;
    \item \textbf{Създаване на структура по задачи (WBS)} - процес на разделяне на резултатите от проекта и работата по проекти на по-малки и лесноуправлявеми компоненти;
    \item \textbf{Верифициране на обхвата и мисията} - процес на формално приемане на завършените резултати от проекта;
    \item \textbf{Контрол на обхвата и мисията} - процес на наблюдение на състоянието на проекта и обхвата и мисията на продукта, както и управление на промените в обхвата и мисията;
\end{enumerate}

\subsection{Планиране на обхвата на проекта}
\subsubsection{Събиране на изисквания}

Планирането на обхвата на проекта се състои от фазата на събиране на изискванията.
\textbf{\textit{Събиране на изискванията}} дефинираме като процес на определяне и документиране на нуждите на ЗЛ за постигане на целите на проекта.
Без тях не могат да започнат другите процеси от управлението на обхвата.
Изискванията могат да се класифицират като:
\begin{itemize}
    \item \textbf{Изисквания към проекта} - включват бизнес изисквания, изисквания за доставка на проекта, изисквания за управление на проекти и т.н.;
    \item \textbf{Изисквания към продукта} - включват функционалните и качествените изисквания към продукта;
\end{itemize}

\textbf{Входът на събиране на изискванията} включва заданието на проекта и регистър на заинтересованите страни (ЗС).
\bigbreak

Съществуват следните техники за събиране на изискванията:
\begin{itemize}
    \item \textbf{Интервюта} - подход за откриване на информация от ЗЛ чрез директен разговор с тях;
    \item \textbf{Фокус групи} - събират предваритално определени ЗС и експерти в областта, за да се научи за техните очаквания и нагласи относно предложения продукт, услуга или резултат;
    \item \textbf{Ръководни семинари (Facilitated workshops)} - фокусирани сесии, които обединяват ключови междуфункционално ЗС, за да определят продуктовите изисквания. Помага за постигане на консенсус;
    \item \textbf{Групови креативни техники} като мозъчна атака, делфи техника и картографиране на идеи;
    \item \textbf{Техники за групово вземане на решения} - процес на оценка на множество алтернативи с очакван резултат под формата на разрешаване на бъдещи действия.
    Такива са например \textbf{консенсус} (всички на едно мнение), \textbf{мнозинство}, \textbf{плурализъм} (както при изборите за парламент) и \textbf{диктатура};
    \item \textbf{Въпросници и анкети} - писмени множества от въпроси, предназначени за бързо натрупване на статистика (по анонимен начин);
    \item \textbf{(Етнографски) наблюдения} - осигуряват директен начин за наблюдение на хора в тяхната среда и това как те изпълняват техните ежедневни задачи;
    \item \textbf{Прототипи} - метод за получаване на ранна обратна връзка върху изискванията чрез предоставяне на работещ модел на очаквания продукт, преди действителното му изграждане;
\end{itemize}

\textbf{Изходът на събирането на изискванията} се състои от документация на изискванията, план за управление на изискванията и матрица за проследяване на изискванията.

\subsubsection{Дефиниране на обхвата}

\textbf{\textit{Дефиниране на обхвата}} на проекта е процес на разработване на подробно описание на проекта и продукта.
Детайлността се изразява в анализ на съществуващите рискове, предположения и ограничения, както и добавяне на нови такива.
\bigbreak
\textbf{Входът на дефиниране на обхвата} е заданието на проекта, документацията с изискванията и активите на организационния процес (напр. полици и опит).
\bigbreak

Съществуват следните техники за дефиниране на обхвата:
\begin{itemize}
    \item \textbf{Експертна преценка} - използва се за анализ на информацията, необходима за разработване на описанието на обхвата на проекта;
    \item \textbf{Анализ на продукта} - за проекти, където се доставя продукт.
    Следва се конкретна методика за превръщане на описанията в проудкт.
    Анализът включва техники като product breakdown, системен анализ, value engineering  и value analysis;
    \item \textbf{Идентифициране на алтернативи} - техника, използвана за генериране на различни подходи за изпълнение на работата по проекта;
    \item \textbf{Ръководни семинари (Facilitated workshops)};
\end{itemize}

\textbf{Изходът от дефинирането на обхвата} е (DECAA):
\begin{itemize}
    \item \textbf{Описание на обхвата на проекта (Description)} - включва подробно описание на резултатите от проекта (project deliverables) и критериите за приемане продукт;
    \item \textbf{Изключения на проекта (Exclusions)} - описание на това, което не е част от обхвата;
    \item \textbf{Ограничения на проекта (Constraints)} - например бюджет, крайни дати, договори и други;
    \item \textbf{Предположения за проекта (Assumptions)} - изброява проектните предположения, свързани с обхвата и въздействието им ако се окажат неверни;
    \item \textbf{Актулизация на обхвата на проекта (Actualisation)} - може да се модифицират други документи като регистъра на ЗС, документацията на изискванията и матрицата им;
\end{itemize}

\subsection{Определяне на структура на работа (WBS)}

\textbf{Йерархичната стурктура на работата (Work breakdown structure (WBS))} е процес на разделяне на резултатите от проекта (deliverables) и работата по проекта на по-малки, по-лесно управляеми проекти.
WBS декомпозира структурата на работата по рекурсивен начин, като всяко ниво по-навътре представя по-детайлна дефиниция на проектната работа от родителското си.
Планираната работа се съдържа в WBS компонентите от най-ниско ниво, наречени \textbf{работни пакети}.
\bigbreak
\textbf{Входът на WBS} е описание на обхвата на проекта, документацията с изисквания и активите на организационния процес.
\bigbreak

Стандартният начин за създаване на \textbf{WBS} е:
\begin{enumerate}
    \item избираме резултат от проекта (project deliverable), който е твърде абстрактен за да бъде наречен работен пакет;
    \item разбиваме го на няколко подрезултата;
    \item всеки подрезултат документираме по-подробно от родителския си;
    \item проверяваме, дали всички подрезултати са достатъчно подробни за да се определят като работни пакети. Ако не са продължаваме от 1;
\end{enumerate}

\textbf{Изходът от WBS} е огромен и досаден за помнене, част от който е описание на работата, списък на контролните точки (milestones), график от дейности, необходими ресурси и отговорна организация.
\bigbreak

Един работен пакет (аналог на jira/github issue) се състои от:
\begin{itemize}
    \item описание на извършваните дейности;
    \item други работни пакети, от които зависи;
    \item специфични резултати като acceptance criteria;
    \item от информация за изпълнителя си (напр. кой екип);
    \item оценка за необходимите ресурси;
    \item контролни точки и до кога трябва да е изпълнен;
\end{itemize}
\bigbreak

\subsection{План на контролните точки}

\textbf{Контролните точки} са инструмент за управление, който маркира началото и края на главна фаза от работата върху проект.
Те:
\begin{itemize}
    \item посочват логическата последователност на развитие на проекта;
    \item посочват как междинните резултати водят до крайната цел;
    \item показват КАКВО трябва да бъде изпълнено за да е постигната конкретна контролна точка;
    \item служат за комуникация с външната среда;
    \item формират стабилна работна схема;
\end{itemize}

Рецепта за планиране на контролни точки е:
\begin{enumerate}
    \item Постига се общо съгласие за крайната цел.
    \item Генерират се варианти на контролни точки. Препоръчва се да се ползва brainstorming.
    \item Обсъждат се алтернативните варианти, в резултат на което може да се променят някои от тях.
    \item Генерират се предложения за пътища към резултатите.
    \item Изчертават се логическите зависимости, започвайки от крайната цел. Може да се наложи пренареждане/реструктуриране на контролните точки.
    \item Построява се окончателната схема на плана.
\end{enumerate}

\section{Дейности по управление и контрол}

\subsection{Планиране на време за изпълнение на задачите}
\subsection{Планиране на бюджет и необходими ресурси}
\subsection{Дейности по управление и контрол}

\section{Създаване на план}

\subsection{График на проекта}
\subsection{Методи и средства за създаване на план-график}
\subsection{Метод на критичния път}
\subsection{Метод PERT}
\subsection{GANTT диаграми}


\end{document}
