
\documentclass[fleqn,12pt]{article}

\usepackage[margin=15mm]{geometry}
\usepackage[utf8]{inputenc}
\usepackage[bulgarian]{babel}
\usepackage[unicode]{hyperref}
\usepackage{amsfonts}
\usepackage{amssymb}
\usepackage{enumitem, hyperref}
\usepackage{upgreek}
\usepackage{indentfirst}
\usepackage{graphicx}

\usepackage{amsmath}

\graphicspath{ {./img/} }

\title{Тема 21 \\Планиране на проекта – същност и основни елементи, обхват на проекта, времеви и финансови ресурси. Дейности по управление и контрол, методи и средства за създаване на план-график на проекта.}

\author{v0.1}
\date{30 юни 2021}

\begin{document}

\maketitle
\tableofcontents
\pagebreak

\section{Планиране на проекта}

\subsection{Същност}
\subsection{Основни елементи на плана на проекта}

\section{Обхват на проекта}

\subsection{Планиране на обхвата на проекта}
\subsection{Определяне на структура на работа (WBS)}
\subsection{План на контролните точки}

\section{Дейности по управление и контрол}

\subsection{Планиране на време за изпълнение на задачите}
\subsection{Планиране на бюджет и необходими ресурси}
\subsection{Дейности по управление и контрол}

\section{Създаване на план}

\subsection{График на проекта}
\subsection{Методи и средства за създаване на план-график}
\subsection{Метод на критичния път}
\subsection{Метод PERT}
\subsection{GANTT диаграми}


\end{document}
