\documentclass[fleqn,12pt]{article}
\usepackage[margin=15mm]{geometry}
\usepackage[T2A]{fontenc}
\usepackage[utf8]{inputenc}
\usepackage[bulgarian]{babel}
\usepackage{indentfirst}

\makeatletter
\newcommand\subsubsubsection{\@startsection{paragraph}{4}{\z@}{-2.5ex\@plus -1ex \@minus -.25ex}{1.25ex \@plus .25ex}{\normalfont\normalsize\bfseries}}
\newcommand\subsubsubsubsection{\@startsection{subparagraph}{5}{\z@}{-2.5ex\@plus -1ex \@minus -.25ex}{1.25ex \@plus .25ex}{\normalfont\normalsize\bfseries}}
\makeatother

\title{Софтуерна архитектура. Проектиране и документиране на софтуерни
архитектури.}
\date{June 2021}

\begin{document}

\maketitle 

\section{Дефиниция на софтуерна архитектура. Структури и изгледи  на архитектурата.}
\subsection{Дефиниция на софтуерна архитектура}
Архитектура на дадена софтуерна система е съвкупност от структури, показващи различните софтуерни елементи на системата, външно видимите им свойства и връзките между тях.

Софтуерната архитектура е абстракция, която скрива
детайлите, от които взаимодействието между елементите не зависи. Детайли като алгоритми, представяне на данни, реализация, и т.н. не са
предмет на СА- тя се занимава с поведението и връзките между елементите, разглеждани като ''черни кутии''.

Съгласно дефиницията става ясно, че системите могат да имат (и имат)
повече от една структура. Нито една от тях самостоятелно не
представлява Архитектурата на системата (структура на модулите, на
процесите; елементите може да са обекти, модули, процеси, БД,
библиотеки, продукти, екипи и т.н.);
\subsection{Структури и изгледи на архитектурата. }
Структура – съвкупност от софтуерни елементи, техните външно видими свойства и връзките между тях.

Изглед - конкретно документирано представяне на дадена структура. (Двете понятия в голяма степен са взаимозаменяеми).

Структурите се делят на няколко типа.
\subsubsection{Модулни структури}
Елементите в модулните структури са модули – единици работа за изпълнение. Модулите предлагат поглед, ориентиран към реализацията на системата, без значение какво става по време на изпълнението.
Отговарят на въпросите:
\begin {itemize}
\item Коя функционалност в кой модул се реализира? 
\item Кои други модули може да използва (и използва) дадения модул? 
\item Как са свързани модулите по отношение на специализация и генерализация (наследяване)
\end {itemize}
Ще разгледаме няколко типа модулни структури.
\subsubsubsection{ Декомпозиция на модулите}
При тази структура връзките между модулите са от вида “Х е подмодул на У” и модулите биват рекурсивно разбивани на по-прости единици, докато станат лесни за разбиране. Декомпозицията на модулите обуславя в голяма степен възможността за лесна промяна, като обособява логически свързани функционалности на едно място.

\subsubsubsection{Употреба на модулите}
При този вид структура връзките между модулите са от вида “X използва Y”. Структурата за употребата на модули обуславя възможността за лесно добавяне на нова функционалност, обособяване на [в голяма степен] самостоятелни подмножества от функционалност, както и позволява последователната разработка.

\subsubsubsection{Структура на слоевете}
Частен случай на “употреба на модулите”. Модулите са разделени на слоеве, като модулите от слoй N може да ползват услугите само на модули от слой N-1.Слоевете често са реализирани като виртуални машини или обособени подсистеми, които скриват детайлите относно работата си от следващия слой. Подобна структура позволява лесна смяна на даден слой.

\subsubsubsection{Йерархия на класовете}
В терминологията на ООП, модулите се наричат “класове”, а в настоящата структура връзките между класовете са от вида “класът X наследява класа Y” и “обекта X е инстанция на клас Y”. Тази структура обосновава наследяването – защо подобни поведения или въобще функционалности са обособени в супер-класове или пък защо са дефинирани под-класове за обслужване на параметризирани различия.

\subsubsection{Структури на процесите}
Елементите са процеси (или нишки), изпълнявани в системата (компоненти) и комуникационни, синхронизационни или блокиращи операции между тях (конектори); Връзките между тях (attachments) показват как компонентите и конекторите се отнасят помежду си.

\subsubsection{Структури на разположението}
Структурите на разположението показват връзката между софтуерните елементи и елементите на околната среда, в която се намира системата по време на разработката или по време на изпълнението;
Отговарят на въпроситe
\begin {itemize}
\item На кой процесор се изпълнява всеки от елементите? 
\item В кои файлове се записва сорс кода на елементите по време на разработката?
\item Какво е разпределението на софтуерните елементи по екипи, които създават системата?
\end {itemize}

Ще разгледаме няколко типа структури на разпределението.

\subsubsubsection{Структура на внедряването}
Показва как софтуера се разполага върху хардуера и комуникационното оборудване. Елементите са процеси, хардуерни устройства и комуникационни канали. Връзките са напр. “внедрен върху” или “мигрира върху” - показвайки върху кое устройство е разположен даден софтуерен елемент.  Тази структура може да се използва за поглед върху производителността, сигурността и др. на дадена система.

\subsubsubsection{Файлова структура}
Показва кой модул къде се помещава във файловата структура по време на различните фази на реализация. Структурата е критична за управлението на дейностите по разработка и за създаването и поддържането на обкръжение за build-ове.
\subsubsubsection{Разпределение на работата}
Показва кой модул от кой екип се реализира. Елементите са модули и екипи. Екипите често не са списък от хора, а по-скоро виртуална група хора с подходящ опит, знания и умения.



\section{Изисквания към качеството (нефункционални изисквания) на системата}
Качествените изисквания определят как софтуерната система да работи. Качеството е субективно възприятие – различните ЗЛ могат да не одобрят даден дизайн, тъй като тяхната идея за качество се различава от идеята за качество на Архитекта. Бизнес целите определят Качествата, които трябва да бъдат вградени в архитектурата на системата. 

Качествата се разделят на следните три основни групи
\begin{itemize}
\itemТехнологични качества – напр. Надеждност, Изменяемост Производителност, Сигурност, Изпитаемост, Използваемост; 
\itemБизнес качества – напр. време за пускане на продукта на пазара; 
\itemАрхитектурни качества – присъщи на самата архитектура като напр. идейна цялост (влияят косвено върху всички останали качества);
\end{itemize}


Тези Качества поставят изисквания отвъд функционалните (описание на основните възможности на системата и услугите които тя предоставя). Въпреки че функционалността и Качествата са тясно свързани, функционалността често е единственото, което се взема под внимание по време на проектирането. Като следствие много системи се преправят не защото им липсва функционалност, а защото е трудно да се поддържат, трудно е да се смени платформата, не са скалируеми, прекалено са бавни, или пък са несигурни. СА е тази стъпка в процеса на създаването на системата, в която за пръв път се разглеждат качествените изисквания и в зависимост от тях се създават съответните структури, на които се вменява функционалност. За да притежава дадена система изискваните качествени характеристики, те трябва да се имат предвид както по време на проектирането, така и по време на разработката и внедряването.

От перспективата на архитекта, има три основни проблема:
\begin {itemize}
\item Не могат да се тестват - напр. Как бихме тествали дали система е “изменяема”?
\item Често се водят спорове към кое Качество принадлежи даден аспект на системата.
\item За всяко Качество си има собствен речник. Специалистите по Производителност говорят за “събития”, тези по Сигурност – за “атаки”, тези по Надеждност – за откази, и т.н. Всички тези термини всъщност могат да обозначават едно и също събитие.
\end {itemize}

Изискванията за качество трябва да се формализират от архитекта посредством т.н. “сценарии за качество”, за да бъдат те поставени на обективна основа. 

\subsection {Сценарии за качество}
Сценарият за качество е специфично изискване към поведението на системата в дадена ситуация, в светлината на дадено качество.Te играят същата роля за дефиниране на нефункционалните изисквания, каквато роля играят сценариите за употреба (usecases) за дефиниция на функционалните изисквания.  Всеки сценарий описва някаква случка и се характеризира с 
\begin {itemize}
\item Въздействие – състояние/събитие, което подлежи на обработка 
\item Източник – обект (човек, система или нещо друго) който генерира въздействието
\item Обект – системата, или конкретна нейна част, върху която се случва въздействието
\item Контекст – условията, при които се намира обекта по време на обработка на въздействието
\item Резултат – действията, предприети от обекта при случването на въздействието
\item Количествени параметри – резултатът трябва да подлежи на някакви количествени измервания, така че да позволи проверката дали сценарият се изпълнява съгласно изискванията
\end{itemize}
Пример: По време на експлоатация на системата, външен източник изпраща на процеса Х, съобщение за препълване на опашката с потребителски заявки. Х трябва да информира оператора за получаването на съобщението и да продължи работа без прекъсване.

\section{Проектиране на софтуерната архитектура. Процес за проектиране. Избор на
подходящи структури. Последователност на създаване на архитектурата.
Тактики (архитектурни решения) за постигане на желаните качествени
показатели.}
\section{}

\end{document}
