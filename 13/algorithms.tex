
\documentclass[fleqn,12pt]{article}

\usepackage[margin=15mm]{geometry}
\usepackage[utf8]{inputenc}
\usepackage[bulgarian]{babel}
\usepackage[unicode]{hyperref}
\usepackage{amsfonts}
\usepackage{amssymb}
\usepackage{enumitem, hyperref}
\usepackage{upgreek}
\usepackage{indentfirst}

\usepackage{amsmath}
\DeclareMathOperator{\cotg}{cotg}
\DeclareMathOperator{\LCS}{LCS}
\DeclareMathOperator{\longer}{longer}

\title{Структури от данни и алгоритми. Анализ на алгоритми. Абстрактни
типове от данни. Стек, опашка, списък, дърво. Сортиране.}

\author{v0.1}
\date{24 юни 2021}

\begin{document}

\maketitle

\tableofcontents

\section{Анализ на алгоритми}
\subsection{Въведение / необходимост}
TODO

\subsection{Асимптотична нотация на сложността}
TODO

\subsection{Основни рекурентни формули}
TODO

\subsection{Примери за анализ на алгоритми}
TODO

\section{Абстрактни типове от данни}
\subsection{Дефиниция и идея}
TODO

\subsection{Интерфейс и реализация}
TODO

\section{Свързани списъци}
\subsection{Дефиниция и структура}
TODO

\subsection{Обработка на списъци}
TODO

\section{Структура от данни стек}
\subsection{Дефиниция и структура}
TODO

\subsection{Реализация}
TODO

\section{Структура от данни опашка}
\subsection{Дефиниция и структура}
TODO

\subsection{Реализация}
TODO

\section{Дървета}
\subsection{Дефиниция и структура}
TODO

\subsection{Типове дървета}
TODO

\section{Сортиране}
\subsection{Дефиниция}
TODO

\subsection{Елементарни методи за сортиране}
TODO

\subsection{Сортиране - QuickSort}
TODO

\end{document}
