
\documentclass[fleqn,12pt]{article}

\usepackage[margin=15mm]{geometry}
\usepackage[utf8]{inputenc}
\usepackage[bulgarian]{babel}
\usepackage[unicode]{hyperref}
\usepackage{amsfonts}
\usepackage{amssymb}
\usepackage{enumitem, hyperref}
\usepackage{upgreek}
\usepackage{indentfirst}
\usepackage{array}
\usepackage{listings}

\usepackage{amsmath}
\DeclareMathOperator{\cotg}{cotg}
\DeclareMathOperator{\LCS}{LCS}
\DeclareMathOperator{\longer}{longer}

\title{Тема 12\\Обектно ориентирано програмиране (на базата на C++ или Java): Основни
принципи. Класове и обекти. Конструктори и деструктори. Оператори.
Производни класове и наследяване.}

\author{v0.1}
\date{28 юни 2021}

\begin{document}

\maketitle

\tableofcontents

\section{Основни принципи}
\subsection{Капсулация}
Обектите крият своето вътрешно състояние - само те могат да го модифицират директно. 
Клиентският код има достъп до него само през специфични методи на самия обект.

\subsection{Абстракция}
Обектите показват на външния свят само своя интерфейс, но не и допълнителни детайли. 
За всеки обект се знае какво прави, но не и как го прави.

\subsection{Наследяване}
Клас може да наследява друг клас. При това той включва в себе си всички полета и методи на класа, който е наследил.
Дали дадено поле от базовия клас ще е видимо в наследнения зависи от неговия спецификатор на достъп:
\begin{itemize}
    \item \textbf{public} - вижда се навсякъде
    \item \textbf{private} - вижда се само в базовия клас
    \item \textbf{protected} - вижда се в базовия и всички наследници
\end{itemize}

\subsection{Полиморфизъм}
Буквален превод - \textit{много форми}. В случая означава, че обект от наследен клас може да се ползва
като обект от базов клас.

\section{Класове}
\subsection{Дефиниция}
TODO

\subsection{Дефиниране на класове}
TODO

\subsection{Област на класове}
TODO

\subsection{Обекти}
Обектите са инстанции (екземпляри) на класове.

\section{Конструктори}
\subsection{Дефиниция}
Конструкторите са специални методи, които се извикват единствено при създаване на обект.

\subsection{Дефиниране на конструктори}
Дефиниите на конструктори са от вида \textbf{<class name>(<parameter list>) : <field>(value), <field>(value), \dots}.
Съответно може да има само дефиниця, при която веднага следва блок с кода на конструктора,
или декларация + дефиниция.

\subsection{Видове конструктори}
\subsubsection{По подразбиране}
Важен вид конструктор е констуктора по подразбиране. Той се генерира автоматично от компилатора, 
когато не са дефинирани конструктори. Не приема аргументи и единствено инициализира всички полета.
Извиква се, когато обект се създава без скобки, например:
\begin{lstlisting}[language=C++, caption=Default constructor]
class X
{
    // ...
};

X x; // default constructor called
\end{lstlisting}

\subsection{Параметризиран конструктор}
Това е конструктор, който приема параметри. Пише се ръчно от програмиста, който решава и какво точно да прави с тях
в тялото на конструктора. Може да има произволен брой параметризирани конструктори.

\subsection{Копиращ конструктор (copy constructor)}
Това е специален параметризиран конструктор, който като единствен параметър получава псевдоним (\textbf{T\&}) на обект от същия клас.
Работата му е да клонира подадения обект. Компилатор създава копиращ конструктор по подразбиране.

\subsection{Преместващ конструктор (move constructor)}
Този приема единствен параметър от тип \textbf{T\&\&} и работата му е да \textit{премести} съдържанието на подадения обект 
в новосъздаващият се. Обектът-параметър остава \textit{празен} и най-вероятно не трябва да се използва.
Компилатора създава такъв конструктор по подразбиране.

\section{Указатели към обекти. Масиви и обекти. Динамични обекти}
\subsection{Указатели към обекти}
TODO

\subsection{Масиви от обекти}
TODO

\subsection{Динамични обекти}
TODO


\section{Деструктори}
\subsection{Дефиниция}
Деструкторите са специални методи, които се извикват при унищожаване на обекти.

\subsection{Създаване и разрушаване на обекти на класове}
TODO


\section{Оператори}
\subsection{Дефиниция}
TODO

\subsection{Предефиниране на оператори}
TODO


\section{Производни класове}
\subsection{Дефиниция}
TODO

\subsection{Дефиниране}
TODO

\subsection{Наследяване и достъп до наследените компоненти}
TODO


\end{document}
