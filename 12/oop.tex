
\documentclass[fleqn,12pt]{article}

\usepackage[margin=15mm]{geometry}
\usepackage[utf8]{inputenc}
\usepackage[bulgarian]{babel}
\usepackage[unicode]{hyperref}
\usepackage{amsfonts}
\usepackage{amssymb}
\usepackage{enumitem, hyperref}
\usepackage{upgreek}
\usepackage{indentfirst}
\usepackage{array}
\usepackage{listings}

\usepackage{amsmath}
\DeclareMathOperator{\cotg}{cotg}
\DeclareMathOperator{\LCS}{LCS}
\DeclareMathOperator{\longer}{longer}

\title{Тема 12\\Обектно ориентирано програмиране (на базата на C++ или Java): Основни
принципи. Класове и обекти. Конструктори и деструктори. Оператори.
Производни класове и наследяване.}

\author{v0.1}
\date{28 юни 2021}

\begin{document}

\maketitle

\tableofcontents

\section{Основни принципи}

\section{Класове}
\subsection{Дефиниция}
TODO

\subsection{Дефиниране на класове}
TODO

\subsection{Област на класове}
TODO

\subsection{Обекти}
TODO


\section{Конструктори}
\subsection{Дефиниция}
TODO

\subsection{Дефиниране на конструктори}
TODO

\subsection{Видове конструктори}
TODO


\section{Указатели към обекти. Масиви и обекти. Динамични обекти}
\subsection{Указатели към обекти}
TODO

\subsection{Масиви от обекти}
TODO

\subsection{Динамични обекти}
TODO


\section{Деструктори}
\subsection{Дефиниция}
TODO

\subsection{Създаване и разрушаване на обекти на класове}
TODO


\section{Оператори}
\subsection{Дефиниция}
TODO

\subsection{Предефиниране на оператори}
TODO


\section{Производни класове}
\subsection{Дефиниция}
TODO

\subsection{Дефиниране}
TODO

\subsection{Наследяване и достъп до наследените компоненти}
TODO


\end{document}
