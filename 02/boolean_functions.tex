
\documentclass[fleqn,12pt]{article}

\usepackage[margin=15mm]{geometry}
\usepackage[utf8]{inputenc}
\usepackage[bulgarian]{babel}
\usepackage[unicode]{hyperref}
\usepackage{amsfonts}
\usepackage{amssymb}
\usepackage{enumitem, hyperref}
\usepackage{upgreek}

\usepackage{amsmath}
\DeclareMathOperator{\cotg}{cotg}
\DeclareMathOperator{\LCS}{LCS}
\DeclareMathOperator{\longer}{longer}
\renewcommand{\arraystretch}{1.3}         % because math expressions

\title{Булеви функции. Пълнота}
\author{v0.9}
\date{30 май 2021}

\begin{document}

\maketitle

\tableofcontents

\begin{flushleft}

\section{Дефиниция на булева функция (БФ) и формула над множество БФ}

\subsection{Булева функция}
Нека $n \in \mathbb{N}, n \geq 1$ и означим $J_2 = \{ 0, 1 \}$. 
Булева функция на $n$ променливи наричаме всяка функция $f : J_2^n \rightarrow J_2$.
Множеството от всички булеви функции на $n$ променливи означаваме
$\mathcal{F}_2^n$, а множеството на всички булеви функции $\mathcal{F}_2 = \cup_{n \in \mathbb{N}} \mathcal{F}_2^n$.

\subsection{Формула над множество БФ}
Нека $X = \{ x_0, x_1, \dots \}$ - изброимо множество от променливи за
всяка булева функция от $\mathcal{F}_2$, $F = \{ f_i \} \subseteq \mathcal{F}_2, I \subseteq \mathbb{N}$
и $H = \{ f, x, (, ), \text{запетая}  \} \cup I $ - азбука.
Формула над множеството $F$, дефинираме като всяка дума $w \in H*$, удовлетворяваща:
\begin{enumerate}
    \item $f_i \in F \Rightarrow f_i(x_1,x_2,\dots,x_n)$ е формула над $F$.
    \item \label{formulas:superposition} Нека $f_i \in F$ и $\varphi_1, \varphi_2, \dots, \varphi_n$ са формули над $F$ или променливи от $X$.
    Тогава $f_i(\varphi_1, \varphi_2, \dots, \varphi_n) \in H*$ е формула над $F$. Формулите $\varphi_1, \varphi_2, \dots, \varphi_n$ наричаме подформули на $\varphi$.
    \item Няма други формули освен горедефинираните.
\end{enumerate}

\section{БФ с 1 и 2 променливи}

\subsection{Булеви функции с 1 променлива}
Съществуват 4 булеви функции на една променлива:
\begin{center}
\begin{tabular}{ |c|c|c|c|c| } 
    \hline
    $x$ & $\widetilde{0}$ & $x$ & $\overline{x}$ & $\widetilde{1}$ \\ 
    \hline
    0 & 0 & 0 & 1 & 1 \\ 
    1 & 0 & 1 & 0 & 1 \\ 
    \hline
\end{tabular}
\end{center}

\subsection{Булеви функции с 2 променлива}
Съществуват 16 булеви функции на две променливи, от които именуваните са в тази таблица:
\begin{center}
\begin{tabular}{ |c|c|c|c|c|c|c|c|c| } 
    \hline
    $x$ & $y$ & $x \vee y$ & $x \wedge y$ & $x \oplus y$ & $x \rightarrow y$ & $y \rightarrow x$ & $x \downarrow y$ & $x | y$ \\ 
    \hline
    0 & 0 & 0 & 0 & 0 & 1 & 1 & 1 & 1 \\ 
    0 & 1 & 1 & 0 & 1 & 1 & 0 & 0 & 1 \\ 
    1 & 0 & 1 & 0 & 1 & 0 & 1 & 0 & 1 \\ 
    1 & 1 & 1 & 1 & 0 & 1 & 1 & 0 & 0 \\ 
    \hline
\end{tabular}
\end{center}
Имената им са както следва:
\begin{itemize}
    \item $x \vee y$ - дизюнкция
    \item $x \wedge y$ - конюнкция. За кратко записваме $xy = x \wedge y$
    \item $x \oplus y$ - изключващо или
    \item $x \rightarrow y$ и $y \rightarrow x$ - импликации
    \item $x \downarrow y$ - стрелка на Пиърс (отрицание на дизюнкцията)
    \item $x|y$ - черта на Шефер (отрицание на конюнкцията)
\end{itemize}

\section{Свойства на булеви функции}

Нека $x,y,z \in J_2$. В сила са следните свойства:
\begin{enumerate}
    \item Комутативност: $x \vee y = y \vee x, x \wedge y = y \wedge x, x \oplus y = y \oplus x$
    \item Асоциативност: $(x \vee y) \vee z = x \vee (y \vee z), (x \wedge y) \wedge z = x \wedge (y \wedge z), (x \oplus y) \oplus z = x \oplus (y \oplus z)$
    \item Дистрибутивност: $(x \wedge y) \vee z = (x \vee z) \wedge (y \vee z)$, $(x \vee y) \wedge z = xz \vee yz$, $(x \oplus y) \wedge z = xz \oplus yz$
    \item Идемпотентност: $x \wedge x = x$, $x \wedge x = x$, $x \oplus x = \widetilde{0}$
    \item Свойства на отрицанието: $x\overline{x} = \widetilde{0}$, $x \vee \overline{x} = \widetilde{1}$, $x \oplus \overline{x} = \widetilde{1}$, $\overline{\overline{{x}}} = x$
    \item Свойства на константите: $x\widetilde{0} = 0$, $x\widetilde{1} = x$, $x\vee \widetilde{0} = x$, $x \vee \widetilde{1} = 1$, $x \oplus \widetilde{0} = x$, $x \oplus \widetilde{1} = \overline{x}$
    \item Закони на де Морган: $\overline{x \vee y} = \overline{x} \wedge \overline{y}$, $\overline{x \wedge y} = \overline{x} \vee \overline{y}$.  
\end{enumerate}


\section{Дефиниция на пълно множество БФ}
\subsection{Затваряне на множество булеви функции}
Нека $F \subseteq \mathcal{F}_2$ и $\upphi_F$ е множеството от формули над $F$.
На всяка формула $\varphi \in \upphi_F$ съпоставяме функция $f \in F$ и записваме $\varphi : f$, като 
следваме дефиницията на $\varphi$ по следния начин:
\begin{itemize}
    \item Ако $\varphi$ е съпоствена на $f_i : F$, то $\varphi : f = f_i$.
    \item Ако е $\varphi$ съставена по \ref{formulas:superposition}, тогава на всякa подформула $\varphi_i$ съпоставяме $g_i \in F$, 
    която е или проектираща функция, или се получава след рекурсивно прилагане на тези правила. На $\varphi$ съпоставяме 
    суперпозицията $h(x_1, \dots, x_k) = f(g_1(x_1, \dots, x_k), \dots, g_1(x_n, \dots, x_k))$.
\end{itemize}

Затваряне на $F$ наричаме множеството $[F] = \{ f \in \mathcal{F}_2 | \exists \varphi \in \upphi_F (\varphi : f) \}$

\subsection{Пълно множество булеви функции}
Множество булеви функции $F$ наричаме пълно $\Leftrightarrow [F] = \mathcal{F}_2$.

\section{Теорема за разбиване на БФ по част от променливите и теорема на Бул}

\subsection{Помощни дефиниции}
\[ x^\sigma = \begin{cases}
    x, & \sigma = 1 \\
    \overline{x}, & \sigma = 0 \\
\end{cases}\]
\textbf{Лема: } $x^\sigma = 1 \Leftrightarrow x = \sigma$

\textbf{Дефиниция: } Формули от вида $x_{i_1}^{\sigma_1} x_{i_2}^{\sigma_2} \dots x_{i_n}^{\sigma_n}$, където
$\sigma_i \in J_2$ и $x_i \neq x_j$ при $i \neq j$ наричаме елементарни конюнкции.
 
\textbf{Дефиниция: } Наричаме горната формула пълна елементарна конюнкция, ако и променливите $x_{i_1}, \dots, x_{i_n}$ са фиксирани.

\textbf{Лема: } $x_{i_1}^{\sigma_1} x_{i_2}^{\sigma_2} \dots x_{i_n}^{\sigma_n} = 1 \Leftrightarrow x_{i_k} = \sigma_k$

\subsection{Теорема за разбиване на БФ}
Нека $f(x_1, x_2, \dots, x_n) \in \mathcal{F}_2$ и нека са избрани $1 \leq i \leq n$ променливи.
Без ограничение на общността, допускаме че това са първите $i$ променливи. Тогава

\[ f(x_1, x_2, \dots, x_n) = \bigvee_{\forall \sigma_1, \sigma_2, \dots, \sigma_i} x_1^{\sigma_1} x_2^{\sigma_2} \dots x_i^{\sigma_i} f(\sigma_1, \sigma_2, \dots, \sigma_n, x_{i+1}, \dots, x_n) \]

\textbf{Доказателство: } Нека фиксираме произволен вектор $(a_1, a_2, \dots, a_n) \in J_2^n$.
Разглеждаме дясната страна на твърдението. Знаем, че 
$x_1^{\sigma_1} x_2^{\sigma_2} \dots x_i^{\sigma_i} = 1 \Leftrightarrow x_i = \sigma_i \Leftrightarrow a_i = \sigma_i$.
Следвателно за всички останали комбинации $a_1^{\sigma_1} a_2^{\sigma_2} \dots a_i^{\sigma_i} = 0$. Тогава дясната страна става равна на 
$f(a_1, a_2, \dots, a_i, a_{i+1}, a_n)$, което е точно лявата страна на твърдението. Твъдението е вярно за произволен вектор, следователно
функциите са еквивалентни за всички възможни вектори.

\subsection{Теорема на Бул}
Множеството $\{ \vee, \wedge, \overline{x} \}$ е пълно.

\textbf{Доказателство: } Ще покажем как можем да построим всяка функция $f \in \mathcal{F}_2$:
\begin{itemize}
    \item Ако $f = \widetilde{0}$, то $f = x \wedge \overline{x}$.
    \item В противен случай, разлагаме $f$ по всичко променливи и получаваме
    \[ f(x_1, x_2, \dots, x_n) = \bigvee_{\forall \sigma_1, \sigma_2, \dots, \sigma_n} x_1^{\sigma_1} x_2^{\sigma_2} \dots x_n^{\sigma_n} f(\sigma_1, \sigma_2, \dots, \sigma_n) =
    \bigvee_{\substack{\forall \sigma_1, \sigma_2, \dots, \sigma_n \\ f(\sigma_1, \sigma_2, \dots, \sigma_n) = 1}} x_1^{\sigma_1} x_2^{\sigma_2} \dots x_n^{\sigma_n}\]
    Това е формула над множеството $\{ \vee, \wedge, \overline{x} \}$.
\end{itemize}

\subsection{Съвършена дизюнктивна нормална форма}
Нека $f(x_1, x_2, \dots, x_n) \in \mathcal{F}_2$. Съвършена дизюнктивна нормална форма на $f$ наричаме представянето
\[ f(x_1, x_2, \dots, x_n) = 
    \bigvee_{\substack{\forall \sigma_1, \sigma_2, \dots, \sigma_n \\ f(\sigma_1, \sigma_2, \dots, \sigma_n) = 1}} x_1^{\sigma_1} x_2^{\sigma_2} \dots x_n^{\sigma_n}\]

\subsection{Други пълни множества}
TODO

\section{Теорема на Пост}
\subsection{Полином на Жегалкин}
Нека $f \in \mathcal{F}_2^n$. Казваме, че $f$ е полином на Жегалкин, ако $f$ има вида:
\[ f(x_1, x_2, \dots, x_n) = a_0 x_0 \oplus a_1 x_1 \oplus \dots \oplus a_n x_n \oplus a_{n+1} x_0 x_1 \oplus \dots \oplus a_m x_0 x_1 \dots x_n = \]
\[ = \sum_{X' \subseteq \{x_1, x_2, \dots, x_n\}} a_i \prod(X') \]

\subsection{"Запазващи" функции}
Казваме, че функцията $f \in \mathcal{F}_2$ запазва нулата, ако $f(0, 0, \dots, 0) = 0$.
Аналогично казваме, че запазва 1, ако $f(1, 1, \dots, 1) = 1$.
Дефинираме множествата $T_i = \{f \in \mathcal{F}_2 | f \text{ запазва } i\}$ за $i=0,1$.

\subsection{Линейни функции}
Казваме, че функцията $f \in \mathcal{F}_2^n$ е линейна, ако $f = a_0 x_0 \oplus a_1 x_1 \oplus \dots a_n x_n$.
Множеството на всички линейни функции означаваме със $L$.

\subsection{Двойствени и самоспрегнати функции}
Казваме, че функцията $f^* \in \mathcal{F}_2^n$ е двойствена на $f \in \mathcal{F}_2^n \Leftrightarrow$
$f^*(x_1, x_2, \dots, x_n) = \overline{f(\overline{x_1}, \overline{x_2}, \dots, \overline{x_n})}$.
Казваме, че функцията $f \in \mathcal{F}_2$ е самоспрегната $\Leftrightarrow f = f^*$ и обозначаваме със $S$ множеството
на всички самоспрегнати функции.

\subsection{Монотонни функции}
Нека $\alpha = (a_1, a_2, \dots, a_n), \beta = (b_1, b_2, \dots, b_n) \in J_2^n$. Казваме, че
$\alpha$ е предшественик на $\beta$ и обозначаваме $\alpha \preceq \beta \Leftrightarrow a_1 \leq b_1, a_2 \leq b_2, \dots, a_n \leq b_n$.

Казваме, че $f \in \mathcal{F}_2^n$ е монотонна, ако за произволни $\alpha, \beta \in J_2^n$ е изпълнено
$\alpha \preceq \beta \Rightarrow f(\alpha) \leq f(\beta)$. Обозначаваме с $M$ множеството на всичко монотонни функции.

\section{Критерий на Пост}
Нека $F \subseteq \mathcal{F}_2$. $F$ е пълно $\Leftrightarrow F \nsubseteq T_0, F \nsubseteq T_1, F \nsubseteq L, F \nsubseteq M, F \nsubseteq S$.    

\end{flushleft}
\end{document}
