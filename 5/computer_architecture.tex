
\documentclass[fleqn,12pt]{article}

\usepackage[margin=15mm]{geometry}
\usepackage[utf8]{inputenc}
\usepackage[bulgarian]{babel}
\usepackage[unicode]{hyperref}
\usepackage{amsfonts}
\usepackage{amssymb}
\usepackage{enumitem, hyperref}
\usepackage{upgreek}

\usepackage{amsmath}
\DeclareMathOperator{\cotg}{cotg}
\DeclareMathOperator{\LCS}{LCS}
\DeclareMathOperator{\longer}{longer}
\renewcommand{\arraystretch}{1.3}         % because math expressions

\title{Компютърни архитектури. Формати на данните. Вътрешна структура на централен процесор – блокове и конвейерна обработка, инструкции.}
\author{v0.1}
\date{31 май 2021}

\begin{document}

\maketitle

\tableofcontents

\begin{flushleft}

\section{Обща структура на компютрите и концептуално изпълнение на инструкциите,
запомнена програма.}

\section{Формати на данните}
\subsection{Цели двоични числа}
\subsection{Двоично-десетични числа}
\subsection{Двоично числа с плаваща запетая}
\subsection{Символни данни и кодови таблици}

\section{Вътрешна структура на централен процесор}
\subsection{Регистри}
\subsection{Аритметико-логическо устройство}
\subsection{Регистър на състоянието и флагове}
\subsection{Блок за управление}

\section{Инструкции на централен процесор}
\subsection{Префикси}
\subsection{Код на операцията}
\subsection{Местоположение на операндите}
\subsection{Модели на адресация на операндите}
\subsection{Аритметико-логически инструкции}
\subsection{Низови инструкции}
\subsection{Безусловни и условни преходи}
\subsection{Управление на програмата}


\end{flushleft}
\end{document}
