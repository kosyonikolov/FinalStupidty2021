\documentclass[fleqn,12pt]{article}

\usepackage[margin=15mm]{geometry}
\usepackage[utf8]{inputenc}
\usepackage[bulgarian]{babel}
\usepackage[unicode]{hyperref}
\usepackage{amsfonts}
\usepackage{amssymb}
\usepackage{enumitem, hyperref}

\usepackage{amsmath}
\usepackage{tikz}

\DeclareMathOperator{\cotg}{cotg}
\DeclareMathOperator{\LCS}{LCS}
\DeclareMathOperator{\longer}{longer}

\setlength{\parskip}{1em}
\usetikzlibrary{automata,positioning}

\title{Крайни автомати. Регулярни езици. Thm на Клини.}
\author{v0.1}
\date{30 май 2021}

\begin{document}

\maketitle

\tableofcontents
\pagebreak
\begin{flushleft}

\section{Дефиниции на автомати и регулярни езици}
\subsection{Детерминирани крайни автомати}

\textit{\textbf{Деф}} - Краен детерминиран автомат (КДА) наричаме наредената петорка $M = (K, \Sigma, \delta, s, F)$, където:
\begin{itemize}
    \item $K$ - крайна азбука от състояния,
    \item $\Sigma$ - основна азбука (крайна),
    \item $\delta : K \times \Sigma \rightarrow K$ функция на преходите,
    \item $s \in K$, $s$ - начално състояние,
    \item $F \subseteq K$, $F$ - мн-во на крайните състояния.
\end{itemize}

\textit{\textbf{Деф}} - Конфигурация за M се нарича всяка двойка $(q, w)$, където $q \in K$ и $w \in \Sigma^*$

\textit{\textbf{Деф}} - Казваме, че конфигурацията $(q, w)$ се преработва за 1 стъпка в конф. $(q', w') \iff \exists a \in \Sigma : \delta(q, a) = q'$ и $w = aw'$. Означаваме с $\vdash_M$, т.е. $(q, w) \vdash_M (q', w')$.

\textit{\textbf{Деф}} - $\vdash_M^*$ наричаме рефлексивното и транзитивно затваряне на $\vdash_M$.

\textit{\textbf{Деф}} - Казваме, че $M$ разпознава (приема) думата $w \iff (s, w) \vdash_M^* (f, \varepsilon)$, където $f \in F$.

\textit{\textbf{Деф}} - $L(M) = \{w | w \in \Sigma^*$ и $w$ се разпознава от $M\}$ се нарича езика, който се разпознава от автомата $M$. Казваме, че езика $L$ се разпознава от автомата $M$ ако $L = L(M)$.

КДА $M = (K, \Sigma, \delta, s, F)$ представяме чрез краен ориентиран мултиграф, където върховете са елементите на $K$, такива че върховете съотвестващи на $q, p \in K: \delta(q, x) = p, x \in \Sigma$, са свързани с ребро излизащо от $q$ и влизащо в $p$ с надпис $x$.

\begin{tikzpicture}[shorten >=1pt,node distance=2cm,on grid,auto] 
    \node[state,initial,accepting] (s)   {$s$}; 
    \node[state] (q) [right=of s] {$q$}; 
    \path[->] 
    (s) edge[bend left, above] node{a} (q)
    (q) edge[bend left, below]  node{a} (s);
\end{tikzpicture}

\subsection{Регулярни операции}

\textit{\textbf{Деф}} - $\Sigma^*$ наричаме множеството от всички думи в азбуката $\Sigma$.

\textit{\textbf{Деф}} - Казваме, че $L$ е език в азб. $\Sigma$ ако $L \subseteq \Sigma^*$.

\textit{\textbf{Деф}} - Конкатенация на езиците $L_1$, $L_2$ в азбуката $\Sigma$ отбелязваме като $L_1 \circ L_2$ и дефинираме като $L_1 \circ L_2 = \{w_1 \circ w_2 | w_1 \in L_1 $ и $ w_2 \in L_2\}$.

\textit{Пример - $\{li, on\} \circ \{leo, pard\} = \{lileo, lipard, onleo, onpard\}$}.

\textit{\textbf{Деф}} - Обединението на езиците $L_1$ и $L_2$ в азб. $\Sigma$ бележим като $L_1 \cup L_2$ и дефинираме като $L_1 \cup L_2 = \{w | w \in L_1 $ или $ w \in L_2\}$.

\textit{Пример - $\{a, b\} \cup \{pumba, jay, b\} = \{a, b, pumbda, jay, b\}$}.

\textit{\textbf{Деф}} - Степен в език $L$ в азбука $\Sigma$ дефинираме индуктивно като:
\begin{enumerate}
    \item $L^0 = \{\varepsilon\}$
    \item $L^{i+1} = L^i \circ L, i \in \mathbb{N}$
\end{enumerate}}

\textit{\textbf{Деф}} - Звезда на Клини на език $L$ в азбука $\Sigma$ бележим като $L^*$ и дефинираме като $L^* = \bigcup\limits_{i=0}^{\infty} L^{i} = L^0 \cup L^1 \cup L^2 \dots = \{w | w = w_1 \circ \dots \circ w_k, w_0, \dots, w_k \in L $ и $ k \in \mathbb{N}\}$

\textit{Пример - $\{a, b\}^* = \{\varepsilon, a, b, aa, bb, ab, ba, bb, aaa, aab, \dots\}$}.

\textit{\textbf{Деф}} - Допълнение на език $L$ наричаме $L^C = \Sigma^* \textbackslash L$.

\subsection{Недетерминирани крайни автомати}

\textit{\textbf{Деф}} - Недетерминиран краен автомат (КНА) наричаме петицата $M = (K, \Sigma, \Delta, s, F)$, където:
\begin{itemize}
    \item $K$ - крайна азбука от състояния,
    \item $\Sigma$ - крайна основна азбука,
    \item $\Delta$ - релация на преходите, $\Delta \subseteq K \times (\Sigma \cup \{ \varepsilon \} ) \times K$,
    \item $F$ - мн-во от заключителни състояния, $F \subseteq K$,
    \item $s$ - начално състояние, $s \in K$.
\end{itemize}

\textit{\textbf{Заб} - Всеки КДА $M = (K, \Sigma, \delta, s, F)$ може да се разглежда като КНА $M = (K, \Sigma, G_\delta, s, F)$, където $G_\delta = \{(q, a, q') | \delta(q, a) = q' \}$.}

\textit{\textbf{Деф}} - Конфигурация за КНА $M = (K, \Sigma, \Delta, s, F)$ наричаме всяка двойка $(q, w)$, такава че $q \in K$ и $w \in \Sigma^*$.

\textit{\textbf{Деф}} - Нека $M = (K, \Sigma, \Delta, s, F)$ е КНА. Казваме, че конфигурацията $(q, w)$ се преработва за 1 стъпка в конф. $(q', w') \iff \exists u \in \Sigma \cup \{\varepsilon\}$, такова че $w = uw'$ и $(q, u, q') \in \Delta$. Означаваме като $(q, w) \vdash_M (q', w')$.

\textit{\textbf{Деф}} - $\vdash_M^*$ наричаме рефлексивното и транзитивно затваряне на $\vdash_M$.

\textit{\textbf{Деф}} - Казваме, че $w \in \Sigma^*$ се приема (разпознава) от КНА $M = (K, \Sigma, \Delta, s, F) \iff (s, w) \vdash_M^* (f, \varepsilon)$, където $f \in F$. 

\textit{\textbf{Деф}} - $L(M) = \{w | w \in \Sigma^*: (s, w) \vdash_M^* (f, \varepsilon)$, където $f \in F\}$ се нарича езика, който се разпознава от автомата $M = (K, \Sigma, \Delta, s, F)$.

\section{Представяне на всеки КНА с КДА}

\textit{\textbf{Thm}} - Нека $M = (K, \Sigma, \Delta, s, F)$ е КНА. Тогава $\exists$ КДА $M'$, такъв че $L(M) = L(M')$.

\textit{\textbf{Д-во:}} \\
Нека $M' = (K', \Sigma, \delta', s', F')$ е КДА. Ще построим $M'$, така че $L(M) = L(M')$.

\textit{\textbf{Деф}} - Определяме релацията $E(q) = \{p | (q, \varepsilon) \vdash_M^* (p, \varepsilon)\}$, т.е. $E(q)$ е затваряне на $\{q\}$ относно релацията $\{(p_1, p_2) | (p_1, \varepsilon, p_2) \in \Delta\}$.

\textit{\textbf{Пример} - $E(q_0) = \{q_0, q_1, q_2, q_3\}$, $E(q_1) = \{q_1, q_2, q_3\}$, $E(q_2) = \{q_2\}$, $E(q_3) = \{q_3\}$, $E(q_4) = \{q_4, q_3\}$.}

\begin{tikzpicture}[shorten >=1pt,node distance=2cm,on grid,auto] 
    \node[state,initial] (q0) {$q_0$};
    \node[state] (q1) [above right=of q0] {$q_1$};
    \node[state] (q2) [below right=of q0] {$q_2$};
    \node[state] (q3) [right=of q1] {$q_3$};
    \node[state, accepting] (q4) [right=of q2] {$q_4$};
    \path[->] 
    (q0) edge[above] node{$\varepsilon$} (q1)
    (q1) edge[below] node{$\varepsilon$} (q3)
    (q1) edge[left] node{$\varepsilon$} (q2)
    (q4) edge[right] node{$\varepsilon$} (q3)
\end{tikzpicture}

Нека $K' = \rho(K)$, $s' = E(s)$, $F' = \{Q | Q \in K' $ и $ Q \cap K \neq \emptyset \}$, $\delta'(Q, a) = \bigcup \{E(p) | \exists q \in Q: (q, a, p) \in \Delta\}$, където $Q \in K'$ и $a \in \Sigma$.

\textit{\textbf{Лема 1}} - $(s, w) \vdash_M^* (p, \varepsilon) \iff (E(s), w) \vdash_{M'}^* (P, \varepsilon)$, където $p \in P$ и $P \in K'$.

Допускаме, че сме доказали \textit{\textbf{Лема 1}} и тогава: \\

$w \in L(M')$ $\iff$ $(E(s), w) \vdash_{M'}^* (P, \varepsilon), P \in F'$ $\iff$ $(s, w) \vdash_M^* (f, \varepsilon), f \in P$, т.е. $w \in L(M)$.

Следователно доказахме, че $L(M) = L(M')$.

\textit{\textbf{Д-во  на Лема 1:}} \\

Ще го докажем чрез индукция относно $|w|$:

\textbf{База} - Нека $|w| = 0$, т.е. $w = \varepsilon$. Тогава: \\
$(s, \varepsilon) \vdash_M^* (p, \varepsilon) \iff p \in E(s)$ \\
$(E(s), \varepsilon) \vdash_{M'}^* (P, \varepsilon), P \in K' \iff P = E(s)$ \\
Очевидно е изпълнено за $|w| = 0$ \\

\textbf{ИП} - Допускаме, че \textit{\textbf{Лема 1}} е изпълнена за $|w| = n$. \\

\textbf{Стъпка} - Ще го докажем за $|w| = n + 1$. Нека $w = va, a \in \Sigma$, т.е. $|v| = n$. \\

$(\implies)$ Нека $(s, w) \vdash_M^* (p, \varepsilon)$, т.е. $(s, va) \vdash_M^* (r_1, a) \vdash_M (r_2, \varepsilon) \vdash_M^* (p, \varepsilon)$ $\implies$ $(s, v) \vdash_M^* (r_1, \varepsilon)$. Съгласно ИП $\exists R_1: r_1 \in R_1: (E(s), v) \vdash_{M'}^* (R_1, \varepsilon)$. \\
Разглеждаме $P = \delta'(R_1, a) = \bigcup \{E(r_2') | \exists r_1' \in R_1: (r_1', a, r_2') \in \Delta\}$, но знаем, че $r_1 \in R_1$ и $(r_1, a, r_2) \in \Delta$ $\implies$ $E(r_2) \in \delta'(R_1, a)$, като от $p \in E(r_2)$ $\implies$ $(E(s), va) \vdash_{M'}^* (R_1, a) \vdash_{M'} (P, \varepsilon)$. \\

$(\impliedby)$ Нека $(E(s), va) \vdash_{M'}^* (P, \varepsilon)$, т.е. $(E(s), va) \vdash_{M'}^* (R_1, a) \vdash_{M'} (P, \varepsilon)$, $R_1 \in K'$, т.е. $(E(s), v) \vdash_{M'}^* (R_1, \varepsilon)$.\\
От ИП за някое $r_1 \in R_1$ е изпълнено, че $(s, v) \vdash_M^* (r_1, \varepsilon)$ и от дефиницията $\implies$ $\exists r_2 \in \delta'(R_1, a): (r_1, a, r_2) \in \Delta$ и $\exists p \in F: p \in E(r_2)$ $\implies$ $(s, va) \vdash_M^* (r_1, a) \vdash_M (r_2, \varepsilon) \vdash_M^* (p, \varepsilon), p \in P$. \square


\end{flushleft}
\end{document}
