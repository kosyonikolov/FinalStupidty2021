\documentclass[fleqn,12pt]{article}

\usepackage[margin=15mm]{geometry}
\usepackage[utf8]{inputenc}
\usepackage[bulgarian]{babel}
\usepackage[unicode]{hyperref}
\usepackage{amsfonts}
\usepackage{amssymb}
\usepackage{enumitem, hyperref}

\usepackage{amsmath}
\usepackage{tikz}

\DeclareMathOperator{\cotg}{cotg}
\DeclareMathOperator{\LCS}{LCS}
\DeclareMathOperator{\longer}{longer}

\setlength{\parskip}{1em}
\usetikzlibrary{automata,positioning}

\title{Крайни автомати. Регулярни езици. Thm на Клини.}
\author{v0.1}
\date{29 май 2021}

\begin{document}

\maketitle

\tableofcontents
\pagebreak
\begin{flushleft}

\section{Детерминирани крайни автомати}

\textit{\textbf{Деф}} - Краен детерминиран автомат (КДА) наричаме наредената петорка $M = (K, \Sigma, \delta, s, F)$, където:
\begin{itemize}
    \item $K$ - крайна азбука от състояния,
    \item $\Sigma$ - основна азбука (крайна),
    \item $\delta : K \times \Sigma \rightarrow K$ функция на преходите,
    \item $s \in K$, $s$ - начално състояние,
    \item $F \subseteq K$, $F$ - мн-во на крайните състояния.
\end{itemize}

\textit{\textbf{Деф}} - Конфигурация за M се нарича всяка двойка $(q, w)$, където $q \in K$ и $w \in \Sigma^*$

\textit{\textbf{Деф}} - Казваме, че конфигурацията $(q, w)$ се преработва за 1 стъпка в конф. $(q', w') \iff \exists a \in \Sigma : \delta(q, a) = q'$ и $w = aw'$. Означаваме с $\vdash_M$, т.е. $(q, w) \vdash_M (q', w')$.

\textit{\textbf{Деф}} - $\vdash_M^*$ наричаме рефлексивното и транзитивно затваряне на $\vdash_M$.

\textit{\textbf{Деф}} - Казваме, че $M$ разпознава (приема) думата $w \iff (s, w) \vdash_M^* (f, \varepsilon)$, където $f \in F$.

\textit{\textbf{Деф}} - $L(M) = \{w | w \in \Sigma^*$ и $w$ се разпознава от $M\}$ се нарича езика, който се разпознава от автомата $M$. Казваме, че езика $L$ се разпознава от автомата $M$ ако $L = L(M)$.

КДА $M = (K, \Sigma, \delta, s, F)$ представяме чрез краен ориентиран мултиграф, където върховете са елементите на $K$, такива че върховете съотвестващи на $q, p \in K: \delta(q, x) = p, x \in \Sigma$, са свързани с ребро излизащо от $q$ и влизащо в $p$ с надпис $x$.

\begin{tikzpicture}[shorten >=1pt,node distance=2cm,on grid,auto] 
    \node[state,initial,accepting] (s)   {$s$}; 
    \node[state] (q) [right=of s] {$q$}; 
    \path[->] 
    (s) edge[bend left, above] node{a} (q)
    (q) edge[bend left, below]  node{a} (s);
\end{tikzpicture}

\section{Регулярни операции}

\textit{\textbf{Деф}} - $\Sigma^*$ наричаме множеството от всички думи в азбуката $\Sigma$.

\textit{\textbf{Деф}} - Казваме, че $L$ е език в азб. $\Sigma$ ако $L \subseteq \Sigma^*$.

\textit{\textbf{Деф}} - Конкатенация на езиците $L_1$, $L_2$ в азбуката $\Sigma$ отбелязваме като $L_1 \circ L_2$ и дефинираме като $L_1 \circ L_2 = \{w_1 \circ w_2 | w_1 \in L_1 $ и $ w_2 \in L_2\}$.

\textit{Пример - $\{li, on\} \circ \{leo, pard\} = \{lileo, lipard, onleo, onpard\}$}.

\textit{\textbf{Деф}} - Обединението на езиците $L_1$ и $L_2$ в азб. $\Sigma$ бележим като $L_1 \cup L_2$ и дефинираме като $L_1 \cup L_2 = \{w | w \in L_1 $ или $ w \in L_2\}$.

\textit{Пример - $\{a, b\} \cup \{pumba, jay, b\} = \{a, b, pumbda, jay, b\}$}.

\textit{\textbf{Деф}} - Степен в език $L$ в азбука $\Sigma$ дефинираме индуктивно като:
\begin{enumerate}
    \item $L^0 = \{\varepsilon\}$
    \item $L^{i+1} = L^i \circ L, i \in \mathbb{N}$
\end{enumerate}}

\textit{\textbf{Деф}} - Звезда на Клини на език $L$ в азбука $\Sigma$ бележим като $L^*$ и дефинираме като $L^* = \bigcup\limits_{i=0}^{\infty} L^{i} = L^0 \cup L^1 \cup L^2 \dots = \{w | w = w_1 \circ \dots \circ w_k, w_0, \dots, w_k \in L $ и $ k \in \mathbb{N}\}$

\textit{Пример - $\{a, b\}^* = \{\varepsilon, a, b, aa, bb, ab, ba, bb, aaa, aab, \dots\}$}.

\textit{\textbf{Деф}} - Допълнение на език $L$ наричаме $L^C = \Sigma^* \textbackslash L$.

\section{Недетерминирани крайни автомати}

\textit{\textbf{Деф}} - Недетерминиран краен автомат (КНА) наричаме петицата $M = (K, \Sigma, \Delta, s, F)$, където:
\begin{itemize}
    \item $K$ - крайна азбука от състояния,
    \item $\Sigma$ - крайна основна азбука,
    \item $\Delta$ - релация на преходите, $\Delta \subseteq K \times (\Sigma \cup \{ \varepsilon \} ) \times K$,
    \item $F$ - мн-во от заключителни състояния, $F \subseteq K$,
    \item $s$ - начално състояние, $s \in K$.
\end{itemize}

\textit{\textbf{Заб} - Всеки КДА $M = (K, \Sigma, \delta, s, F)$ може да се разглежда като КНА $M = (K, \Sigma, G_\delta, s, F)$, където $G_\delta = \{(q, a, q') | \delta(q, a) = q' \}$.}

\textit{\textbf{Деф}} - Конфигурация за КНА $M = (K, \Sigma, \Delta, s, F)$ наричаме всяка двойка $(q, w)$, такава че $q \in K$ и $w \in \Sigma^*$.

\textit{\textbf{Деф}} - Нека $M = (K, \Sigma, \Delta, s, F)$ е КНА. Казваме, че конфигурацията $(q, w)$ се преработва за 1 стъпка в конф. $(q', w') \iff \exists u \in \Sigma \cup \{\varepsilon\}$, такова че $w = uw'$ и $(q, u, q') \in \Delta$. Означаваме като $(q, w) \vdash_M (q', w')$.

\textit{\textbf{Деф}} - $\vdash_M^*$ наричаме рефлексивното и транзитивно затваряне на $\vdash_M$.

\textit{\textbf{Деф}} - Казваме, че $w \in \Sigma^*$ се приема (разпознава) от КНА $M = (K, \Sigma, \Delta, s, F) \iff (s, w) \vdash_M^* (f, \varepsilon)$, където $f \in F$. 

\textit{\textbf{Деф}} - $L(M) = \{w | w \in \Sigma^*: (s, w) \vdash_M^* (f, \varepsilon)$, където $f \in F\}$ се нарича езика, който се разпознава от автомата $M = (K, \Sigma, \Delta, s, F)$.


\end{flushleft}
\end{document}
