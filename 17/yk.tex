\documentclass[fleqn,12pt]{article}

\usepackage[margin=15mm]{geometry}
\usepackage[utf8]{inputenc}
\usepackage[bulgarian]{babel}
\usepackage[unicode]{hyperref}
\usepackage{amsfonts}
\usepackage{amssymb}
\usepackage{indentfirst}
\usepackage{enumitem, hyperref}
\usepackage{blindtext}
\usepackage{multicol}

\usepackage{tikz}
\usepackage{amsmath}
\usepackage{listings}
\usepackage{xcolor}



\title{Управление на качеството на софтуерни приложения. Тестване на софтуер.}
\author{v1.0}
\date{21 юни 2021}


\begin{document}

\maketitle

\tableofcontents

\begin{flushleft}

\section{Осигуряване на качество}
Набор от дейности, проектирани да оценят процеса, спрямо, който се създават и поддържат продуктите.

\section{Качество на софтуера}
Степента, до която система, компонент или процес отговаря на специфицираните изисквания. Степента, до която система, компонент или\\
процес удовлетворява нуждите или очакванията на клиента или потребителя.

\section{Алтернативи за осигуряване на качество}
  \subsection{Софтуерна разработка (ориентирана към процеса)}
  \begin{itemize}
    \item Осигуряване на приемливо ниво на увереност, че софтуерът ще отговаря на:
      \begin{itemize}
        \item Функционалните технически изисквания
        \item Управленските графици и бюджетните изисквания
      \end{itemize}
    \item Иницииране и управление на дейности за повишаване на ефективността на софтуерната разработка и дейностите по осигуряване на качеството.
  \end{itemize}
  \subsection{Софтуерна поддръжка (ориентирана към продукта)}
  \begin{itemize}
    \item Осигуряване на приемливо ниво на увереност, че софтуерните дейности по поддръжка ще отговарят на:
        \begin{itemize}
            \item Функционалните технически изисквания.
            \item Управленските графици и бюджетните изисквания.
        \end{itemize}
    \item Иницииране и управление на дейности за подобрение и повишаване на ефективността на софтуерната поддръжка и дейностите по осигуряване на качеството.
  \end{itemize}
\subsection{Тестови дейности, управление и автоматизация}
  \subsubsection{Тестови дейности}
    \begin{itemize}
        \item Тестово планиране и подготовка -- планиране на ресурси и персонал, определяне на цел, избор на формални модели и техники за тестване,\\
        дефиниране на тестови сценарии, създаване и управление на тестови пакети, дефиниране на тестова процедура.
        \item Изпълнение на тестове и наблюдение:
            \begin{itemize}
                \item Базови стъпки при изпълнение на тестови сценарии
                    \begin{itemize}
                        \item Заделяне на време и ресурси
                        \item Стартиране и изпълнение на тестове и събиране на резултати от изпълнението им
                        \item Проверка на тестовите резултати
                    \end{itemize}
                \item Тестов оракул - средство за проверка на тестов резултат
                \item Тестово измерване - събиране на данни от изпълнението на тестовия сценарии
            \end{itemize}
        \item Анализ и проследяване - анализиране на индивидуални тестови серии и проверка на резултата и идентифициране на повреди
        \item Базов тестов процес - инстанция на процеса по осигуряване на качество
    \end{itemize}
    \subsubsection{Управление}
        \begin{itemize}
            \item Тестване и Инспекция
            \item Тестване и Превенция от грешки - редукция на дефектите, отстранени на етап тестване
            \item Тестване и Формална верификация
            \item Тестване и Стратегии за устойчивост на грешки и ограничаване на повреди - стратегиите се прилагат при критични системи, които трудно се тестват
            \item Организация и управление на екипа за тестване
                \begin{itemize}
                    \item Вертикален модел на организация - организацията е около продукта, като една или повече тестови задачи се асоциират с определен изпълнител
                    \item Хоризонтален модел на организация - за големи организации, един тестов екип изпълнява един тип тестване за всички продукти
                    \item Смесен модел на организация
                \end{itemize}
        \end{itemize}
    \subsubsection{Автоматизация}
        \begin{itemize}
            \item Цел на автоматизацията - освобождаване на тестерите от досадни и повтарящи се задачи и повишаване на тестовата производителност
            \item Избор на софтуерни инструмент за автоматизация, анализ на необходимите ресурси
            \item Възможност за автоматизиране на тестови дейности
        \end{itemize}

\section{Видове тестване}
    \subsection{Тестване с контролни списъци}
        \begin{itemize}
            \item Представлява стартиране на софтуера, наблюдение на поведението и идентифициране на проблемите. Тестването спира когато се изчерпят елементите от списъка с елементи за тестване
            \item Типове контролно списъци
                \begin{itemize}
                    \item Базови - прости списъци от елементи които трябва да се изтестват
                    \item Йерархични - Списъците са разделени на нива, като тези от по-горни нива съдържат списъци от по-ниски нива като елементи
                    \item Смесени
                \end{itemize}
            \item Проблеми и ограничения при контролни
                \begin{itemize}
                    \item Припокриване на елементи в различни контролни списъци
                    \item Трудност при описване на сложни взаимодействия м/у различни компоненти на системата
                \end{itemize}
        \end{itemize}
    \subsection{Тестване с покритие на класове}
    \subsection{Тестване с машина на крайните състояния}
    \subsection{Тестване с класове на еквивалентност}
        \begin{itemize}
            \item Дефиниране на класове на еквивалентност
            \item Избор на един тестов сценарии за всеки клас на еквивалентност
            \item Постигане на пълно покритие на класовете на еквивалентност
            \item Разделяне на класове
                \begin{itemize}
                    \item Формално представяне на клас на еквивалентност - множество от множества със следните свойства:
                        \begin{itemize}
                            \item Подмножествата са взаимоизключващи се
                            \item Подмножествата са взаимно изчерпателни
                        \end{itemize}
                    \item Релации м/у елементите в клас на еквивалентност - рефлексни релации, симетрични релации и транзитивни релации
                \end{itemize}
            \item Типове разделяне на класове
                \begin{itemize}
                    \item Базирано на софтуерни елементи
                    \item Базирано на определени свойства, релации, логически условия
                \end{itemize}
            \item Дърво и таблици за взимане на решения -- може да се разгледа като йерархичен контролен списък. Прилагат се за тестване, базирано на решения, и тестване, базирано на предикати
        \end{itemize}
    \subsection{Разделяне на входния домейн и тестване на границите}
        \begin{itemize}
            \item Анализ на входния домейн - генериране на тестови сценарии посредством присвояване на специфични стойности на входните променливи въз основа на анализ на входния домейн
            \item Тестване с разделяне на входния домейн - покриване на малък брой входни ситуации посредством систематичен избор на определени входни стойности
            \item Характеристики
                \begin{itemize}
                    \item Тестване на входно/изходните зависимости - осигуряване на стойности за всички входни променливи
                    \item Изходните променливи не се специфицират експлицитно
                    \item Прилага се главно за функционално тестване
                    \item Вътрешните детайли, свързани с реализацията могат да се използват за анализ на входните променливи, което е предпоставка за извършване на структурно тестване
                \end{itemize}
            \item Основни стъпки при тестване на домейн
                \begin{itemize}
                    \item Идентифициране на входното пространство и дефиниране на входен домейн - определяне чрез white/black box testing
                    \item Разделяне на входния домейн на поддомейни
                    \item Определяне на границите на поддомейните
                \end{itemize}
            \item Проблеми при разделянето на входния домейн
                \begin{itemize}
                    \item Неопределеност на даден вход - тестваната програма не обработва някои входни стойности
                    \item Противоречивост за даден вход - повреда в системата или производство на различни изходи при един и същ вход
                \end{itemize}
            \item Проблеми с границите на домейните
                \begin{itemize}
                    \item Проблем със затвореността на границите - отворена граница се специфицира като затворена
                    \item Изместване на границата
                    \item Липсваща граница
                    \item Излишна граница
                \end{itemize}
        \end{itemize}
\section{Нива на тестване и приложение на техниките за тестване}
    \subsection{Тестови под-фази}
        \begin{itemize}
            \item Тестване на ниво програмна единица - тестване на малки програмни единици. Използва се най-вече техники за тестване по метода на бялата кутия
            \item Компонентно тестване - тестване на софтуерните компоненти от малка група разработчици. Тества се по метода на бялата кутия, метода на черната кутия. Използва се за тестване на COTS продукти.
            \item Интеграционно тестване - тестване на интерфейси и взаимодействието м/у компоненти при интегрирането им. Управлението на изпълнението на тестовете става чрез машина с крайни състояния
            \item Системно тестване - тестване на външните системни операции като цяло, от гледна точка на клиента. Тази фаза е съответства на фаза от модела на „водопада``. Тестване чрез - машини с крайните състояния, модел на Марков
            \item Тестване, базирано на дефекти - тестване, при което се използват открити или потенциални дефекти. Стратегии за тестване - инжектиране на дефекти и тестване, тестване на мутации
            \item Тестване за приемане на системата (Acceptance testing) - тестване готовността на продукта за доставяне до крайните потребители. Извършва се в края на системното тестване. Техниките за тестване\\
            които се използват - статистическо тестване, базирано на употреба и профили на Муса и Марков
            \item Бета тестване - тестване при контролирана и ограничена доставка на продукта до крайните потребители
            \item Регресионно тестване - използва се за проверка дали софтуерните функции са засегнати от новите версии на продукта. Фокус в/у интеграционно тестване м/у новите и старите компоненти
            \item Диагностично тестване - пресъздаване диагностициране на проблеми, възникнали при клиента
                \begin{itemize}
                    \item Реализация - изпълнение на последователност от свързани тестове
                    \item Прилагат се при инспекция на качеството
                \end{itemize}
        \end{itemize}

\section{Измерване. Метрики - характеристики и класификация. Метрики за качество на софтуерен продукт и софтуерен процес}
    \subsection{Измерване}
    процес, при който в съответствие с определени правила на характеристиките на изследвания обект се съпоставя стойности
        \begin{itemize}
            \item Нива
                \begin{itemize}
                    \item Номинална скала - изброимо множество от категории, стойности, като не се взима предвид количествените съотношения
                    \item Порядкова скала - наредено множество от категории
                    \item Интервална скала - числови стойности, като равномерно нарастват
                    \item Относителна скала - интервална скала + ясно дефинирана мярка 0
                \end{itemize}
            \item Свойства
                \begin{itemize}
                    \item Обективност - получените мерки не зависят от субекта, извършващ измерването
                    \item Надеждност - при повтаряне на измерването при еднакви условия едни и същи резултати се получават
                    \item Валидност - отразяват реално свойствата на измервания обект
                    \item Точност - мярката има необходимата различаваща способност
                \end{itemize}
        \end{itemize}
  \subsection{Дейности}
    \begin{itemize}
        \item Формулиране на метрична система
        \item Събиране на данни
        \item Анализиране
        \item Интерпретиране
        \item Връщане на обратна връзка
    \end{itemize}
\section{Метрики}
    \begin{itemize}
        \item Мярка - число, представящо различните състояния на измерваната характеристика
        \item Функция, чиито входове са софтуерни данни, а изходът е една числова стойност - степента до която даден атрибут влияе в/у качеството на продукта/процеса.
        \item Характеристики
            \begin{itemize}
                \item Общи изисквания
                    \begin{itemize}
                        \item Уместна - измерва атрибут от голямо значение
                      \item Валидна - измерва искания атрибут
                        \item Надеждна - дава еднакви резултати, при еднакви условия
                        \item Взаимно изключваща - не измерва вече измерен атрибут
                   \end{itemize}
                 \item Оперативни изисквания - базира се на текущо използваните системи за събиране на данни
            \end{itemize}
        \item Класификация
            \begin{itemize}
                \item В зависимост от предназначението - оценяване на необходими ресурси, измерване на производителността на разработчиците, прогнозиране на надеждност на продукта
                \item В зависимост от целта на прилагане на метриката - за оценяване, за прогнозиране, за подобряване, за характеризиране
                \item В зависимост от типа на изследвания обект - метрики на продукт, процес, проект
                \item В зависимост от начина на получаване на информацията - регистрационен подход, измервателен подход
            \end{itemize}
    \end{itemize}
\section{Метрики за качество на софтуерен процес}
    \begin{itemize}
        \item Метрики за честота на грешките - размер на софтуера, брой грешки 
        \item Метрики за сериозност на грешката
    \end{itemize}
\section{Метрики за качество на софтуерен продукт}
    \begin{itemize}
        \item Средно време до отказа
        \item Процент грешки
        \item Проблеми идентифицирани от потребителите
        \item Удовлетвореност на потребителите
    \end{itemize}
\end{flushleft}

\end{document}
