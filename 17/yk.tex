\documentclass[fleqn,12pt]{article}

\usepackage[margin=15mm]{geometry}
\usepackage[utf8]{inputenc}
\usepackage[bulgarian]{babel}
\usepackage[unicode]{hyperref}
\usepackage{amsfonts}
\usepackage{amssymb}
\usepackage{indentfirst}
\usepackage{enumitem, hyperref}
\usepackage{blindtext}
\usepackage{multicol}

\usepackage{tikz}
\usepackage{amsmath}
\usepackage{listings}
\usepackage{xcolor}



\title{Тема 17 \\Управление на качеството на софтуерни приложения. Тестване на софтуер.}
\author{v1.0}
\date{21 юни 2021}


\begin{document}

\maketitle

\tableofcontents
\pagebreak

\section{Осигуряване на качество}
Набор от дейности (например тестване), проектирани да \textbf{оценят процеса}, спрямо който се създават и поддържат продуктите.
Целта е да се докаже качеството, т.е. \textbf{правилното поведение}, както и да се \textbf{установяват} и \textbf{отстраняват} проблеми.

\subsection{Качество на софтуера}

\textbf{\textit{Качество}} дефинираме като степента, до която система, компонент или процес \textit{отговаря на специфицираните изисквания} и \textit{удовлетворява нуждите или очакванията на заинтересованите лица}.
\bigbreak
Алтернативи за осигуряване на качеството са \textbf{софтуерна разработка (ориентирана към процеса)} и \textbf{софтуерна поддръжка (ориентирана към продукта)}.

\subsection{Тестови дейности, управление и автоматизация}
\subsubsection{Тестови дейности}

Основните дейности са:
\begin{enumerate}
    \item \textbf{\textit{Планиране на тестовете}} - дефиниране на тестова цел, планиране на ресурси и персонал и избор на формални модели и техники за тестване.
    \item \textbf{\textit{Конструиране на тестов модел}} - идентифициране на източник на информация и събиране на данни, както и анализ и създаване на модела.
    \item \textbf{\textit{Генериране на тестови сценарии от модела}}, където:
    \begin{itemize}
        \item \textbf{Тестови сценарий (Test case)} дефинираме като колекция от елементи и свързана с тях информация, осигуряващи изпълнението на тест или тестова серия.
        \item \textbf{Тестова серия (Test run)} е динамична единица от специфични тестови дейности в общата тестова последователност върху избран тестов обект.
        На прост език изпълнение на няколко теста.
    \end{itemize}
    \item \textbf{\textit{Създаване и управление на тестови пакети (test cases)}}, където:
    \begin{itemize}
        \item \textbf{Тестов пакет (Test case)} наричаме колеция от отделни тестови сценарии, които се стартират в тестова последователност докато не се удовлетвори даден критерий за спиране.
    \end{itemize}
    \item \textbf{\textit{Подготовка на тестова процедура}}, която определя последователността и превключването на тестовите серии.
    \item \textbf{\textit{Изпълнение на тестове и наблюдение}}, което се състои от:
    \begin{itemize}
        \item Заделяне на време и ресурси.
        \item Стартиране и изпълнение на тестове и събиране на резултати от изпълнението им.
        \item Проверка на тестовите резултати чрез тестови оракули.
    \end{itemize}
    \item \textbf{\textit{Анализ и проследяване}} –- анализиране на индивидуални тестови серии и проверка на резултатите от тях с цел идентифициране на повреди.
\end{enumerate}

\subsubsection{Управление}
Съществуват следните видове организация и управление на екипа за тестване:
\begin{itemize}
    \item \textbf{\textit{Вертикален модел на организация}} -- организацията е около продукта, като една или повече тестови задачи се асоциират с определен изпълнител.
    \item \textbf{\textit{Хоризонтален модел на организация}} -- за големи организации, като един тестов екип изпълнява един тип тестване за всички продукти.
    \item \textbf{\textit{Смесен модел на организация}}, където:
    \begin{itemize}
        \item ниските нива на тестване се изпълняват от екипите отговорящи за съответните проекти.
        \item системното тестване се споделя между подобни проекти.
        \item $\exists$ обща поддръжка на проекта от осигурена централна единица.
    \end{itemize}
\end{itemize}

\subsubsection{Автоматизация}

Целта на \textbf{\textit{автоматизацията}} е да освободи тестерите от досадни и повтарящи се задачи и да повиши на тестовата производителност.
Важни въпроси касаещи автоматизацията са възможността за автоматизация на специфични тестови сценарии, изборът на достъпни софтуерни инструменти за автоматизация и крайната цена на автоматизацията.

\begin{center}
\begin{tabular}{ |c|c| }
    \hline
    Тестова дейност & Възможност за автоматизация \\
    \hline
    \multicolumn{2}{|c|}{Изпълнение} \\
    \hline
    Изпълнение на тестове & Висока \\
    \hline
    \multicolumn{2}{|c|}{Планиране и подготовка} \\
    \hline
    Генериране на тестови сценарии & Висока \\
    Създаване на формални модели & Средна \\
    Подготовка на тестови сценарии & Средна \\
    Цялостно планиране на тестовия процес & Ниска \\
    Планиране на тестовата процедура & Ниска \\
    \hline
    \multicolumn{2}{|c|}{Анализ и проследяване} \\
    \hline
    Анализ на надеждността & Висока \\
    Анализ на тестовото покритие & Висока \\
    Действия за подобряване на продукта & Ниска \\
    \hline
\end{tabular}
\end{center}

\section{Видове тестване}
\subsection{Тестване с контролни списъци}

\textbf{\textit{Ad-hoc}} тестването представлява стартиране на софтуера, наблюдение на поведението и идентифициране на проблемите.
Тестването с \textbf{\textit{контролни списъци}} представлява създаване неформални \textbf{TODO} списъци за проследяване на изтестваните по \textbf{ad-hoc} начин елементи.
\bigbreak

Съществуват следните типове контролни списъци:
\begin{itemize}
    \item \textbf{Базови} - прости списъци от елементи които трябва да се изтестват.
    \item \textbf{Йерархични} - списъците са разделени на нива, като елементите от по-горни нива съдържат списъци от по-ниски нива.
    \item \textbf{Комбинирани (Многомерни)} - списъците са многомерни, като всеки контролен списък се обхожда за всички елементи от останалите контролни списъци.
    \item \textbf{Смесени} - съчетание на йерархичните и комбинираните контролни списъци.
\end{itemize}

При контролните списъци се срещат следните проблеми и ограничения:
\begin{itemize}
    \item \textbf{Трудности при покриване на всички функционалности} от различни гледни точки и нива на гранулярност.
    \item \textbf{Припокриване на елементи} в различни контролни списъци.
    \item \textbf{Трудност при описване на сложни взаимодействия} м/у различни компоненти на системата.
\end{itemize}

\subsection{Тестване с класове на еквивалентност}

Множествата $C_1, \dots, C_n, n \in \mathbb{N}$ наричаме класове на еквивалентност на релацията на еквивалентност $~$ над множеството $S$ ако:
\begin{itemize}
    \item $\forall i, j, i \neq j$ е изпълнено, че $C_i \cap C_j = \emptyset$
    \item $\bigcup\limits_{i=1}^{n} C_i = S$
    \item $C_i = \{x \in S | x ~ a\}$ където $a \in C_i$.
\end{itemize}

Практическата им импликация при тестването е, че входа на тестовите сценарии се разделя на класове на еквивалентност спрямо очаквания тип на взаимодействие.
Може да се раздели на следните стъпки:
\begin{enumerate}
    \item Дефиниране на класове на еквивалентност.
    \item Избор на един тестов сценарии за всеки клас на еквивалентност.
    \item Постигане на пълно покритие на класовете на еквивалентност.
\end{enumerate}

Типовете разделяне на класове на еквивалентност са:
\begin{itemize}
    \item Базирано на \textbf{софтуерни елементи}.
    \item Базирано на \textbf{определени свойства, релации, логически условия}.
    \item Базирано на комбинация от горните две.
\end{itemize}

\textbf{\textit{Дърво и таблици за взимане на решения}} - може да се разгледат като йерархичен контролен списък.
Прилагат се за тестване, базирано на решения, и тестване, базирано на предикати.

\subsection{Разделяне на входния домейн и тестване на границите}

Генерираме на тестови сценарии посредством присвояване на специфични стойности на входните променливи въз основа на \textbf{\textit{анализ на входния домейн}}.
\bigbreak
\textbf{\textit{Тестване с разделяне на входния домейн}} наричаме покриване на малък брой входни ситуации посредством систематичен избор на определени входни стойности.
Характеризира се с:
\begin{itemize}
    \item \textit{Тестване на входно/изходните зависимости} - осигуряване на стойности за всички входни променливи.
    \item \textit{Изходните променливи не се специфицират експлицитно} - проверяваме дали изход се получава от вход.
    \item Прилага се главно за \textit{функционално тестване}.
    \item Детайлите по реализацията могат да се използват за анализ на входните променливи, което е предпоставка за извършване на \textit{структурно тестване}.
\end{itemize}

Разделя се на следните \textbf{стъпки}:
\begin{enumerate}
    \item \textit{Идентифициране} на входното пространство и \textit{дефиниране} на входен домейн.
    \item \textit{Разделяне на входния домейн} на поддомейни.
    \item \textit{Анализ на поддомейните} с цел определяне на границите им по всички измерения.
    \item \textit{Избор на всички тестови точки (сценарии)} покриващи поддомейните.
    \item \textit{Тестване с избраните тестови точки, решаване на проблеми и анализиране на резултатите}.
\end{enumerate}

\textbf{Проблеми} при разделянето на входния домейн са:
\begin{itemize}
    \item \textbf{неопределеност на даден вход} - тестваната програма не обработва някои входни стойности.
    \item \textbf{противоречивост за даден вход} - повреда в системата или производство на различни изходи при един и същ вход.
\end{itemize}

\textbf{Проблеми} при определянето на границите на поддомейните са:
\begin{itemize}
    \item \textbf{Проблем със затвореността на границите} - отворена граница се специфицира като затворена
    \item \textbf{Изместване на границата}
    \item \textbf{Липсваща граница}
    \item \textbf{Излишна граница}
\end{itemize}

\subsection{Други}

Други начини за тестване са чрез:
\begin{itemize}
    \item тестване с \textbf{покритие на класове}.
    \item тестване с \textbf{машина на крайните състояния (автомат)}.
    \item тестване с \textbf{граф на данновия поток}.
\end{itemize}


\section{Нива на тестване и приложение на техниките за тестване}
\subsection{Тестови под-фази}

Откъм йерархична гледна точка тестовете са:
\begin{enumerate}
    \item \textbf{\textit{Тестване на ниво програмна единица (Unit testing)}} - тестване на малки програмни единици, като функции и класове, по метода на бялата кутия.
    Представлява най-ниското ниво на абстракция в йерархията на тестовете.
    \item \textbf{\textit{Компонентно тестване (Component testing)}} - тестване на софтуерните компоненти, като например библиотечни пакети, като се допуска тестване, както по метода на бялата, така и на черната кутия.
    \item \textbf{\textit{Интеграционно тестване (Integration testing)}} - тестване на интерфейси и взаимодействието м/у компоненти при интегрирането им, както по метода на бялата, така и на черната кутия.
    Управлението на изпълнението на тестовете става чрез машина на крайни състояния.
    \item \textbf{\textit{Системно тестване (System testing)}} - тестване на външните системни операции като цяло, от гледна точка на клиента, т.е. по метода на черната кутия.
    Тества се чрез машини на крайните състояния и модел на Марков.
    \item \textbf{\textit{Тестване за приемане на системата (Acceptance testing)}} - тестване готовността на продукта за доставяне до крайните потребители.
    Извършва се срещу среда където са разгърнати всички компоненти на системата като настъпва в края на системното тестване.
    При процеси като \textbf{scrum/kanban} се извършва поне веднъж в края на всеки \textbf{sprint}, като част от тестовия цикъл, преди да се направи \textbf{release}.
    Техниките за тестване които се използват са статистическо тестване, базирано на употреба и профили на Муса и Марков.
\end{enumerate}

Някои други видове тестове са:
\begin{itemize}
    \item \textbf{\textit{Тестване, базирано на дефекти}} - тестване, при което се използват открити или потенциални дефекти.
    Стратегии за тестване са инжектиране на дефекти и тестване на мутации.
    \item \textbf{\textit{Бета тестване}} - тестване при контролирана и ограничена доставка на продукта до крайните потребители.
    \item \textbf{\textit{Регресионно тестване}} - използва се за проверка дали съществуващите софтуерните функции са засегнати от новите версии на продукта.
    Има фокус над интеграционно тестване м/у новите и старите компоненти.
    \item \textbf{\textit{Диагностично тестване}} - пресъздаване и диагностициране на проблеми, възникнали при клиента.
    Реализира се чрез изпълнение на последователност от свързани тестове като се прилага при инспекция на качеството.
\end{itemize}

\section{Измерване и метрики}

\subsection{Дефиниции}

\textbf{\textit{Измерване}} наричаме процес, при който в съответствие с определени правила, на характеристиките на изследвания обект се съпоставят стойности.
\bigbreak

\textbf{\textit{Мярка}} наричаме стойност съпоставена на някое измерване.
Следователно тя показва състоянието на измерваната характеристика.
\bigbreak

\textbf{\textit{Метрика}} на дадена характеристика, наричаме функцията съпоставяща нейните състояния на мярките им.

\subsection{Измерване}

Нивата на измерване са:
\begin{itemize}
    \item \textbf{Номинална (Nominal) скала} - изброимо множество от категории/стойности, които се присвояват, без да се взимат предвид количествени съотношения.
    \item \textbf{Порядкова (Ordinal) скала} - наредено множество от категории.
    \item \textbf{Интервална (Interval) скала} - числови стойности, като разликата между всеки две последователни е една и съща.
    Могат да се прилагат математически операции.
    \item \textbf{Относителна (Ratio) скала} - интервална скала, където съществува ясно дефинирана мярка 0.
\end{itemize}

Важни свойства на измерването са:
\begin{itemize}
    \item \textbf{Обективност} - получените мерки не зависят от субекта, извършващ измерването.
    \item \textbf{Надеждност (Еднозначност)} - еднакви резултати при еднакви условия.
    \item \textbf{Валидност} - отразяват реално свойствата на измервания обект.
    \item \textbf{Точност (Accuracy)} - мярката има необходимата различаваща способност.
\end{itemize}

Дейности при измерването са:
\begin{enumerate}
    \item Формулиране на метрична система
    \item Събиране на данни
    \item Анализиране
    \item Интерпретиране
    \item Връщане на обратна връзка
\end{enumerate}

\subsection{Xарактеристики на метрики}

Характеристики
\begin{itemize}
    \item Общи изисквания
        \begin{itemize}
            \item \textbf{Надеждна (Еднозначна)} - еднакви резултати при еднакви условия.
            \item \textbf{Валидна} - измерва искания атрибут.
            \item \textbf{Уместна} - измерва значим атрибут.
            \item \textbf{Взаимно изключваща} - не измерва вече измерен атрибут.
        \end{itemize}
    \item Оперативни изисквания:
    \begin{itemize}
        \item Неподатлива на предубедени намеси от заинтересовани страни
        \item Не изисква независимо събиране на данни
    \end{itemize}
\end{itemize}

\subsection{Класификация на метрики}

Класификация
\begin{itemize}
    \item В зависимост от \textbf{предназначението} - оценяване на необходими ресурси, производителността на разработчиците, прогнозиране на надеждност на продукта и други.
    \item В зависимост от \textbf{целта на прилагане на метриката} - за оценяване, за прогнозиране, за debug-ване и други.
    \item В зависимост от \textbf{типа на изследвания обект} - метрики на софтуерен продукт, процес, проект.
    \item В зависимост от \textbf{начина на получаване на информацията} - регистрационен подход, измервателен подход.
\end{itemize}

\subsection{Метрики за качество на софтуерен процес}

Те са:
\begin{itemize}
    \item Метрики за \textbf{честота на грешките} - размер на софтуера, брой грешки.
    \item Метрики за \textbf{сериозност на грешката} - средна сериозност на грешките в кода, средна сериозност на грешките в разработката.
\end{itemize}

\subsection{Метрики за качество на софтуерен продукт}

Те са:
\begin{itemize}
    \item Средно време до отказ (\textbf{MTTF}).
    \item Процент грешки (\textbf{Defect rate}).
    \item Брой проблеми идентифицирани от потребителите (\textbf{Customer problems}).
    \item Удовлетвореност на потребителите (\textbf{Customer satisfaction}.)
\end{itemize}

\end{document}
