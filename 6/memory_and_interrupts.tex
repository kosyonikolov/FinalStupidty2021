
\documentclass[fleqn,12pt]{article}

\usepackage[margin=15mm]{geometry}
\usepackage[utf8]{inputenc}
\usepackage[bulgarian]{babel}
\usepackage[unicode]{hyperref}
\usepackage{amsfonts}
\usepackage{amssymb}
\usepackage{enumitem, hyperref}
\usepackage{upgreek}
\usepackage{indentfirst}

\usepackage{amsmath}
\DeclareMathOperator{\cotg}{cotg}
\DeclareMathOperator{\LCS}{LCS}
\DeclareMathOperator{\longer}{longer}

\title{Структура и йерархия на паметта. Сегментна и странична преадресация.
Система за прекъсване – приоритети и обслужване.}

\author{v0.1}
\date{21 юни 2021}

\begin{document}

\maketitle

\tableofcontents

\section{Структура и йерархия на паметта}

\subsection{Основна памет}
TODO

\subsection{Кеш памет}
TODO

\subsection{Виртуална памет}
TODO

\section{Сегментна преадресация}
TODO

\subsection{Идея и основание}
TODO

\subsection{Сегментен селектор}
TODO

\subsection{Сегментен дескриптор}
TODO

\subsection{Сегментни таблици и регистри}
TODO

\section{Странична преадресация}
TODO

\subsection{Идея и основание}
TODO

\subsection{Каталог на страниците}
TODO

\subsection{Описател на страница}
TODO

\subsection{Стратегии на подмяна на страниците}
TODO

\section{Прекъсвания}
TODO

\subsection{Структура и обработка}
TODO

\subsection{Типове прекъсвания}
TODO

\subsection{Конкурентност и приоритети}
TODO

\subsection{Обслужване на прекъсванията}
TODO

\subsection{Контролери на прекъсванията}
TODO

\end{document}
